%!TEX TS-program = lualatex
%!TEX encoding = UTF-8 Unicode

\documentclass[12pt]{article}
%\usepackage{graphicx}
%	\graphicspath{{/Users/goby/Pictures/teach/153/lab/}} % set of paths to search for images

\usepackage{geometry}
\geometry{letterpaper}                   
\geometry{bottom=1in, left=1.5in}
%\geometry{landscape}                % Activate for for rotated page geometry
\usepackage[parfill]{parskip}    % Activate to begin paragraphs with an empty line rather than an indent
%\usepackage{amssymb}
%\usepackage{mathtools}
%	\everymath{\displaystyle}

%\pagenumbering{gobble}

\usepackage{fontspec}
\setmainfont[Ligatures={Common,TeX}, BoldFont={* Bold}, ItalicFont={* Italic}, Numbers={Proportional, OldStyle}]{Linux Libertine O}
\setsansfont[Scale=MatchLowercase,Ligatures=TeX, Numbers=OldStyle]{Linux Biolinum O}
%\setmonofont[Scale=MatchLowercase]{Inconsolata}
\usepackage{microtype}

% This defines \amper for the fancy ampersand
% to be used in the header. See
% https://tex.stackexchange.com/a/58185/39194
\usepackage{xspace}
\newfontfamily\amperfont[Style=Alternate]{Linux Libertine O}    
\makeatletter
\DeclareRobustCommand{\amper}{{\amperfont\ifx\f@shape\scname\smaller[1.2]\fi\&}\xspace}
\makeatother

\usepackage{unicode-math}
\setmathfont[Scale=MatchLowercase]{Asana-Math.otf}
%\setmathfont{XITS Math}

% To define fonts for particular uses within a document. For example, 
% This sets the Libertine font to use tabular number format for tables.
%\newfontfamily{\tablenumbers}[Numbers={Monospaced}]{Linux Libertine O}
%\newfontfamily{\libertinedisplay}{Linux Libertine Display O}


%\usepackage{booktabs}
%\usepackage{multicol}
%\usepackage{tabularx}
%\usepackage{longtable}
%\usepackage{siunitx}
%\usepackage[justification=raggedright, singlelinecheck=off]{caption}
%\captionsetup{labelsep=period} % Removes colon following figure / table number.
%\captionsetup{tablewithin=none}  % Sequential numbering of tables and figures instead of
%\captionsetup{figurewithin=none} % resetting numbers within each chapter (Intro, M&M, etc.)
%\captionsetup[table]{skip=0pt}

\usepackage{array}
\newcolumntype{L}[1]{>{\raggedright\let\newline\\\arraybackslash\hspace{0pt}}p{#1}}
\newcolumntype{C}[1]{>{\centering\let\newline\\\arraybackslash\hspace{0pt}}p{#1}}
\newcolumntype{R}[1]{>{\raggedleft\let\newline\\\arraybackslash\hspace{0pt}}p{#1}}

\usepackage{enumitem}
%\usepackage{hyperref}
%\usepackage{placeins} %PRovides \FloatBarrier to flush all floats before a certain point.
%\usepackage{hanging}
%\usepackage{color}
%\usepackage{calc}

%\usepackage{titling}
%\setlength{\droptitle}{-60pt}
%\posttitle{\par\end{center}}
%\predate{}\postdate{}

\usepackage[sc]{titlesec}


\usepackage{fancyhdr}
\fancyhf{}
\pagestyle{fancy}
\lhead{}
\chead{}
\rhead{\footnotesize pg. \thepage }
\renewcommand{\headrulewidth}{0.4pt}

\fancypagestyle{plain}{%
	\fancyhf{}
	\lhead{\textsc{zo} 478 / 678: Ichthyology}
	\rhead{Annotated Bibliography}
	\renewcommand{\headrulewidth}{0pt}
}
	

\begin{document}
%\maketitle
\thispagestyle{plain}


\subsection*{Annotated bibliography for your life history paper (50 points)}

Your ecological life history must be based on the scientific literature. This assignment is intended to get you to start searching the literature and summarizing the papers that you find. This will help you develop and write your life history report. 

\subsubsection*{Requirements}

\begin{enumerate}[leftmargin=*]

\item You must have 10 unique scientific publications or or edited chapters. I will allow you to use a limited number of technical reports. You must obtain copies of these papers because you will be required to upload the \textsc{pdf} files to the drop box along with your completed annotated bibliography.

\item Each paper must be properly cited following the formats given to you in the handout that outlines the requirements for your ecological life history paper. That handout was given to you previously and is available on the course Moodle page.

\item You must type a \emph{thorough} single-spaced one paragraph summary of each paper. Two or three sentences is not a paragraph. You must have a second paragraph that describes how how you intend to use information from that study in your life history paper. I imagine you should have about two citations and summaries per page. 

\item Each summary is worth five points. This is a significant chunk of the total points for your life history paper so you should take this assignment seriously. I will.

\end{enumerate}

\subsubsection*{Other information}

Your summary should focus on the results and discussion as they relate to your plan. Do not simply paraphrase or summarize the abstract. I will grade harshly if I suspect this is the case. Be professional. Read the paper thoroughly and summarize it appropriately. If you wish, during the first week of the assignment, you may send me one paper with your summary and intent. I will let you know if you are on the right track. You must do this early. I will not do this after Monday, 29 October, 5pm.

\textsc{Due date:} Wednesday, 7 November, 5:00 pm. \emph{This due date supercedes the date given in the outline handout.} Upload your completed assignments and the \textsc{pdf} files to the drop box available on the course website. See the syllabus for information on late assignments.

\vspace*{\baselineskip}

Please let me know as soon as possible if you have any questions. If I missed an important detail, I need to let other students know as soon as possible.


\end{document}  