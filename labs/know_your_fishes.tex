%!TEX TS-program = lualatex
%!TEX encoding = UTF-8 Unicode

\documentclass[11pt, hidelinks]{exam}
\usepackage{graphicx}
\graphicspath{{/Users/goby/Pictures/teach/434/handouts/}
	{img/}} % set of paths to search for images

\usepackage{geometry}
\geometry{letterpaper, left=1.5in, bottom=1in}                   
%\geometry{landscape}                % Activate for for rotated page geometry
\usepackage[parfill]{parskip}    % Activate to begin paragraphs with an empty line rather than an indent
\usepackage{amssymb, amsmath}
\usepackage{mathtools}
\everymath{\displaystyle}

\usepackage{fontspec}
\setmainfont[Ligatures={TeX}, BoldFont={* Bold}, ItalicFont={* Italic}, BoldItalicFont={* BoldItalic}, Numbers={OldStyle}]{Linux Libertine O}
\setsansfont[Scale=MatchLowercase,Ligatures=TeX]{Linux Biolinum O}
\setmonofont[Scale=MatchLowercase]{Linux Libertine Mono O}
\usepackage{microtype}

% This defines \amper for the fancy ampersand
% to be used in the header. See
% https://tex.stackexchange.com/a/58185/39194
\usepackage{xspace}
\newfontfamily\amperfont[Style=Alternate]{Linux Libertine O}    
\makeatletter
\DeclareRobustCommand{\amper}{{\amperfont\ifx\f@shape\scname\smaller[1.2]\fi\&}\xspace}
\makeatother

% To define fonts for particular uses within a document. For example, 
% This sets the Libertine font to use tabular number format for tables.
%\newfontfamily{\tablenumbers}[Numbers={Monospaced}]{Linux Libertine O}
% \newfontfamily{\libertinedisplay}{Linux Libertine Display O}

\usepackage{booktabs}
\usepackage{multicol}
\usepackage[normalem]{ulem}

\usepackage{longtable}
%\usepackage{siunitx}
\usepackage{array}
\newcolumntype{L}[1]{>{\raggedright\let\newline\\\arraybackslash\hspace{0pt}}p{#1}}
\newcolumntype{C}[1]{>{\centering\let\newline\\\arraybackslash\hspace{0pt}}p{#1}}
\newcolumntype{R}[1]{>{\raggedleft\let\newline\\\arraybackslash\hspace{0pt}}p{#1}}

\usepackage{enumitem}
\usepackage{hyperref}
%\usepackage{placeins} %PRovides \FloatBarrier to flush all floats before a certain point.
\usepackage{hanging}

\usepackage{pdflscape}

\usepackage[sc]{titlesec}

\pagestyle{headandfoot}
\firstpageheader{\textsc{zo}\,478/678 Ichthyology}{}{}
\runningheader{}{}{\footnotesize{pg. \thepage}}
\footer{}{}{}
\runningheadrule

%\newcommand{\VSpace}{\vspace{\baselineskip}}

\title{Cheat Sheet: Cyprinidae, Catostomidae, Ictaluridae}
\author{ZO 478 / 678}
\date{}                                           % Activate to display a given date or no date

\begin{document}

\section*{Cheat Sheet: Cyprinidae, Catostomidae, Ictaluridae}

Much of the following information was put together by Dr. David J. Eisenhour at Morehead State University in Kentucky.  I modified it somewhat to suit our purposes and added many species not in Kentucky. Careful comparison of the characters described below will help you to distinguish among some of the Cyprinidae, Catostomidae and Ictaluridae in our region.

\subsection*{Cyprinidae\,—\,minnows and  carps}

\noindent\begin{enumerate}

\item \textit{Campostoma oligolepis/pullum} - largescale stoneroller and central stoneroller:  Both are unique in having a cartilaginous lower lip which they use to scrape algae from rocks.  Some of the most common minnows over most of Missouri. You will not have to distinguish between them.

\item \textit{Cyprinella galactura}\,—\,whitetail shiner: Note the white patch at the base of the caudal fin. Tends to have a dark blotch on the posterior part of the dorsal fin.  All \textit{Cyprinella} shiners tend to be laterally compressed shiners with melanophores forming diamond-like pigments patterns extending below the lateral stripe. The diamonds clearly outline the scales.

\item \textit{Cyprinella lutrensa}\,—\,red shiner: Lacks the dark blotch in the posterior dorsal fin. Rather plain but shorter and deeper bodied than \textit{C. whipplei}.

\item \textit{Cyprinella whipplei} –steelcolor shiner:  Typically with a dark blotch on the posterior part of dorsal fin and lacking a white band or dark spot at the base of the caudal fin.  Longer and not as deep bodied as \textit{C. lutrensa}.

\item \textit{Cyprinella venusta}\,—\,blacktail shiner. Only \textit{Cyprinella} with a large, dark blotch at the base of the caudal fin. Only shiner on your list of fishes to know with a large, dark blotch at the base of the caudal fin.

\item \textit{Cyprinus carpio}\,—\,common carp:  A large, exotic species now common over most of Missouri.  Deep-bodied fish, similar to carpsuckers in shape.  Has barbels in the corners of its mouth.

\item \textit{Hybognathus nuchalis}\,—\,Mississippi silvery minnow: This rather plain minnow has a coiled gut, a black peritoneum, and a distinctive groove extending from the angle the jaw well up on the snout.  Also note the rather pointed fins. Fairly common in big rivers like the Mississippi River. We tend to catch them in the Mississippi River.

\item \textit{Luxilus chrysocephalus}\,—\,striped shiner: This is a rather large, deep-bodied species with stripes that converge on the back behind the dorsal fin.  In addition, it has rather deep anterior lateral-line scales; twice as deep as long, which is characteristic of the genus \textit{Luxilus}. Thick pre-dorsal stripe.

\item \textit{Luxilus zonatus}\,—\,bleeding shiner: Largish eyes, rather large, deep-bodied species. Lacks the converging stripes on the back behind the dorsal fin. Dark, wide lateral stripe that extends forward on to snout. Thick pre-dorsal stripe.

\item \textit{Lythrurus umbratilus}\,—\,redfin shiner:  Dorsal fin origin behing the pelvic fin origin and crowded predorsal scales (scales smaller in front of the dorsal fin than behind the dorsal fin).  In addition, has a small dark blotch at the base of the dorsal fin. Anal fin longer than many other shiners.

\item \textit{Macrhybopsis aestivalis}\,—\,speckled chub: Slender fish with inferior mouth, barbels at the corner of the mouth and prominent black spots scattered over the body.

\item \textit{Nocomis biguttatus}\,—\,hornyhead chub: Large, cylindrical minnows with small barbels at the corner of the mouth, a slightly subterminal mouth and larger scales than \textit{Semotilus atromaculatus}, which it superficially resembles. Small specimens tend to have a wide, faint lateral stripe but it is lost in larger, older individuals.  One of the largest of Missouri’s minnows. In fresh specimens the caudal fin is often red or orange.

\item \textit{Notropis amblops}\,—\,bigeye chub: Subterminal mouth; largish eyes that seem to look upward; tiny barbel at the corner of the mouth; dark lateral stripe below a pale stripe.

\item \textit{Notropis atherinoides/N. rubellus}\,—\,emerald shiner and rosyface shiner: Long slender minnows with dorsal fin origins well behind pelvic fin origins and 9-11 anal fin rays. \textit{N. atherinoides} tends to be restricted to larger rivers, like the Mississippi River where we collected it. 
\textit{N. rubellus} is widespread in medium to small streams, such as the Whitewater River.  These two can be separated by the presence (\textit{N. atherinoides}) or absence (\textit{N. rubellus}) of dusky pigment sprinkled on the chin. \textbf{\textit{N. rubellus} is now called \textit{N. percobromus}.}

\item \textit{Notropis boops}\,—\,bigeye shiner: This shiner has a large eyes and a large mouth. Has a dark lateral stripe that extends forward on to the snout. Above the dark lateral stripe is a pale lateral stripe that occurs because the scales lack pigment around the edge. Eight anal fin rays.

\item \textit{Notropis nublilus}\,—\,Ozark minnow: slender minnow with moderately-sized eyes (not near as large as \textit{N. boops}). Dusky lateral stripe extends forward on to snout but without the pale stripe above it as seen in \textit{N. boops}.  Distinct pre-dorsal stripe.

\item \textit{Notropis telescopus}\,—\,telescope shiner.  This small minnow has large eyes and faint stripes converging behind the dorsal fin.  In addition, it has marginal and submarginal dark bands on the posterior edge of each scale, separated by a narrow area nearly devoid of pigment.  About 10 anal rays. 

\item \textit{Notropis texanus}\,—\,weed shiner: Superficially similar to \textit{N. boops} (dark and pale lateral stripes) but has much smaller eyes and only 7 anal fin rays. 

\item \textit{Notropis wickliffi/N. volucellus}\,—\,channel shiner and mimic shiner: Both are very plain, small minnows with a small mouth.  You do not have to distinguish between these two.

\item \textit{Phoxinus erythrogaster}\,—\,southern redbelly dace:  This is one of the most beautiful of Missouri’s fishes.  It is also quite hardy in the aquarium.  Confined to springs and spring-fed streams.  Note the tiny scales (not visible to the naked eye), dark peritoneum, coiled gut, and small mouth. 

\newpage

\item \textit{Pimephales notatus}\,—\,bluntnose minnow:  All \textit{Pimephales} have a broad back with crowded predorsal scales (smaller anteriorly than posteriorly).  This species has a black peritoneum, a caudal spot and a prominent lateral stripe.

\item \textit{Pimephales vigilax}\,—\,bullhead minnow:  Seems to replace \textit{P. notatus} in larger streams. We typically collect \textit{P. notatus} in the Whitewater and \textit{P. vigilax} in the Mississippi River. Note the silvery/white peritoneum and caudal spot.  \textit{P. vigilax} has small dark crescent marks between nostril and mouth, absent in \textit{P. notatus}. Also note large black spot at front of dorsal fin.

\item \textit{Semotilus atromaculatus}\,—\,creek chub:  A rather large, common minnow in small streams.  Note the relatively small scales, especially in front of the dorsal fin, a nice predorsal spot and basicaudal spot, a large terminal mouth and a tiny, flaplike barbel in the groove above the upper lip.
\end{enumerate}


\subsection*{Catostomidae\,—\,suckers}

Suckers differ from native species of minnows in having more dorsal rays (10 or greater) and a more posteriorly placed anal fin.  All of them tend to have subterminal or inferior mouths with fleshy lips.

\begin{enumerate}
\item \textit{Carpiodes} spp.\,—\,carpsuckers:  These large, deep-bodied suckers have relatively long dorsal fins.  Distinguishing among the different species can be tricky.  No barbels at corners of mouth so don’t confuse with \textit{Cyprinus carpio}, which is a cyprinid.  You only need to know these to genus.

\item \textit{Cycleptus elongatus}\,—\,blue sucker: Check out the tiny head!  Also note the pecular caudal fin pigmentation and papillose lips.  This fish occupies fast current in medium to large, undisturbed rivers.  Not on your list to know but we occasionally collect a specimen from the Mississippi River.

\item \textit{Hypentelium nigricans}\,—\,northern hog sucker:  Our most common sucker in small to medium streams.  Note the box-like head and bold dorsal saddles.  No other sucker looks like it.

\item \textit{Moxostoma duquesnei/erythrurum}\,—\,black redhorse and golden redhorse:  These long, slender suckers are very similar.  They are very plain, with little in the way of distinguishing marks.  Both have a short dorsal fin, elongate body, evenly sized scales, and a complete lateral line.  Many have red or orange in the caudal fin.  The major difference is lateral scale count and relative proportions of the caudal peduncle. \textit{M. duquesnei} has 44--47 lateral line scales and a proportionately longer, slenderer caudal peduncle. \textit{M. erythrurum} has 39--42 lateral line scales and a proportionately shorter deeper caudal peduncle. See couplet 14 on page 174 of \textit{Fishes of Missouri}.
\end{enumerate}


\subsection*{Ictaluridae\,—\,catfishes}

The three genera of catfishes listed here are easy to distinguish.  \textit{Ictalurus} has a free adipose fin and forked tail. \textit{Ameiurus} has a free adipose fin and an unforked tail.\textit{Noturus} has an adipose fin that merges with an unforked tail.

\begin{enumerate}

\item \textit{Ameiurus natalis}\,—\,yellow bullhead:  Note the white chin barbels.  Probably our most common bullhead around here, typically inhabiting pools and debris areas of small streams. \textit{A. melas}, which you are not required to know, is similar but has dusky chin barbels.

\item \textit{Ictalurus furcatus} - blue catfish:  This is our largest species of catfish in the state, primarily confined to large rivers and reservoirs.  Note straight-edged anal fin and forked caudal fin. The eyes are smaller than those of \textit{I. punctatus} (compare side by side).

\item \textit{Ictalurus punctatus}\,—\,channel catfish:  This common species has a forked caudal fin, but has has a shorter, more rounded anal fin compared to \textit{I. furcatus}.  This species is typically spotted when young…however, some juvenile \textit{I. punctatus} lack spots and rarely \textit{I. furcatus} young have spots. The eyes are larger than those of \textit{I. furcatus} (compare side by side)

\item \textit{Noturus exilis}\,—\,slender madtom:  Dark edging on dorsal, caudal and anal fins. Upper jaw does not extend past lower jaw. Long and slender compared to most other madtoms.

\item \textit{Noturus nocturus}\,—\,freckled madtom: Uniformly dusky. Upper jaw extends past lower jaw. Lower lip and chink heavily sprinkled with melanophores.

\end{enumerate}

\end{document}  