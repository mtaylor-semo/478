%!TEX TS-program = lualatex
%!TEX encoding = UTF-8 Unicode

\documentclass[11pt]{article}
%\usepackage{graphicx}
%	\graphicspath{{/Users/goby/Pictures/teach/153/lab/}} % set of paths to search for images

\usepackage{geometry}
\geometry{letterpaper}                   
\geometry{bottom=1in}
%\geometry{landscape}                % Activate for for rotated page geometry
%\usepackage[parfill]{parskip}    % Activate to begin paragraphs with an empty line rather than an indent
%\usepackage{amssymb}
%\usepackage{mathtools}
%	\everymath{\displaystyle}

%\pagenumbering{gobble}

\usepackage{fontspec}
\setmainfont[Ligatures={Common}, BoldFont={* Bold}, ItalicFont={* Italic}, Numbers={Proportional}]{Linux Libertine O}
\setsansfont[Scale=MatchLowercase,Ligatures=TeX]{Linux Biolinum O}
\setmonofont[Scale=MatchLowercase]{Inconsolata}
\usepackage{microtype}

\usepackage{unicode-math}
\setmathfont[Scale=MatchLowercase]{Asana-Math.otf}
%\setmathfont{XITS Math}

% To define fonts for particular uses within a document. For example, 
% This sets the Libertine font to use tabular number format for tables.
%\newfontfamily{\tablenumbers}[Numbers={Monospaced}]{Linux Libertine O}
%\newfontfamily{\libertinedisplay}{Linux Libertine Display O}


\usepackage{booktabs}
%\usepackage{tabularx}
%\usepackage{longtable}
%\usepackage{siunitx}
%\usepackage[justification=raggedright, singlelinecheck=off]{caption}
%\captionsetup{labelsep=period} % Removes colon following figure / table number.
%\captionsetup{tablewithin=none}  % Sequential numbering of tables and figures instead of
%\captionsetup{figurewithin=none} % resetting numbers within each chapter (Intro, M&M, etc.)
%\captionsetup[table]{skip=0pt}

\usepackage{array}
\newcolumntype{L}[1]{>{\raggedright\let\newline\\\arraybackslash\hspace{0pt}}p{#1}}
\newcolumntype{C}[1]{>{\centering\let\newline\\\arraybackslash\hspace{0pt}}p{#1}}
\newcolumntype{R}[1]{>{\raggedleft\let\newline\\\arraybackslash\hspace{0pt}}p{#1}}

%\usepackage{enumitem}
%\usepackage{hyperref}
%\usepackage{placeins} %Provides \FloatBarrier to flush all floats before a certain point.
\usepackage{multicol}

\usepackage{pdflscape}

\usepackage{titling}
\setlength{\droptitle}{-60pt}
\posttitle{\par\end{center}}
\predate{}\postdate{}

%\usepackage{fancyhdr}
%\fancyhf{}
%\pagestyle{fancy}
%\lhead{}
%\chead{}
%\rhead{Name: \rule{5cm}{0.4pt} }
%\renewcommand{\headrulewidth}{0pt}

\newcommand{\VSpace}{\vspace{\baselineskip}}
\newcommand{\BigVSpace}{\vspace{2\baselineskip}}


\title{Form and Function}
\author{ZO 478 / 678 Lab 2}
\date{}                                           % Activate to display a given date or no date

\begin{document}
\maketitle
%\thispagestyle{fancy}

Each pair or table of students has several species of fishes to examine.  There may also be a tray of species on the back counter. These species contain morphological features that 1) serve as useful characters for identification and 2) provides clues about the ecological function of the fish in its native habitat.  The species\footnote{I may add or change a few species if necessary.} at your table may include:\VSpace

\begin{tabular}{ll}
\toprule
Lamprey 	&	Catfish\\
Shark 	&	Pickerel\\
Gar 	&	Midshipman \\
Shad 	&	Topminnow \\
Shiner (minnow)	&	Sunfish \\
Redhorse Sucker	&	Darter \\
\bottomrule
\end{tabular}

\VSpace\VSpace

Here are the traits for four characters that we’ll discuss in today’s lab. Refer to your notes and the lecture slides for details.

\VSpace

\begin{tabular}{llll}
\toprule
Scale Types	&	Mouth Position	&	Fin Position	&	Tail Shape \\
\midrule
Placoid		& 	Superior		&	Abdominal	&	Homocercal \\
Ganoid		&	Terminal		&	Thoracic		&	Lunate \\
Cycloid		&	Subterminal	&	Jugular		&	Heterocercal \\
Ctenoid		&	Inferior		&				&	Isocercal \\
\bottomrule
\end{tabular}

\VSpace\VSpace

Here are the ecological functions based on body form.\VSpace

\begin{tabular}{ll}
\toprule
Rover-Predator 	&	Bottom-Rovers \\
Lie-in-Wait Predator 	&	Clingers/Hiders \\
Surface-Oriented 	&	Flatfish \\
Deep-Bodied 		&	Rattails \\
Eel-Like 			& 	 \\
\bottomrule
\end{tabular}

\VSpace\VSpace
For each species listed on the next page, identify the morphology type of each character. Write “None” if the species does not have that character.  In the last column, write the ecological function of the species based on its overall form.  The last row (“Other”) row can be used for additional species on the back counter, if provided.

\newpage
\newgeometry{left=0.75in, right=0.75in, bottom=1in}
\begin{landscape}
\VSpace\centering
\noindent\begin{tabular}{@{}lccccc@{}}
	&	Scale Type	&	Mouth Position	&	Fin Position	&	Tail Shape	&	Function \\[2em]
Lamprey	& \rule{3cm}{0.4pt} & \rule{3cm}{0.4pt} & \rule{3cm}{0.4pt} & \rule{3cm}{0.4pt} & \rule{3cm}{0.4pt} \\[2em]
Shark	& \rule{3cm}{0.4pt} & \rule{3cm}{0.4pt} & \rule{3cm}{0.4pt} & \rule{3cm}{0.4pt} & \rule{3cm}{0.4pt} \\[2em]
Gar		& \rule{3cm}{0.4pt} & \rule{3cm}{0.4pt} & \rule{3cm}{0.4pt} & \rule{3cm}{0.4pt} & \rule{3cm}{0.4pt} \\[2em]
Shad		& \rule{3cm}{0.4pt} & \rule{3cm}{0.4pt} & \rule{3cm}{0.4pt} & \rule{3cm}{0.4pt} & \rule{3cm}{0.4pt} \\[2em]
Shiner	& \rule{3cm}{0.4pt} & \rule{3cm}{0.4pt} & \rule{3cm}{0.4pt} & \rule{3cm}{0.4pt} & \rule{3cm}{0.4pt} \\[2em]
Sucker	& \rule{3cm}{0.4pt} & \rule{3cm}{0.4pt} & \rule{3cm}{0.4pt} & \rule{3cm}{0.4pt} & \rule{3cm}{0.4pt} \\[2em]
Catfish	& \rule{3cm}{0.4pt} & \rule{3cm}{0.4pt} & \rule{3cm}{0.4pt} & \rule{3cm}{0.4pt} & \rule{3cm}{0.4pt} \\[2em]
Pickerel	& \rule{3cm}{0.4pt} & \rule{3cm}{0.4pt} & \rule{3cm}{0.4pt} & \rule{3cm}{0.4pt} & \rule{3cm}{0.4pt} \\[2em]
Midshipman	& \rule{3cm}{0.4pt} & \rule{3cm}{0.4pt} & \rule{3cm}{0.4pt} & \rule{3cm}{0.4pt} & \rule{3cm}{0.4pt} \\[2em]
Topminnow	& \rule{3cm}{0.4pt} & \rule{3cm}{0.4pt} & \rule{3cm}{0.4pt} & \rule{3cm}{0.4pt} & \rule{3cm}{0.4pt} \\[2em]
Sunfish	& \rule{3cm}{0.4pt} & \rule{3cm}{0.4pt} & \rule{3cm}{0.4pt} & \rule{3cm}{0.4pt} & \rule{3cm}{0.4pt} \\[2em]
Darter	& \rule{3cm}{0.4pt} & \rule{3cm}{0.4pt} & \rule{3cm}{0.4pt} & \rule{3cm}{0.4pt} & \rule{3cm}{0.4pt} \\[2em]
Other	& \rule{3cm}{0.4pt} & \rule{3cm}{0.4pt} & \rule{3cm}{0.4pt} & \rule{3cm}{0.4pt} & \rule{3cm}{0.4pt} \\
\end{tabular}
\end{landscape}
\end{document}  