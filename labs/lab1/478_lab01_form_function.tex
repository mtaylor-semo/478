%!TEX TS-program = lualatex
%!TEX encoding = UTF-8 Unicode

\documentclass[12pt, hidelinks]{exam}
\usepackage{graphicx}
	\graphicspath{{/Users/goby/Pictures/teach/434/handouts/}
	{img/}} % set of paths to search for images

\usepackage{geometry}
\geometry{letterpaper, left=1.5in, bottom=1in}                   
%\geometry{landscape}                % Activate for for rotated page geometry
\usepackage[parfill]{parskip}    % Activate to begin paragraphs with an empty line rather than an indent
\usepackage{amssymb, amsmath}
\usepackage{mathtools}
	\everymath{\displaystyle}

\usepackage{fontspec}
\setmainfont[Ligatures={TeX}, BoldFont={* Bold}, ItalicFont={* Italic}, BoldItalicFont={* BoldItalic}, Numbers={OldStyle}]{Linux Libertine O}
\setsansfont[Scale=MatchLowercase,Ligatures=TeX]{Linux Biolinum O}
\setmonofont[Scale=MatchLowercase]{Linux Libertine Mono O}
\usepackage{microtype}

% This defines \amper for the fancy ampersand
% to be used in the header. See
% https://tex.stackexchange.com/a/58185/39194
\usepackage{xspace}
\newfontfamily\amperfont[Style=Alternate]{Linux Libertine O}    
\makeatletter
\DeclareRobustCommand{\amper}{{\amperfont\ifx\f@shape\scname\smaller[1.2]\fi\&}\xspace}
\makeatother

% To define fonts for particular uses within a document. For example, 
% This sets the Libertine font to use tabular number format for tables.
 %\newfontfamily{\tablenumbers}[Numbers={Monospaced}]{Linux Libertine O}
% \newfontfamily{\libertinedisplay}{Linux Libertine Display O}

\usepackage{booktabs}
\usepackage{multicol}
\usepackage[normalem]{ulem}

\usepackage{longtable}
%\usepackage{siunitx}
\usepackage{array}
\newcolumntype{L}[1]{>{\raggedright\let\newline\\\arraybackslash\hspace{0pt}}p{#1}}
\newcolumntype{C}[1]{>{\centering\let\newline\\\arraybackslash\hspace{0pt}}p{#1}}
\newcolumntype{R}[1]{>{\raggedleft\let\newline\\\arraybackslash\hspace{0pt}}p{#1}}

\usepackage{enumitem}
\usepackage{hyperref}
%\usepackage{placeins} %PRovides \FloatBarrier to flush all floats before a certain point.
\usepackage{hanging}

\usepackage{pdflscape}

\usepackage[sc]{titlesec}

%% Commands for Exam class
\renewcommand{\solutiontitle}{\noindent}
\unframedsolutions
\SolutionEmphasis{\bfseries}

\renewcommand{\questionshook}{%
	\setlength{\leftmargin}{-\leftskip}%
}

%Change \half command from 1/2 to .5
\renewcommand*\half{.5}

\pagestyle{headandfoot}
\firstpageheader{\textsc{zo}\,478/678 Ichthyology}{}{}
\runningheader{}{}{\footnotesize{pg. \thepage}}
\footer{}{}{}
\runningheadrule

\newcommand*\AnswerBox[2]{%
    \parbox[t][#1]{0.92\textwidth}{%
    \begin{solution}#2\end{solution}}
%    \vspace*{\stretch{1}}
}

\newenvironment{AnswerPage}[1]
    {\begin{minipage}[t][#1]{0.92\textwidth}%
    \begin{solution}}
    {\end{solution}\end{minipage}
    \vspace*{\stretch{1}}}

\newlength{\basespace}
\setlength{\basespace}{5\baselineskip}

%\usepackage{mdframed}
%\mdfsetup{%
%	innerleftmargin=0pt,%
%	innerrightmargin=0pt,
%	innertopmargin=0pt,
%	innerbottommargin=0pt,
%	hidealllines=true
%}%end mdfsetup

%
%\makeatletter
%\def\SetTotalwidth{\advance\linewidth by \@totalleftmargin
%\@totalleftmargin=0pt}
%\makeatother


%\printanswers


\begin{document}

\subsection*{Form and Function}

You have several species of fishes at your table to examine.  You may also find a tray of species on the back counter. These species contain morphological features that serve as useful characters for identification and that provides clues about the ecological function of the fish in its native habitat.  The species\footnote{I may add or change a few species if necessary.} at your table may include:

\bigskip

\begin{tabular}{ll}
\toprule
Lamprey 		&	Catfish\\
Shark 			&	Pickerel\\
Gar 			&	Midshipman \\
Shad 			&	Killifish \\
Shiner (minnow)	&	Sunfish \\
Redhorse Sucker	&	Darter \\
\bottomrule
\end{tabular}

\bigskip

Here are the traits for four characters that we will discuss in today’s lab. Refer to your notes and the lecture slides for details.

\bigskip

\begin{tabular}{llll}
\toprule
Scale Types	& Mouth Position	& Fin Position	& Tail Shape \\
\midrule
Placoid		& Superior			& Abdominal		& Homocercal \\
Ganoid		& Terminal			& Thoracic		& Lunate \\
Cycloid		& Subterminal		& Jugular		& Heterocercal \\
Ctenoid		& Inferior			&				& Isocercal \\
\bottomrule
\end{tabular}

\bigskip

Here are the ecological functions based on body form.\bigskip

\begin{tabular}{ll}
\toprule
Rover-Predator 			&	Bottom-Rovers \\
Lie-in-Wait Predator 	&	Clingers/Hiders \\
Surface-Oriented 		&	Flatfish \\
Deep-Bodied 			&	Rattails \\
Eel-Like 				& 	 \\
\bottomrule
\end{tabular}

\bigskip

For each species listed on the next page, identify the morphology type of each character. Write “None” if the species does not have that character.  In the last column, write the ecological function of the species based on its overall form.  Use the last row (“Other”) for additional species on the back counter, if provided.

%\newpage
\newgeometry{left=0.75in, right=0.75in, bottom=1in}
\begin{landscape}
\centering
\noindent\begin{tabular}{@{}lccccc@{}}
	&	Scale Type	&	Mouth Position	&	Fin Position	&	Tail Shape	&	Function \\[1.98em]
Lamprey	& \rule{3cm}{0.4pt} & \rule{3cm}{0.4pt} & \rule{3cm}{0.4pt} & \rule{3cm}{0.4pt} & \rule{3cm}{0.4pt} \\[1.98em]
Shark	& \rule{3cm}{0.4pt} & \rule{3cm}{0.4pt} & \rule{3cm}{0.4pt} & \rule{3cm}{0.4pt} & \rule{3cm}{0.4pt} \\[1.98em]
Gar		& \rule{3cm}{0.4pt} & \rule{3cm}{0.4pt} & \rule{3cm}{0.4pt} & \rule{3cm}{0.4pt} & \rule{3cm}{0.4pt} \\[1.98em]
Shad		& \rule{3cm}{0.4pt} & \rule{3cm}{0.4pt} & \rule{3cm}{0.4pt} & \rule{3cm}{0.4pt} & \rule{3cm}{0.4pt} \\[1.98em]
Shiner	& \rule{3cm}{0.4pt} & \rule{3cm}{0.4pt} & \rule{3cm}{0.4pt} & \rule{3cm}{0.4pt} & \rule{3cm}{0.4pt} \\[1.98em]
Sucker	& \rule{3cm}{0.4pt} & \rule{3cm}{0.4pt} & \rule{3cm}{0.4pt} & \rule{3cm}{0.4pt} & \rule{3cm}{0.4pt} \\[1.98em]
Catfish	& \rule{3cm}{0.4pt} & \rule{3cm}{0.4pt} & \rule{3cm}{0.4pt} & \rule{3cm}{0.4pt} & \rule{3cm}{0.4pt} \\[1.98em]
Pickerel	& \rule{3cm}{0.4pt} & \rule{3cm}{0.4pt} & \rule{3cm}{0.4pt} & \rule{3cm}{0.4pt} & \rule{3cm}{0.4pt} \\[1.98em]
Midshipman	& \rule{3cm}{0.4pt} & \rule{3cm}{0.4pt} & \rule{3cm}{0.4pt} & \rule{3cm}{0.4pt} & \rule{3cm}{0.4pt} \\[1.98em]
Killifish	& \rule{3cm}{0.4pt} & \rule{3cm}{0.4pt} & \rule{3cm}{0.4pt} & \rule{3cm}{0.4pt} & \rule{3cm}{0.4pt} \\[1.98em]
Sunfish	& \rule{3cm}{0.4pt} & \rule{3cm}{0.4pt} & \rule{3cm}{0.4pt} & \rule{3cm}{0.4pt} & \rule{3cm}{0.4pt} \\[1.98em]
Darter	& \rule{3cm}{0.4pt} & \rule{3cm}{0.4pt} & \rule{3cm}{0.4pt} & \rule{3cm}{0.4pt} & \rule{3cm}{0.4pt} \\[1.98em]
Other	& \rule{3cm}{0.4pt} & \rule{3cm}{0.4pt} & \rule{3cm}{0.4pt} & \rule{3cm}{0.4pt} & \rule{3cm}{0.4pt} \\
\end{tabular}
\end{landscape}

\end{document}  