%!TEX TS-program = lualatex
%!TEX encoding = UTF-8 Unicode

\documentclass[12pt]{exam}
\usepackage{graphicx}
	\graphicspath{{/Users/goby/Pictures/teach/438/homework/}} % set of paths to search for images

\usepackage{geometry}
\geometry{letterpaper, bottom=0.9in}                   

\usepackage{afterpage}
\usepackage{pdflscape}

\newlength{\myindent}
\setlength{\myindent}{\parindent}
\newcommand{\ind}{\hspace*{\myindent}}

\newlength{\litindent}
\setlength{\litindent}{\parindent}

\usepackage[parfill]{parskip} 

\usepackage{fontspec}
\setmainfont[Ligatures={TeX}, BoldFont={* Bold}, ItalicFont={* Italic}, BoldItalicFont={* BoldItalic}, Numbers={OldStyle,Proportional}]{Linux Libertine O}
\setsansfont[Scale=MatchLowercase,Ligatures=TeX, Numbers=OldStyle]{Linux Biolinum O}
\setmonofont[Scale=MatchLowercase]{Linux Libertine Mono O}
\usepackage{microtype}

\usepackage{unicode-math}
\setmathfont[Scale=MatchLowercase]{Asana Math}
%\setmathfont[Scale=MatchLowercase]{XITS Math}

% To define fonts for particular uses within a document. For example, 
% This sets the Libertine font to use tabular number format for tables.
\newfontfamily{\tablenumbers}[Numbers={Monospaced}]{Linux Libertine O}
\newfontfamily{\libertinedisplay}{Linux Libertine Display O}

\usepackage{longtable}

\usepackage{booktabs}
\usepackage{multirow}
\usepackage{multicol}

\usepackage[justification=raggedright, labelsep=period]{caption}
\captionsetup{singlelinecheck=off}
\captionsetup{skip=0.2em}

%\usepackage{tabularx}
%\usepackage{siunitx}
\usepackage{array}
\newcolumntype{L}[1]{>{\raggedright\let\newline\\\arraybackslash\hspace{0pt}}p{#1}}
\newcolumntype{C}[1]{>{\centering\let\newline\\\arraybackslash\hspace{0pt}}p{#1}}
\newcolumntype{R}[1]{>{\raggedleft\let\newline\\\arraybackslash\hspace{0pt}}p{#1}}

\newcolumntype{M}[1]{>{\centering\let\newline\\\arraybackslash\hspace{0pt}}m{#1}}

\usepackage{tikz}

\usepackage{enumitem}
\setlist{leftmargin=*}
\setlist[1]{labelindent=\parindent}
%\setlist[enumerate]{label=\textsc{\alph*}., ref=\textsc{\alph*}}

\usepackage[colorlinks=true, allcolors=blue, allbordercolors={1 1 1}]{hyperref}
%\usepackage{hanging}

\usepackage[sc]{titlesec}

\renewcommand{\solutiontitle}{\noindent}
\unframedsolutions
\SolutionEmphasis{\bfseries}

\renewcommand{\questionshook}{%
	\setlength{\leftmargin}{-\leftskip}%
}
%Change \half command from 1/2 to .5
%\renewcommand*\half{.5}


\makeatletter
\def\SetTotalwidth{\advance\linewidth by \@totalleftmargin
\@totalleftmargin=0pt}
\makeatother

\pagestyle{headandfoot}
\firstpageheader{\textsc{zo}~478/678: Ichthyology}{}{\ifprintanswers\textbf{KEY}\else Name: \enspace \makebox[2.5in]{\hrulefill}\fi}
\runningheader{Richness comparison}{}{\footnotesize{pg. \thepage}}
\footer{}{}{}
\runningheadrule

\newcommand*\AnswerBox[2]{%
    \parbox[t][#1]{0.92\textwidth}{%
    \begin{solution}#2\end{solution}}
    \vspace{\stretch{1}}
}

\newenvironment{AnswerPage}[1]
    {\begin{minipage}[t][#1]{0.92\textwidth}%
    \begin{solution}}
    {\end{solution}\end{minipage}
    \vspace{\stretch{1}}}

\newlength{\basespace}
\setlength{\basespace}{5\baselineskip}

\newcommand{\shortblank}{\quad\rule{0.5in}{0.4pt}}
\newcommand{\tableblank}{\rule{0.8in}{0.4pt}}

%\printanswers

\begin{document}

\subsection*{Taxonomic richness: Missouri, North America \& the world (20 pts)}

Missouri has some of the greatest diversity of freshwater fishes in the United
States. How does this compare to the diversity of fishes in North
America and in the World?

Use the data sheets provided to answer the following questions. You may work in pairs
but \emph{please} share the workload. Try working independently for all sections except
Section~\textsc{c}, then compare your results. If your answers differ, recount to see who is correct. You will have to work together for Section~\textsc{c} because I don't have enough copies of Plieger for everyone.


Turn in this assignment at the end of lab today. 

The numbers in parentheses in the \textsc{usfw} column are the number of species
found only in freshwater, the others are estuarine species.

\vspace{0.5cm}

\subsubsection*{A. World ichthyofauna}

\begin{enumerate}
\item How many extant \textbf{orders} of fishes are there in world fauna? \shortblank

\item How many extant \textbf{families} of fishes are there in the world? \shortblank

\item How many orders have \textbf{no} freshwater representatives? \shortblank

\item What \textbf{percent} of the orders is represented in fresh water? \shortblank

\item How many extant \textbf{species} are found in the world fauna? \shortblank

\end{enumerate}

\vspace{1cm}

\subsubsection*{B. U.S.~\& Canada ichthyofauna compared to the world}

\begin{enumerate}[resume]
\item How many \textbf{orders} are representated in the U.S.~\& Canadian ichthyofauna? \shortblank

\item What \textbf{percent} of the world orders is found in the U.S. \& Canada? \shortblank

\item How many \textbf{families} are representated in the U.S.~\& Canadian ichthyofauna? \shortblank

\item What \textbf{percent} of the world families is found in the U.S.~\& Canada? \shortblank

\item How many \textbf{species} are found in the U.S.~\& Canadian ichthyofauna? \shortblank

\item What \textbf{percent} of the world species is found in the U.S.~\& Canada? \shortblank

\item How many species occur in \textbf{freshwater} \textbf{only} in the U.S.~\&
Canada? \shortblank

\item What \textbf{percent} of the world \textbf{freshwater} species is
found in U.S.~\& Canada? \shortblank

\end{enumerate}

\newpage

\subsubsection*{C. Missouri ichthyofauna compared to the U.S., Canada and the
world}

Compile data about the Missouri ichthyofauna from Pflieger 1997 to
complete the table below. You will have to work in pairs for this section.

\emph{Leave the blanks empty for orders not represented in Missouri.}

{\renewcommand{\arraystretch}{1.21}
\begin{longtable}[l]{@{}llll@{}}
\toprule
Order & MO Families & MO Genera & MO Species \tabularnewline
\midrule
\endhead
Myxiniformes & \tableblank & \tableblank & \tableblank \tabularnewline
Petromyzontiformes & \tableblank & \tableblank  & \tableblank \tabularnewline
Chimaeriformes & \tableblank & \tableblank  & \tableblank \tabularnewline
Heterodontiformes & \tableblank & \tableblank  & \tableblank \tabularnewline
Orectolobiformes & \tableblank & \tableblank  & \tableblank \tabularnewline
Lamniformes & \tableblank & \tableblank  & \tableblank \tabularnewline
Carcharhiniformes & \tableblank & \tableblank  & \tableblank \tabularnewline
Hexanchiformes & \tableblank & \tableblank  & \tableblank \tabularnewline
Echinorhiniformes & \tableblank & \tableblank  & \tableblank \tabularnewline
Squaliformes & \tableblank & \tableblank  & \tableblank \tabularnewline
Squatiniformes & \tableblank & \tableblank  & \tableblank \tabularnewline
Pristiophoriformes & \tableblank & \tableblank  & \tableblank \tabularnewline
Torpediniformes & \tableblank & \tableblank  & \tableblank \tabularnewline
Pristiformes & \tableblank & \tableblank  & \tableblank \tabularnewline
Rajiformes & \tableblank & \tableblank  & \tableblank \tabularnewline
Myliobatiformes & \tableblank & \tableblank  & \tableblank \tabularnewline
Polypteriformes & \tableblank & \tableblank  & \tableblank \tabularnewline
Acipenseriformes & \tableblank & \tableblank  & \tableblank \tabularnewline
Lepisosteiformes & \tableblank & \tableblank  & \tableblank \tabularnewline
Amiiformes & \tableblank & \tableblank  & \tableblank \tabularnewline
Hiodontiformes & \tableblank & \tableblank  & \tableblank \tabularnewline
Osteoglossiformes & \tableblank & \tableblank  & \tableblank \tabularnewline
Elopiformes & \tableblank & \tableblank  & \tableblank \tabularnewline
Albuliformes & \tableblank & \tableblank  & \tableblank \tabularnewline
Anguilliformes & \tableblank & \tableblank  & \tableblank \tabularnewline
Saccopharyngiformes & \tableblank & \tableblank  & \tableblank \tabularnewline
Clupeiformes~ & \tableblank & \tableblank  & \tableblank \tabularnewline
Gonorynchiformes~ & \tableblank & \tableblank  & \tableblank \tabularnewline
Cypriniformes & \tableblank & \tableblank  & \tableblank \tabularnewline
Characiformes & \tableblank & \tableblank  & \tableblank \tabularnewline
Siluriformes & \tableblank & \tableblank  & \tableblank \tabularnewline
Gymnotiformes & \tableblank & \tableblank  & \tableblank \tabularnewline
Argentiniformes & \tableblank & \tableblank  & \tableblank \tabularnewline
Osmeriformes & \tableblank & \tableblank  & \tableblank \tabularnewline
Salmoniformes & \tableblank & \tableblank  & \tableblank \tabularnewline
Esociformes & \tableblank & \tableblank  & \tableblank \tabularnewline
Stomiiformes & \tableblank & \tableblank  & \tableblank \tabularnewline
Ateleopodiformes & \tableblank & \tableblank  & \tableblank \tabularnewline
Aulopiformes & \tableblank & \tableblank  & \tableblank \tabularnewline
Myctophiformes & \tableblank & \tableblank  & \tableblank \tabularnewline
Lampriformes & \tableblank & \tableblank  & \tableblank \tabularnewline
Polymixiiformes & \tableblank & \tableblank  & \tableblank \tabularnewline
Percopsiformes & \tableblank & \tableblank  & \tableblank \tabularnewline
Gadiformes & \tableblank & \tableblank  & \tableblank \tabularnewline
Ophidiiformes & \tableblank & \tableblank  & \tableblank \tabularnewline
Batrachoidiformes & \tableblank & \tableblank  & \tableblank \tabularnewline
Lophiiformes & \tableblank & \tableblank  & \tableblank \tabularnewline
Mugiliformes & \tableblank & \tableblank  & \tableblank \tabularnewline
Atheriniformes & \tableblank & \tableblank  & \tableblank \tabularnewline
Beloniformes & \tableblank & \tableblank  & \tableblank \tabularnewline
Cyprinodontiformes & \tableblank & \tableblank  & \tableblank \tabularnewline
Stephanoberyciformes & \tableblank & \tableblank  & \tableblank \tabularnewline
Beryciformes & \tableblank & \tableblank  & \tableblank \tabularnewline
Zeiformes & \tableblank & \tableblank  & \tableblank \tabularnewline
Gasterosteiformes & \tableblank & \tableblank  & \tableblank \tabularnewline
Synbranchiformes & \tableblank & \tableblank  & \tableblank \tabularnewline
Scorpaeniformes & \tableblank & \tableblank  & \tableblank \tabularnewline
Perciformes & \tableblank & \tableblank  & \tableblank \tabularnewline
Pleuronectiformes & \tableblank & \tableblank  & \tableblank \tabularnewline
Tetraodontiformes & \tableblank & \tableblank  & \tableblank \tabularnewline
Coelacanthiformes & \tableblank & \tableblank  & \tableblank \tabularnewline
Ceratodontiformes & \tableblank & \tableblank  & \tableblank \tabularnewline
\midrule
TOTAL & \tableblank & \tableblank & \tableblank \tabularnewline
\bottomrule
\end{longtable}
}% end arraystretch

\begin{enumerate}[resume]

\item How many orders, families, genera, and species of the world
ichthyofauna are found in Missouri?

\shortblank~Orders \shortblank~Families \shortblank~Genera \shortblank~Species

\item What percent of the world orders, families, genera, and species is
found in the Missouri ichthyofauna?

\shortblank~\% Orders \shortblank~\% Families \shortblank~\% Genera \shortblank~\% Species

\item How many orders, families, genera, and species of the US \& Canada
ichthyofauna are found in Missouri?

\shortblank~Orders \shortblank~Families \shortblank~Genera \shortblank~Species

\item What percent of the US \& Canada freshwater orders, families, genera,
and species is found in the Missouri ichthyofauna?

\shortblank~\% Orders \shortblank~\% Families \shortblank~\% Genera \shortblank~\% Species
\end{enumerate}

\vfill

Con't next page.

\subsubsection*{D. Largest fish families of Missouri}

In the space below list the two largest families of Missouri fishes, list the genera within each family, and give the
number of species within each genus. You might not need all the blanks for one or both genera.

How many genera in the largest family are not native the U.S.? Put a check mark to the left of each non-native genus for the largest family.

{\renewcommand{\arraystretch}{1.35}
\begin{longtable}[l]{@{}ll@{}}
\toprule
Largest family: & Second largest family: \tabularnewline[0.35cm]

\hrulefill & \hrulefill \tabularnewline

%& \tabularnewline

Genus \hspace{1.5in} \hfill \#spp & Genus \hspace{1.5in} \#spp \tabularnewline

\rule{1.5in}{0.4pt} \hfill \rule{0.5in}{0.4pt} & \rule{1.5in}{0.4pt} \hfill \rule{0.5in}{0.4pt}  \tabularnewline

\rule{1.5in}{0.4pt} \hfill \rule{0.5in}{0.4pt} & \rule{1.5in}{0.4pt} \hfill \rule{0.5in}{0.4pt}  \tabularnewline

\rule{1.5in}{0.4pt} \hfill \rule{0.5in}{0.4pt} & \rule{1.5in}{0.4pt} \hfill \rule{0.5in}{0.4pt}  \tabularnewline

\rule{1.5in}{0.4pt} \hfill \rule{0.5in}{0.4pt} & \rule{1.5in}{0.4pt} \hfill \rule{0.5in}{0.4pt}  \tabularnewline

\rule{1.5in}{0.4pt} \hfill \rule{0.5in}{0.4pt} & \rule{1.5in}{0.4pt} \hfill \rule{0.5in}{0.4pt}  \tabularnewline

\rule{1.5in}{0.4pt} \hfill \rule{0.5in}{0.4pt} & \rule{1.5in}{0.4pt} \hfill \rule{0.5in}{0.4pt}  \tabularnewline

\rule{1.5in}{0.4pt} \hfill \rule{0.5in}{0.4pt} & \rule{1.5in}{0.4pt} \hfill \rule{0.5in}{0.4pt}  \tabularnewline

\rule{1.5in}{0.4pt} \hfill \rule{0.5in}{0.4pt} & \rule{1.5in}{0.4pt} \hfill \rule{0.5in}{0.4pt}  \tabularnewline

\rule{1.5in}{0.4pt} \hfill \rule{0.5in}{0.4pt} & \rule{1.5in}{0.4pt} \hfill \rule{0.5in}{0.4pt}  \tabularnewline

\rule{1.5in}{0.4pt} \hfill \rule{0.5in}{0.4pt} & \rule{1.5in}{0.4pt} \hfill \rule{0.5in}{0.4pt}  \tabularnewline

\rule{1.5in}{0.4pt} \hfill \rule{0.5in}{0.4pt} & \rule{1.5in}{0.4pt} \hfill \rule{0.5in}{0.4pt}  \tabularnewline

\rule{1.5in}{0.4pt} \hfill \rule{0.5in}{0.4pt} & \rule{1.5in}{0.4pt} \hfill \rule{0.5in}{0.4pt}  \tabularnewline

\rule{1.5in}{0.4pt} \hfill \rule{0.5in}{0.4pt} & \rule{1.5in}{0.4pt} \hfill \rule{0.5in}{0.4pt}  \tabularnewline

\rule{1.5in}{0.4pt} \hfill \rule{0.5in}{0.4pt} & \rule{1.5in}{0.4pt} \hfill \rule{0.5in}{0.4pt}  \tabularnewline

\rule{1.5in}{0.4pt} \hfill \rule{0.5in}{0.4pt} & \rule{1.5in}{0.4pt} \hfill \rule{0.5in}{0.4pt}  \tabularnewline

\rule{1.5in}{0.4pt} \hfill \rule{0.5in}{0.4pt} & \rule{1.5in}{0.4pt} \hfill \rule{0.5in}{0.4pt}  \tabularnewline

\rule{1.5in}{0.4pt} \hfill \rule{0.5in}{0.4pt} & \rule{1.5in}{0.4pt} \hfill \rule{0.5in}{0.4pt}  \tabularnewline

\rule{1.5in}{0.4pt} \hfill \rule{0.5in}{0.4pt} & \rule{1.5in}{0.4pt} \hfill \rule{0.5in}{0.4pt}  \tabularnewline

\rule{1.5in}{0.4pt} \hfill \rule{0.5in}{0.4pt} & \rule{1.5in}{0.4pt} \hfill \rule{0.5in}{0.4pt}  \tabularnewline

\rule{1.5in}{0.4pt} \hfill \rule{0.5in}{0.4pt} & \rule{1.5in}{0.4pt} \hfill \rule{0.5in}{0.4pt}  \tabularnewline

\bottomrule
\end{longtable}
} % End array stretch

\end{document}  