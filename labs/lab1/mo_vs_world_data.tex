%!TEX TS-program = lualatex
%!TEX encoding = UTF-8 Unicode

\documentclass[11pt]{exam}
\usepackage{graphicx}
	\graphicspath{{/Users/goby/Pictures/teach/438/homework/}} % set of paths to search for images

\usepackage{geometry}
\geometry{letterpaper, bottom=1in, left=0.75in, right=0.75in} 

\usepackage{afterpage}
\usepackage{pdflscape}

\newlength{\myindent}
\setlength{\myindent}{\parindent}
\newcommand{\ind}{\hspace*{\myindent}}

\newlength{\litindent}
\setlength{\litindent}{\parindent}

\usepackage[parfill]{parskip} 

\usepackage{fontspec}
\setmainfont[Ligatures={TeX}, BoldFont={* Bold}, ItalicFont={* Italic}, BoldItalicFont={* BoldItalic}, Numbers={Lining}]{Linux Libertine O}
\setsansfont[Scale=MatchLowercase,Ligatures=TeX, Numbers=OldStyle]{Linux Biolinum O}
\setmonofont[Scale=MatchLowercase]{Linux Libertine Mono O}
\usepackage{microtype}

\usepackage{unicode-math}
\setmathfont[Scale=MatchLowercase]{Asana Math}
%\setmathfont[Scale=MatchLowercase]{XITS Math}

% To define fonts for particular uses within a document. For example, 
% This sets the Libertine font to use tabular number format for tables.
\newfontfamily{\tablenumbers}[Numbers={Monospaced}]{Linux Libertine O}
\newfontfamily{\libertinedisplay}{Linux Libertine Display O}

\usepackage{longtable}

\usepackage{booktabs}
\usepackage{multirow}
\usepackage{multicol}

\usepackage[justification=raggedright, labelsep=period]{caption}
\captionsetup{singlelinecheck=off}
\captionsetup{skip=0.2em}

%\usepackage{tabularx}
%\usepackage{siunitx}
\usepackage{array}
\newcolumntype{L}[1]{>{\raggedright\let\newline\\\arraybackslash\hspace{0pt}}p{#1}}
\newcolumntype{C}[1]{>{\centering\let\newline\\\arraybackslash\hspace{0pt}}p{#1}}
\newcolumntype{R}[1]{>{\raggedleft\let\newline\\\arraybackslash\hspace{0pt}}p{#1}}

\newcolumntype{M}[1]{>{\centering\let\newline\\\arraybackslash\hspace{0pt}}m{#1}}

\usepackage{tikz}

\usepackage{enumitem}
\setlist{leftmargin=*}
\setlist[1]{labelindent=\parindent}
%\setlist[enumerate]{label=\textsc{\alph*}., ref=\textsc{\alph*}}

\usepackage[colorlinks=true, allcolors=blue, allbordercolors={1 1 1}]{hyperref}
%\usepackage{hanging}

\usepackage[sc]{titlesec}

\renewcommand{\solutiontitle}{\noindent}
\unframedsolutions
\SolutionEmphasis{\bfseries}

\renewcommand{\questionshook}{%
	\setlength{\leftmargin}{-\leftskip}%
}
%Change \half command from 1/2 to .5
%\renewcommand*\half{.5}


\makeatletter
\def\SetTotalwidth{\advance\linewidth by \@totalleftmargin
\@totalleftmargin=0pt}
\makeatother

\pagestyle{headandfoot}
\firstpageheader{\textsc{zo}~478/678: Ichthyology}{}{\ifprintanswers\textbf{KEY}\else Taxonomic Richness Data\fi}
\runningheader{Richness comparison}{}{\footnotesize{pg. \thepage}}
\footer{}{}{}
\runningheadrule

\newcommand*\AnswerBox[2]{%
    \parbox[t][#1]{0.92\textwidth}{%
    \begin{solution}#2\end{solution}}
    \vspace{\stretch{1}}
}

\newenvironment{AnswerPage}[1]
    {\begin{minipage}[t][#1]{0.92\textwidth}%
    \begin{solution}}
    {\end{solution}\end{minipage}
    \vspace{\stretch{1}}}

\newlength{\basespace}
\setlength{\basespace}{5\baselineskip}

\newcommand{\shortblank}{\quad\rule{0.5in}{0.4pt}}
\newcommand{\tableblank}{\rule{0.8in}{0.4pt}}

%\printanswers

\begin{document}

\begin{landscape}
\subsubsection*{Taxonomic richness data.}

\begin{longtable}[l]{@{}L{1.4in}R{0.3in}R{0.4in}R{0.4in}R{0.4in}lR{0.3in}R{0.3in}R{0.3in}R{0.3in}R{0.3in}R{0.4in}R{0.4in}@{}}
\toprule
& \multicolumn{4}{c}{World-wide} & & \multicolumn{7}{c}{U.S.~\& Canada} \tabularnewline
\cmidrule(lr){2-5} \cmidrule(l){7-13}
Order & Fam & Gen & Spp. &
FW & & N.A.\newline Fam & FW-\newline Mar & FW &
Atl & Pac & Atl \&\newline Pac & U.S.~\newline Total \tabularnewline
\midrule
\endhead
%
Myxiniformes & 1 & 7 & 70 & 0 & ~ & 1 & 0 & 0 & 1 & 2 & 0 &
3\tabularnewline
Petromyzontiformes & 3 & 10 & 38 & 29 & & 1 & 17 & 13 & 1 & 3 & 0 &
17\tabularnewline
Chimaeriformes & 3 & 6 & 33 & 0 & & 1 & 0 & 0 & 0 & 1 & 0 &
1\tabularnewline
Heterodontiformes & 1 & 1 & 8 & 0 & & 1 & 0 & 0 & 0 & 1 & 0 &
1\tabularnewline
Orectolobiformes & 7 & 14 & 32 & 0 & & 2 & 0 & 0 & 1 & 0 & 1 &
2\tabularnewline
Lamniformes & 7 & 10 & 15 & 0 & & 3 & 0 & 0 & 3 & 2 & 5 &
10\tabularnewline
Carcharhiniformes & 8 & 49 & 224 & 1 & & 2 & 1 & 0 & 16 & 10 & 5 &
31\tabularnewline
Hexanchiformes & 2 & 4 & 5 & 0 & & 2 & 0 & 0 & 0 & 2 & 1 &
3\tabularnewline
Echinorhiniformes & 1 & 1 & 2 & 0 & & 1 & 0 & 0 & 1 & 0 & 0 &
1\tabularnewline
Squaliformes & 6 & 24 & 97 & 0 & & 3 & 0 & 0 & 7 & 2 & 4 &
15\tabularnewline
Squatiniformes & 1 & 1 & 15 & 0 & & 1 & 0 & 0 & 1 & 1 & 0 &
2\tabularnewline
Pristiophoriformes & 1 & 2 & 5 & 0 & & – & – & – & – & – & – &
–\tabularnewline
Torpediniformes & 2 & 11 & 59 & 0 & & 2 & 0 & 0 & 2 & 1 & 0 &
3\tabularnewline
Pristiformes & 1 & 2 & 7 & 0 & & 1 & 0 & 0 & 2 & 0 & 0 &
2\tabularnewline
Rajiformes & 4 & 32 & 285 & 0 & & 1 & 0 & 0 & 11 & 10 & 0 &
21\tabularnewline
Myliobatiformes & 10 & 27 & 183 & 23 & & 2 & 1 & 0 & 13 & 7 & 1 &
21\tabularnewline
Polypteriformes & 1 & 2 & 16 & 16 & & – & – & – & – & – & – &
–\tabularnewline
Acipenseriformes & 2 & 6 & 27 & 14 & & 2 & 8 & 4 & 2 & 2 & 0 &
8\tabularnewline
Lepisosteiformes & 1 & 2 & 7 & 6 & & 1 & 5 & 5 & 0 & 0 & 0 &
5\tabularnewline
Amiiformes & 1 & 1 & 1 & 1 & & 1 & 1 & 1 & 0 & 0 & 0 & 1\tabularnewline
Hiodontiformes & 1 & 1 & 2 & 2 & & 1 & 2 & 2 & 0 & 0 & 0 &
2\tabularnewline
Osteoglossiformes & 4 & 28 & 218 & 218 & & – & – & – & – & – & – &
–\tabularnewline
Elopiformes & 2 & 2 & 8 & 0 & & 2 & 3 & 0 & 2 & 1 & 0 & 3\tabularnewline
Albuliformes & 3 & 8 & 30 & 0 & & 2 & 1 & 0 & 1 & 0 & 1 &
2\tabularnewline
Anguilliformes & 15 & 141 & 791 & 6 & & 9 & 1 & 0 & 59 & 7 & 1 &
67\tabularnewline
Saccopharyngiformes & 4 & 5 & 28 & 0 & & – & – & – & – & – & – &
–\tabularnewline
Clupeiformes~ & 5 & 84 & 364 & 79 & & 2 & 11 & 0 & 30 & 8 & 3 &
41\tabularnewline
Gonorynchiformes~ & 4 & 7 & 37 & 31 & & – & – & – & – & – & – &
–\tabularnewline
Cypriniformes & 6 & 321 & 3,268 & 3,268 & & 2 & 273 & 273 & 0 & 0 & 0 &
273\tabularnewline
Characiformes & 18 & 270 & 1,674 & 1,674 & & 1 & 1 & 1 & 0 & 0 & 0 &
1\tabularnewline
Siluriformes & 35 & 446 & 2,867 & 2,740 & & 2 & 40 & 39 & 2 & 1 & 0 &
42\tabularnewline
Gymnotiformes & 5 & 30 & 134 & 134 & & – & – & – & – & – & – &
–\tabularnewline
Argentiniformes & 6 & 57 & 202 & 0 & & 2 & 0 & 0 & 3 & 3 & 0 &
6\tabularnewline
Osmeriformes & 3 & 22 & 88 & 82 & & 1 & 6 & 1 & 0 & 6 & 2 &
9\tabularnewline
Salmoniformes & 1 & 11 & 66 & 45 & & 1 & 38 & 25 & 3 & 6 & 4 &
38\tabularnewline
Esociformes & 2 & 4 & 10 & 10 & & 2 & 9 & 9 & 0 & 0 & 0 &
9\tabularnewline
Stomiiformes & 5 & 53 & 391 & 0 & & 2 & 0 & 0 & 0 & 2 & 0 &
2\tabularnewline
Ateleopodiformes & 1 & 4 & 12 & 0 & & – & – & – & – & – & – &
–\tabularnewline
Aulopiformes & 15 & 44 & 236 & 0 & & 7 & 0 & 0 & 12 & 3 & 2 &
17\tabularnewline
Myctophiformes & 2 & 35 & 246 & 0 & & 1 & 0 & 0 & 0 & 9 & 0 &
9\tabularnewline
Lampriformes & 7 & 12 & 21 & 0 & & 5 & 0 & 0 & 4 & 3 & 4 &
11\tabularnewline
Polymixiiformes & 1 & 1 & 10 & 0 & & 1 & 0 & 0 & 1 & 0 & 0 &
1\tabularnewline
Percopsiformes & 3 & 7 & 9 & 9 & & 3 & 9 & 9 & 0 & 0 & 0 &
9\tabularnewline
Gadiformes & 9 & 75 & 555 & 1 & & 3 & 2 & 1 & 23 & 6 & 1 &
31\tabularnewline
Ophidiiformes & 5 & 100 & 385 & 5 & & 3 & 0 & 0 & 19 & 4 & 0 &
23\tabularnewline
Batrachoidiformes & 1 & 22 & 78 & 6 & & 1 & 0 & 0 & 4 & 2 & 0 &
6\tabularnewline
Lophiiformes & 18 & 66 & 313 & 0 & & 6 & 0 & 0 & 17 & 3 & 0 &
20\tabularnewline
Mugiliformes & 1 & 17 & 72 & 1 & & 1 & 2 & 0 & 5 & 0 & 1 &
6\tabularnewline
Atheriniformes & 6 & 48 & 312 & 210 & & 1 & 3 & 2 & 7 & 3 & 0 &
12\tabularnewline
Beloniformes & 5 & 36 & 227 & 98 & & 3 & 1 & 0 & 23 & 8 & 2 &
33\tabularnewline
Cyprinodontiformes & 10 & 109 & 1,013 & 996 & & 3 & 56 & 44 & 15 & 1 & 1
& 61\tabularnewline
Stephanoberyciformes & 9 & 28 & 75 & 0 & & – & – & – & – & – & – &
–\tabularnewline
Beryciformes & 7 & 29 & 144 & 0 & & 1 & 0 & 0 & 11 & 0 & 0 &
11\tabularnewline
Zeiformes & 6 & 16 & 32 & 0 & & 3 & 0 & 0 & 5 & 1 & 0 & 6\tabularnewline
Gasterosteiformes & 11 & 71 & 278 & 21 & & 5 & 5 & 1 & 28 & 8 & 3 &
40\tabularnewline
Synbranchiformes & 3 & 15 & 99 & 96 & & – & – & – & – & – & – &
–\tabularnewline
Scorpaeniformes & 26 & 279 & 1,477 & 60 & & 8 & 28 & 23 & 62 & 194 & 11 &
290\tabularnewline
Perciformes & 160 & 1,539 & 10,033 & 2,040 & & 72 & 233 & 168 & 474 & 194
& 35 & 871\tabularnewline
Pleuronectiformes & 14 & 134 & 678 & 10 & & 4 & 4 & 0 & 54 & 31 & 1 &
86\tabularnewline
Tetraodontiformes & 9 & 101 & 357 & 14 & & 6 & 0 & 0 & 36 & 6 & 6 &
48\tabularnewline
Coelacanthiformes & 1 & 1 & 2 & 0 & & – & – & – & – & – & – &
–\tabularnewline
Ceratodontiformes & 3 & 3 & 6 & 6 & ~ & – & – & – & – & – & – &
–\tabularnewline
\midrule
Total & 515 & 4,494 & 27,977 & 11,952 & & 194 & 761 & 621 & 961 & 554 &
95 & 2,233\tabularnewline
\bottomrule
\end{longtable}

\end{landscape}
\end{document}  