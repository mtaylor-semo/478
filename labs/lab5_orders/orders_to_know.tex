%!TEX TS-program = lualatex
%!TEX encoding = UTF-8 Unicode

\documentclass[12pt, hidelinks]{exam}
\usepackage{graphicx}
\graphicspath{{/Users/goby/Pictures/teach/434/handouts/}
	{img/}} % set of paths to search for images

\usepackage{geometry}
\geometry{letterpaper, left=1.5in, bottom=1in}                   
%\geometry{landscape}                % Activate for for rotated page geometry
\usepackage[parfill]{parskip}    % Activate to begin paragraphs with an empty line rather than an indent
\usepackage{amssymb, amsmath}
\usepackage{mathtools}
\everymath{\displaystyle}

\usepackage{fontspec}
\setmainfont[Ligatures={TeX}, BoldFont={* Bold}, ItalicFont={* Italic}, BoldItalicFont={* BoldItalic}, Numbers={OldStyle}]{Linux Libertine O}
\setsansfont[Scale=MatchLowercase,Ligatures=TeX]{Linux Biolinum O}
\setmonofont[Scale=MatchLowercase]{Linux Libertine Mono O}
\usepackage{microtype}

% This defines \amper for the fancy ampersand
% to be used in the header. See
% https://tex.stackexchange.com/a/58185/39194
\usepackage{xspace}
\newfontfamily\amperfont[Style=Alternate]{Linux Libertine O}    
\makeatletter
\DeclareRobustCommand{\amper}{{\amperfont\ifx\f@shape\scname\smaller[1.2]\fi\&}\xspace}
\makeatother

% To define fonts for particular uses within a document. For example, 
% This sets the Libertine font to use tabular number format for tables.
%\newfontfamily{\tablenumbers}[Numbers={Monospaced}]{Linux Libertine O}
% \newfontfamily{\libertinedisplay}{Linux Libertine Display O}

\usepackage{booktabs}
\usepackage{multicol}
\usepackage[normalem]{ulem}

\usepackage{longtable}
%\usepackage{siunitx}
\usepackage{array}
\newcolumntype{L}[1]{>{\raggedright\let\newline\\\arraybackslash\hspace{0pt}}p{#1}}
\newcolumntype{C}[1]{>{\centering\let\newline\\\arraybackslash\hspace{0pt}}p{#1}}
\newcolumntype{R}[1]{>{\raggedleft\let\newline\\\arraybackslash\hspace{0pt}}p{#1}}

\usepackage{enumitem}
\usepackage{hyperref}
%\usepackage{placeins} %PRovides \FloatBarrier to flush all floats before a certain point.
\usepackage{hanging}

\usepackage[sc]{titlesec}

\pagestyle{headandfoot}
\firstpageheader{\textsc{zo}\,478/678 Ichthyology}{}{Representative Orders of the World}
\runningheader{}{}{\footnotesize{pg.~\thepage}}
\footer{}{}{}
\runningheadrule

\usepackage{etoolbox}
\patchcmd{\quote}{\rightmargin}{\leftmargin 1em \rightmargin}{}{}


\begin{document}

The following combination of characters will allow you to
diagnose the various specimens in the teaching collection and on
laboratory practicals. Although some characters will diagnose any
species within a group (such as order), some groups can only be
diagnosed by characters not readily viewed in a class setting. However,
the characters provided below work for most species in the group or for
this class. Characters marked with an asterisk (*) are reliable only for taxa in this
teaching collection. Some families are included for information only. 

The number after the order or family corresponds to the number
on the jar lid of the specimen (if the specimen is in a jar).

You are responsible for knowing all taxonomic levels given from class down to order, including  divisions (except where noted) and subdivisions, 

\textbf{\textsc{Class Myxini}}


\begin{quote}
\textbf{Myxiniformes (1)}: eel-like, no paired fins. 3 pairs of barbels
around mouth, 4 rows of teeth on tongue.
\end{quote}

\textbf{\textsc{Class Petromyzontida}}

\begin{quote}
\textbf{Petromyzontiformes (2)}: eel-like, no paired fins. Teeth on oral
disc and tongue, 7 pairs of gill openings, single medial nostril.
\end{quote}

\textbf{\textsc{Class Chondrichthys}}

\textbf{Subclass Holocephali}

\begin{quote}
\textbf{Chimaeriformes (3)}: rat-tail body,large head and pectoral fins,
no spiracle, large plate-like teeth, no placoid scales.
\end{quote}

\textbf{Subclass Elasmobranchii}

\textbf{Division Neoselachii} — \textit{You are not required to know this division but you must know the two subdivisions within it.}

\textbf{Subdivision Selachii }

\begin{quote}
\textbf{Carcharhiniformes (4,5)}: two dorsal fins without spines, anal
fin present. Spiracles usually absent. Scyliorhinidae (4): cat-like
eyes, spiracle \emph{present}. Carcharhinidae (5): eyes not cat-like,
spiracles absent.

\textbf{Squaliformes (6)}: two dorsal fins with or without spines, anal
fin absent. Spiracles present.

Dalatiidae: luminous organs present, appear as black dots on ventral
surface.
\end{quote}

\textbf{Subdivision Batoidea}

\begin{quote}
\textbf{Torpediniformes (7)}: Soft and loose skin, eyes small or absent.
Well-developed caudal fin. 0-2 dorsal fins.

\textbf{Rajiformes (8)}: Skin tough, often with rows of spines (modified
placoid scales). Caudal fin absent to moderately developed. 0-2 dorsal
fins.

\textbf{Myliobatiformes (9)}: large spine on caudal fin forms stinging
spine*.
\end{quote}

\textbf{\textsc{Class Actinopterygii}}

\textbf{Subclass Chondrostei}

\begin{quote}
\textbf{Acipenseriformes (10)}: Body covered in five rows of bony scutes
(large, heavy scales) and four barbels anterior to mouth
\emph{\textbf{or}} large paddle-like snout and without bony scutes.
Heterocercal tail in all extant species. Polyodontidae (10) and
Acipenseridae (sturgeons, no number but in green tub).
\end{quote}

\textbf{Subclass Neopterygii}


\begin{quote}
\textbf{Lepisosteiformes (11):} Elongated lie-in-wait / ambush predator
body form, ganoid scales, elongated jaws with sharp teeth.

\textbf{Amiiformes (12)}: Large median gular plate, located between arms
of lower jaw.
\end{quote}

\textbf{Division Teleostei}

\textbf{Subdivision Osteoglossomorpha} — \textit{“Osteoglosso-” means “bony tongue.”}

\begin{quote}
\textbf{Hiodontiformes (13)}: Teeth on bony tongues.*
\end{quote}

\textbf{Subdivision Elopomorpha} — \textit{This subdivision shares a homology of \textbf{leptocephalus larvae.} Here includes only Elopiformes and Anguilliformes.}

\begin{quote}
\textbf{Elopiformes} \textbf{(15)}: gular plate, slender body with
abdominal pelvic fins, wide gill openings, deeply forked caudal fin.

\textbf{Anguilliformes (16,17)}: eel-like bodies, lack pelvic fins,
pectorals sometimes absent. Dorsal and anal fins confluent with caudal
fin (all run together). Muraenidae (16): Pectoral fins absent.
Ophichthyidae (17): Pectoral fins present
\end{quote}

\textit{The remaining orders fall into various divisions and other groups that you do not need to know except for the series Otophysi.}

\begin{quote}
\textbf{Clupeiformes (19)}: Silvery, slim (laterally compressed) body,
with smal scutes along midline of abdomen.
\end{quote}

\textbf{Series Otophysi} — \textit{The Cyprinformes, Characiformes, and Siluriformes (plus Gymnotiformes, not included here) all share the homology of the Weberian apparatus for sound detection (see lecture). I referred to these orders in lecture as the otophysan fishes.  }

\begin{quote}
\textbf{Cyprinformes (20, 21)}: Mouth toothless, well-developed
pharyngeal teeth, cycloid scales, no adipose fin, head scaleless. Upper
jaw protractile (can be extended). Cyprinidae (20): minnows and shiners;
Catostomidae (21): suckers.

\textbf{Characiformes (22)}: Teeth well-developed, adipose fin is
present. Upper jaw not protractile.

\textbf{Siluriformes: (23, 24)}: Adipose fin usually present. Up to
three pairs of barbels: 1 nasal (upper lip), 1 maxillar (corners of the
mouth), 1 mental (chin); nasal or mental barbels occasionally lost.
Usually scaleless, although some have bony armor. Loricariidae (23;
adipose fin absent*), Bagridae (24).

\textbf{Esociformes (26)}: Adipose fin absent, fine scales, maxilla
without teeth but forms the gape of the mouth. Lie-in-wait / ambush
predator body form.

\textbf{Salmoniformes (28)}: Adipose fin present, fine scales*,
rover-predator body form.

\textbf{Stomiiformes (29, 30, 31)}: Sharp elongated or bristle-like
teeth, adipose fin present, ventral rows of photophores.

\textbf{Myctophiformes (33)}: Weakly toothed, none elongated or
bristle-like. Adipose fin present, scattered photophores on body. Larger
luminescent organs often found on head or caudal peduncle.

\textbf{Percopsiformes} \textbf{(34)}: Anus located at throat.

\textbf{Gadiformes (36, 37)}: Pelvics, if present, usually thoracic or
jugular, chin barbel present*, long dorsal and anal fins. Macrouridae
(36), Gadidae (37).

\textbf{Batrachoidiformes (38)}: large head, eyes somewhat dorsally
placed, jugular pelvics, scaleless.

\textbf{Lophiiformes (39, 40)}: illicium and esca present ("fishing rod and lure") on top of head. Pelvic fins, if
present, in advance of pectoral fins. Antennariidae (39), Ogcocephalidae
(40; extreme dorsoventral flattening).

\textbf{Mugiliformes (41)}: Two dorsal fins, widely separated; spines
restricted to first dorsal fin. Mouth terminal. Pectoral fins very high
on body, pelvic fins abdominal.

\textbf{Atheriniformes (42)}: Two dorsal fins, not as widely separated
as Mugiliformes, flexible spines in first dorsal fin, one spine in
second dorsal. Pectoral fins high on body, pelvics abdominal. Mouth
tends to be supraterminal to superior.

\textbf{Beloniformes (43, 44)}: Single dorsal fin. Pectoral fins high on
body, pelvics abdominal. Terminal mouth. Pectoral fins (and sometimes
pelvic fins) greatly enlarged \emph{\textbf{or}} elongated, gar-like
body. Exoceotidae (43), Belonidae (44).

\textbf{Cyprinodontiformes (45, 46)}: Single dorsal fin not as high on
body as previous three orders, pelvics abdominal, mouth supraterminal to
superior. Fundulidae (45), Poeciliidae (46).

\textbf{Gasterosteiformes (47, 48)}: Body usually with rings or plates
of armor. Mouth usually small. Gasterosteidae (47), Syngnathidae (48).

\textbf{Scorpaeniformes (49, 51)}: Suborbital bony stay. Body often
spiny or with bony plates. Caudal fin usually rounded. Scorpaenidae
(49), Cottidae (51).

\textbf{Perciformes (55, 58, 60)}: Two dorsal fins present, first
spinous, second with at least some rays although spines may be present.
Dorsal fins either continuous or separate. Pelvics usually with one
spine. Pectoral fins laterally placed, pelvics thoracic, occasionally
jugular. Sciaenidae (55), Pomacentridae (58), Blennidae (60).%, Osphronemidae (65).

\textbf{Pleuronectiformes (66)}: Both eyes on one side of body. Body is
\emph{not} dorsoventrally flattened. It is laterally compressed and laying on one side.

\textbf{Tetraodontiformes (69, 71)}: Jaws fused, four teeth. Body rigid
or otherwise inflexible (much of skeleton is fused or otherwise
reduced). Skin typically covered in tough scales or sharp spines.
Monacanthidae (69), Diodontidae (71).
\end{quote}


\end{document}  