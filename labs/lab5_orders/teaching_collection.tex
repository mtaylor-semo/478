%!TEX TS-program = lualatex
%!TEX encoding = UTF-8 Unicode

\documentclass[12pt, hidelinks]{exam}
\usepackage{graphicx}
\graphicspath{{/Users/goby/Pictures/teach/434/handouts/}
	{img/}} % set of paths to search for images

\usepackage{geometry}
\geometry{letterpaper, left=1.5in, bottom=1in}                   
%\geometry{landscape}                % Activate for for rotated page geometry
\usepackage[parfill]{parskip}    % Activate to begin paragraphs with an empty line rather than an indent
\usepackage{amssymb, amsmath}
\usepackage{mathtools}
\everymath{\displaystyle}

\usepackage{fontspec}
\setmainfont[Ligatures={TeX}, BoldFont={* Bold}, ItalicFont={* Italic}, BoldItalicFont={* BoldItalic}, Numbers={OldStyle}]{Linux Libertine O}
\setsansfont[Scale=MatchLowercase,Ligatures=TeX]{Linux Biolinum O}
\setmonofont[Scale=MatchLowercase]{Linux Libertine Mono O}
\usepackage{microtype}

% This defines \amper for the fancy ampersand
% to be used in the header. See
% https://tex.stackexchange.com/a/58185/39194
\usepackage{xspace}
\newfontfamily\amperfont[Style=Alternate]{Linux Libertine O}    
\makeatletter
\DeclareRobustCommand{\amper}{{\amperfont\ifx\f@shape\scname\smaller[1.2]\fi\&}\xspace}
\makeatother

% To define fonts for particular uses within a document. For example, 
% This sets the Libertine font to use tabular number format for tables.
%\newfontfamily{\tablenumbers}[Numbers={Monospaced}]{Linux Libertine O}
% \newfontfamily{\libertinedisplay}{Linux Libertine Display O}

\usepackage{booktabs}
\usepackage{multicol}
\usepackage[normalem]{ulem}

\usepackage{longtable}
%\usepackage{siunitx}
\usepackage{array}
\newcolumntype{L}[1]{>{\raggedright\let\newline\\\arraybackslash\hspace{0pt}}p{#1}}
\newcolumntype{C}[1]{>{\centering\let\newline\\\arraybackslash\hspace{0pt}}p{#1}}
\newcolumntype{R}[1]{>{\raggedleft\let\newline\\\arraybackslash\hspace{0pt}}p{#1}}

\usepackage{enumitem}
\usepackage{hyperref}
%\usepackage{placeins} %PRovides \FloatBarrier to flush all floats before a certain point.
\usepackage{hanging}

\usepackage[sc]{titlesec}

\pagestyle{headandfoot}
\firstpageheader{\textsc{zo}\,478/678 Ichthyology}{}{Representative Families of the World}
\runningheader{}{}{\footnotesize{pg.~\thepage}}
\footer{}{}{}
\runningheadrule

\newcommand{\onedent}{\hspace*{1em}}
\newcommand{\twodent}{\hspace*{2em}}
\newcommand{\thrdent}{\hspace*{3em}}

\begin{document}

\noindent\begin{multicols}{2}


\textbf{\textsc{ Class Myxini}} \\
\onedent Myxiniformes\\
\twodent Myxinidae (1)

\textbf{\textsc{Class Petromyzontida}}\\
\onedent Petromyzontiformes \\
\twodent Petromyzontidae (2)

\textbf{\textsc{Class Chondrichthys}}\\
\textbf{Subclass Holocephali} \\
\onedent Chimaeriformes \\
\twodent Chimaeridae (3)

\textbf{Subclass Elasmobranchii}\\
\hspace*{0.33em}\textbf{Division Neoselachii\footnote{Not required to know}}\\
\hspace*{0.67em}\textbf{Subdivision Selachii }\\
\onedent Carcharhiniformes \\
\twodent Scyliorhinidae (4) \\
\twodent Carcharhinidae (5) \\
\onedent Squaliformes \\
\twodent Dalatiidae (6)

\hspace*{0.67em}\textbf{Subdivision Batoidea}\\
\onedent Torpediniformes \\
\twodent Torpedinidae (7) \\
\onedent Rajiformes \\
\twodent Rajidae (8) \\
\onedent Myliobatiformes \\
\twodent Dasyatidae (9)

\textbf{\textsc{Class Actinopterygii}}\\
\textbf{Subclass Chondrostei} \\
\onedent Acipenseriformes\\ 
\twodent Polyodontidae (10)

\textbf{Subclass Neopterygii}\\
\onedent Lepisosteiformes \\
\twodent Lepisosteidae (11) \\
\onedent Amiiformes\\
\twodent Amiidae (12)\\
\columnbreak

\textbf{\textsc{Class Actinopterygii} (con't)}\\
\textbf{Subclass Neopterygii (con't)}\\
\hspace*{0.67em}\textbf{Division Teleostei}\\
\onedent Hiodontiformes\\
\twodent Hiodontidae (13)\\
\onedent Osteoglossiformes\\
\twodent Ignore if no spm (14)\\
\onedent Elopiformes\\
\twodent Elopidae (15)\\
\onedent Anguilliformes\\
%Muraenoidei\\
\twodent Muraenidae (16)\\
%Congroidei\\
\twodent Ophichthyidae (17)\\
\onedent Clupeiformes\\
\twodent Engraulidae (18)\\
\twodent Clupeidae (19)\\
\onedent Cypriniformes\\
\twodent Cyprinidae (20)\\
\twodent Catostomidae (21)\\
\onedent Characiformes\\
\twodent Characidae (22)\\
\onedent Siluriformes\\
\twodent Loricariidae (23)\\
\twodent Bagridae (24)\\
\onedent Gymnotiformes\\
\twodent ignore if no spm (25)\\
\onedent Esociformes\\
\twodent Esocidae (26)\\
\twodent Umbridae (27)\\
\onedent Salmoniformes\\
\twodent Salmonidae (28)\\
\onedent Stomiiformes\\
%Gonostomatoidei\\
\twodent Gonostomatidae (29)\\
\twodent Sternoptychidae (30)\\
%Phosichthyoidei\\
\twodent Stomiidae (31)\\
\onedent Aulopiformes\\
\twodent Notosudidae (32)\\
\onedent Myctophiformes\\
\twodent Myctophidae (33)\\ 
\onedent Percopsiformes\\
\twodent Aphredoderidae (34)\\
\twodent Amblyopsidae (35)\\

\textbf{Subclass Neopterygii (con't)}\\
\hspace*{0.67em}\textbf{Division Teleostei (con't)}\\
\onedent Gadiformes\\
\twodent Macrouridae (36)\\
\twodent Gadidae (37)\\
\onedent Batrachoidiformes\\
\twodent Batrachoididae (38)\\
\onedent Lophiiformes\\
%Antennarioidei\\
\twodent Antennariidae (39)\\
%Ogcocephalioidei\\
\twodent Ogcocephalidae (40)\\
\onedent Mugiliformes\\
\twodent Mugilidae (41)\\
\onedent Atheriniformes\\
\twodent Atherinidae (42)\\
\onedent Beloniformes\\
\twodent Exoceotidae (43)\\
\twodent Belonidae (44)\\
\onedent Cyprinodontiformes\\
\twodent Fundulidae (45)\\
\twodent Poeciliidae (46)\\
\onedent Gasterosteiformes\\
%Gasterosteoidei\\
\twodent Gasterosteidae (47)\\
%Syngnathoidei\\
\twodent Syngnathidae (48)\\
\onedent Scorpaeniformes\\
%Scorpaenoidei\\
\twodent Scorpaenidae (49)\\
\twodent Triglidae (50)\\
%Cottoidei\\
\twodent Cottidae (51)\\
\textbf{Subclass Neopterygii (con't)}\\
\hspace*{0.67em}\textbf{Division Teleostei (con't)}\\
\onedent Perciformes\\
%Percoidei\\
\twodent Moronidae (52)\\
\twodent Echeneidae (53)\\
\twodent Carangidae (54)\\
\twodent Sciaenidae (55)\\
%Elassomatoidei\\
\twodent Elassomatidae (56)\\
%Labroidei\\
\twodent Cichlidae (57)\\
\twodent Pomacentridae (58)\\
%Trachinoidei\\
\twodent Uranoscopidae (59)\\
%Blennioidei\\
\twodent Blennidae (60)\\
%Gobiesocoidei\\
\twodent Gobiesocidae (61)\\
%Gobioidei\\
\twodent Gobiidae (62)\\
%Scombroidei\\
\twodent Scombridae (63)\\
%Stromateoidei\\
\twodent Stromateidae (64)\\
%Anabantoidei\\
\twodent Osphronemidae (65)\\
\onedent Pleuronectiformes\\
\twodent Pleuronectidae (66)\\
\twodent Paralichthyidae (67)\\
\twodent Achiridae (68)\\
\onedent Tetraodontiformes\\
%Balistoidei\\
\twodent Monacanthidae (69)\\
%Tetraodontoidei\\
\twodent Tetraodontidae (70)\\
\twodent Diodontidae (71)

\textbf{\textsc{Class Sarcopterygii}}\\
No specimens, unfortunately

\end{multicols}

\end{document}  