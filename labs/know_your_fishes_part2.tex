%!TEX TS-program = lualatex
%!TEX encoding = UTF-8 Unicode

\documentclass[11pt]{article}
\usepackage{graphicx}
	\graphicspath{{/Users/goby/Pictures/teach/478/lab/}} % set of paths to search for images

\usepackage{geometry}
\geometry{letterpaper}                   
%\geometry{landscape}                % Activate for for rotated page geometry
%\usepackage[parfill]{parskip}    % Activate to begin paragraphs with an empty line rather than an indent
\usepackage{amssymb}
\usepackage{mathtools}
	\everymath{\displaystyle}

\usepackage{fontspec}
\setmainfont[Ligatures={Common}, BoldFont={* Bold}, ItalicFont={* Italic}, Numbers={Proportional}]{Linux Libertine O}
\setsansfont[Scale=MatchLowercase,Ligatures=TeX]{Linux Biolinum O}
\setmonofont[Scale=MatchLowercase]{Inconsolata}
\usepackage{microtype}

\usepackage{unicode-math}
\setmathfont[Scale=MatchLowercase]{Asana-Math.otf}
%\setmathfont{XITS Math}

% To define fonts for particular uses within a document. For example, 
% This sets the Libertine font to use tabular number format for tables.
\newfontfamily{\tablenumbers}[Numbers={Monospaced}]{Linux Libertine O}
\newfontfamily{\libertinedisplay}{Linux Libertine Display O}


%\usepackage{booktabs}
%\usepackage{multicol}
%\usepackage{longtable}
%\usepackage{siunitx}
%\usepackage[justification=raggedright, singlelinecheck=off]{caption}
%\captionsetup{labelsep=period} % Removes colon following figure / table number.
%\captionsetup{tablewithin=none}  % Sequential numbering of tables and figures instead of
%\captionsetup{figurewithin=none} % resetting numbers within each chapter (Intro, M&M, etc.)
%\captionsetup[table]{skip=0pt}

%\usepackage{array}
%\newcolumntype{L}[1]{>{\raggedright\let\newline\\\arraybackslash\hspace{0pt}}p{#1}}
%\newcolumntype{C}[1]{>{\centering\let\newline\\\arraybackslash\hspace{0pt}}p{#1}}
%\newcolumntype{R}[1]{>{\raggedleft\let\newline\\\arraybackslash\hspace{0pt}}p{#1}}

\usepackage{enumitem}
\setlist[enumerate]{leftmargin=1em}
%\usepackage{hyperref}
%\usepackage{placeins} %PRovides \FloatBarrier to flush all floats before a certain point.

\usepackage{titling}
\setlength{\droptitle}{-60pt}
\posttitle{\end{center}}
\predate{}\postdate{}

\newcommand{\VSpace}{\vspace{\baselineskip}}

\title{Cheat Sheet: Ictaluridae, Fundulidae, Centrachidae, Percidae}
\author{ZO 478 / 678}
\date{}                                           % Activate to display a given date or no date

\begin{document}
\maketitle

\subsection*{Ictaluridae\,—\,catfishes}

The three genera of catfishes listed here are easy to distinguish.  \textit{Ictalurus} has a free adipose fin and homocercal tail. \textit{Ameiurus} has a free adipose fin and an isocercal tail.\textit{Noturus} has an adipose fin that merges with the isocercal tail.

\begin{enumerate}
	\item \textit{Ameiurus natalis}\,—\,Yellow Bullhead.  Note the white chin barbels.  Probably our most common bullhead around here, typically inhabiting pools and debris areas of small streams. \textit{A. melas}, which you are not required to know, is similar but has dusky chin barbels.

	\item \textit{Ictalurus furcatus} - Blue Catfish.  This is our largest species of catfish in the state, primarily confined to large rivers and reservoirs.  Note straight-edged anal fin and forked caudal fin. The eyes are smaller than those of \textit{I. punctatus} (compare side by side).

	\item \textit{Ictalurus punctatus}\,—\,Channel Catfish.  This common species has a forked caudal fin, but has has a shorter, more rounded anal fin compared to \textit{I. furcatus}.  This species is typically spotted when young…however, some juvenile \textit{I. punctatus} lack spots and rarely \textit{I. furcatus} young have spots. The eyes are larger than those of \textit{I. furcatus} (compare side by side)

	\item \textit{Noturus exilis}\,—\,Slender Madtom.  Dark edging on dorsal, caudal and anal fins. Upper jaw does not extend past lower jaw. Long and slender compared to most other madtoms. Lighter color than \textit{N. nocturnus}.

	\item \textit{Noturus nocturus}\,—\,Freckled Madtom. Uniformly dusky. Upper jaw extends past lower jaw. Lower lip and chink heavily sprinkled with melanophores.

\end{enumerate}

\subsection*{Fundulidae\,—\,topminnows}

\begin{enumerate}
	\item\textit{Fundulus catenatus}\,—\,Northern Studfish. Does not have a broad dark stripe on the side of the body. Instead, has many narrow horizontal streaks extending from behind head to caudal fin.

	\item\textit{Fundulus notatus}\,—\,Blackstripe Topminnow. This species typically has no or few black spots above the dark lateral stripe. The spots, if present, are not as dark as the lateral stripe and tend to be irregular in outline. Compare side by side with \textit{F. olivaceus}. 

	\item\textit{Fundulus olivaceus}\,—\,Blackspotted Topminnow. This species has few to many dark spots above the dark lateral stripe. The spots are about as dark as the lateral stripe and regular in outline. Compare side by side with \textit{F. notatus}. 
\end{enumerate}

\subsection*{Centrarchidae\,—\,sunfishes}

Both \textit{Lepomis} and \textit{Micropterus} have three anal spines (rarely 2–4). \textit{Lepomis} has a deep, rounded body. \textit{Micropterus} has a more elongated body.  Both \textit{Ambloplites} and \textit{Pomoxis} have at least five anal spines (usually 5–8). \textit{Ambloplites} has a longer spinous dorsal fin with with 11–13 spines. \textit{Pomoxis} has a shorter spinous dorsal fin with 6–8 spines.

\begin{enumerate}
	\item\textit{Ambloplites ariommus}\,—\,Shadow Bass. Shaped somewhat like an elongated \textit{Lepomis} but it has a larger mouth and distinctly dark broad vertical blotches on the sides.

	\item\textit{Lepomis cyanellus}\,—\,Green Sunfish. Tends to have moderately but uniformly darkish body. Has a moderately large mouth compared to most \textit{Lepomis} but not as large as \textit{L. gulosus}. Has a dark botch at the rear base of the soft dorsal fin. Tends to be somewhat ventrally flatted. Has a short, rounded pectoral fin.

	\item\textit{Lepomis gulosus}\,—\,Warmouth. Has three or more lines extended from the back of the eye to the rear of the operculum. Has a large mouth compared to all other \textit{Lepomis}. Has a patch of teeth on the tongue. Has a short, rounded pectoral fin.

	\item\textit{Lepomis macrochirus}\,—\,Bluegill. Deep, rounded body. Smaller individuals have distinct pattern of vertical bars that is different from \textit{L. megalotis}. Small specimens tend to be lighter in color than \textit{L. megalotis}. Ear flap not as long as \textit{L. megalotis}. Has a dark blotch at the rear base of the soft dorsal fin. Has a long, pointed pectoral fin. Compare small individuals with \textit{L. megalotis}.

	\item\textit{Lepomis megalotis}\,—\,Longear Sunfish. Deep, rounded body. Smaller individuals with distinct pattern of vertical bars that is different from \textit{L. macrochirus}. Small specimens tend to be darker in color than \textit{L. macrochirus}. Ear flap longer in large individuals. Does not have a dark blotch at the base of the soft dorsal fin. Has a short, rounded pectoral fin. Compare small individuals with \textit{L. macrochirus}.

	\item\textit{Micropterus dolomieu}\,—\,Smallmouth Bass. Lower jaw does not extend past rear of eye. Does not have have dark lateral stripe. May show moderately dark vertical bars. Overall body color tends to be darker than other \textit{Micropterus}. Small individuals have tri-colored tail fin (light caudal fin base, dark band, light outer edge) without a distinct spot at base of caudal fin.

	\item\textit{Micropterus punctulatus}\,—\,Spotted Bass. Lower jaw does not extend past rear of eye. Has a dark lateral stripe. Ventral belly below lateral stripe has series of horizontal streaks. Small individuals have tri-colored tail fin (light caudal fin base, dark band, light outer edge) with a distinct spot at base of caudal fin.

	\item\textit{Micropterus salmoides}\,—\,Largemouth Bass. Lower jaw extends past rear of eye. Has a dark lateral stripe. Ventral belly below lateral stripe does not have any horizontal streaks. Small individuals have a bi-colored caudal fin (light base, dark outer edge). 

	\item\textit{Pomoxis annularis}\,—\,White Crappie. Six dorsal spines. Body has six vertical bars (annuli, hence the scientific name). Head proportionately longer than \textit{P. nigromaculatus}.

	\item\textit{Pomoxis nigromaculatus}\,—\,Black Crappie. 7–8 dorsal spines. Body has irregular blotches instead of distinct vertical bars. Head proportionately shorter than \textit{P. annularis}.
\end{enumerate}

\subsection*{Percidae\,—\,perches}

Most perches in North America are darters.

\begin{enumerate}
	\item\textit{Ammocrypta clara}\,—\,Western Sand Darter. Body long and slender compared to \textit{Etheostoma} or \textit{Percina}. Body very pale in color usually without dark markings on side. If markings are present on body, they are very weak.

	\item\textit{Etheostoma caeruleum}\,—\,Rainbow Darter. Pointed snout. Gill covers narrowly connected by membrane (see couplet 24, page 306 in Pflieger). Body depth greatest under the spinous dorsal fin. Does not have any horizontal streaks along the side of the body. The vertical bars on the body tend to connect to the dark saddles across the dorsal side of the fish. Adult males in life have orange and blue in anal fin. Compare side by side with \textit{E. spectabile}. 

	\item\textit{Etheostoma spectabile}\,—\,Orangethroat Darter. Pointed snout. Gill covers narrowly connected by membrane (see couplet 24, page 306 in Pflieger). Body depth greatest before the spinous dorsal fin. Has thin horizontal streaks along the side of the body. The vertical bars on the body generally do not connect to the dark saddles on the dorsal side of the fish. Adult males in life do not have orange in anal fin, only blue. Compare side by side with \textit{E. caeruleum}. 

	\item\textit{Etheostoma zonale}\,—\,Banded Darter. Rounded snout. Gill covers broadly connected by membrane (see couplet 24, page 306 in Pflieger). Has 6–7 dark saddle marks across the dorsal side of the fish. Tends to have banded pattern in pectoral fin.

	\item\textit{Percina caprodes}\,—\,Logperch. Subterminal mouth with distinctly long conical snout. 15–20 narrow vertical bars connected across the dorsal side.

	\item\textit{Percina sciera}\,—\,Dusky Darter. Side of body with row of dark blotches that occasionally connect to give appearance of blotchy lateral stripe. Base of caudal fin with three dark spots. First spot is at top of caudal fin, second spot at middle of caudal fin, third spot at base of caudal fin. Lower two spots sometimes merge together to form an irregular bar but the spots can usually recognized in the bar.
\end{enumerate}

\end{document}  