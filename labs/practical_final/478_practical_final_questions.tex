%!TEX TS-program = lualatex
%!TEX encoding = UTF-8 Unicode

\documentclass{exam}
\usepackage[left=0.75in,right=0.75in,top=0.5in, bottom=0.5in]{geometry}     

\usepackage{fontspec}
\def\mainfont{Linux Libertine O}
\setmainfont[Ligatures=TeX, Contextuals={NoAlternate}, BoldFont={* Bold}, ItalicFont={* Italic}, Numbers={Proportional}]{\mainfont}
%\setmonofont[Scale=MatchLowercase]{Linux Libertine Mono O} 
%\setsansfont[Scale=MatchLowercase]{Linux Biolinum O} 
\usepackage{microtype}

\pagestyle{head}

%\pagenumbering{gobble}
% \setlength{\parindent}{0pt}

\renewcommand{\questionshook}{%
	\setlength{\leftmargin}{0pt}
	\setlength{\labelwidth}{\parindent}}

\newcommand{\station}[1]{\fullwidth{\textbf{Station~#1}}}

\begin{document}

\begin{questions}

{\Large 
\station{1}
%1
\question Class for the left fish.

%2
\question Order for the left fish.
%\vspace{2\baselineskip}

%3
\question Class for the right fish.

%4
\question Order for the right fish.

%5 
\question Name a diagnostic character that you used to distinguish between the two.

%6ec
\bonusquestion \textsc{(ec)} Which one produces slime?

\vspace{2\baselineskip}


\station{2}
%7
\question Order for the left fish?
%\vspace{2\baselineskip}

%8
\question Order for the right fish?.

%9
\question Series for both fishes.

%10
\question What diagnostic structure unites these together into a monophyletic group?
\vspace{2\baselineskip}

\station{3}
%11
\question Order of the left fish.

%12
\question Order of the right fish.

%13
\question Division of both fishes.

\vspace{2\baselineskip}

\station{4}
%14
\question Order of the upper fish.

%15
\question Order of the lower fish.

%16
\question Given their form, what is the likely function of this fishes?

\vspace{2\baselineskip}

\newpage

\station{5}
%17
\question Order of the left fish.

%18
\question Order of the right fish.

%19
\question What type of deep-sea fishes are these; i.e., in what depth zone of the open ocean would you find them?

\vspace{2\baselineskip}


\station{6}
%20
\question Order of the left fish.

%21ec
\bonusquestion \textsc{(ec)} Family of the left fish.

%22
\question Order of the right fish.

%23
\question What type of larvae unite these two orders together in the Elopomorpha?

%24ec
\bonusquestion \textsc{(ec)} Given the form, what is the likely function of the left fish?
\vspace{2\baselineskip}

%\newpage

\station{7}
%25
\question Class for this fish.

%26
\question Subclass for this fish.

%27
\question Order for this fish. 
\vspace{2\baselineskip}

\station{8}
%28
\question Subclass of the upper fish. 

%29
\question Order of the upper fish.

%30
\question Subclass of the lower fish.

%31
\question Order of the lower fish.
%\vspace{2\baselineskip}

\newpage

\station{9}
%32
\question Subdivision of the upper fish.
%\vspace{2\baselineskip}

%33
\question Order of the upper fish.

%34
\question Subdivision of the lower fish.

%35
\question Order of the lower fish.

\vspace{2\baselineskip}
%\newpage


\station{10}
%36
\question Order of the left fish.

%37 
\question Order of the center fish.

%38
\question Order of the right fish.
\vspace{2\baselineskip}

\station{11}
%39
\question Order for the left fish.

%40
\question Order for the right fish.
\vspace{2\baselineskip}

\station{12}
%41
\question What is the class of this fish?

%42
\question Order for this fish

%43
\question What character is diagnostic for this fish?


%44
\question \textsc{(ec)} What is your favorite fish?
\vspace{2\baselineskip}


%45
\question What class of fishes was not represented today? (Save this question for last)
\vspace{2\baselineskip}

} %End large font

\end{questions}


%\newpage



%}%End large font size


\end{document}