%!TEX TS-program = lualatex
%!TEX encoding = UTF-8 Unicode

\documentclass[12pt, hidelinks]{exam}
\usepackage{graphicx}
\graphicspath{{/Users/goby/Pictures/teach/434/handouts/}
	{img/}} % set of paths to search for images

\usepackage{geometry}
\geometry{letterpaper, left=1.5in, bottom=1in}                   
%\geometry{landscape}                % Activate for for rotated page geometry
\usepackage[parfill]{parskip}    % Activate to begin paragraphs with an empty line rather than an indent
\usepackage{amssymb, amsmath}
\usepackage{mathtools}
\everymath{\displaystyle}

\usepackage{fontspec}
\setmainfont[Ligatures={TeX}, BoldFont={* Bold}, ItalicFont={* Italic}, BoldItalicFont={* BoldItalic}, Numbers={OldStyle}]{Linux Libertine O}
\setsansfont[Scale=MatchLowercase,Ligatures=TeX]{Linux Biolinum O}
\setmonofont[Scale=MatchLowercase]{Linux Libertine Mono O}
\usepackage{microtype}

% This defines \amper for the fancy ampersand
% to be used in the header. See
% https://tex.stackexchange.com/a/58185/39194
\usepackage{xspace}
\newfontfamily\amperfont[Style=Alternate]{Linux Libertine O}    
\makeatletter
\DeclareRobustCommand{\amper}{{\amperfont\ifx\f@shape\scname\smaller[1.2]\fi\&}\xspace}
\makeatother

% To define fonts for particular uses within a document. For example, 
% This sets the Libertine font to use tabular number format for tables.
%\newfontfamily{\tablenumbers}[Numbers={Monospaced}]{Linux Libertine O}
% \newfontfamily{\libertinedisplay}{Linux Libertine Display O}

\usepackage{booktabs}
\usepackage{multicol}
\usepackage[normalem]{ulem}

\usepackage{longtable}
%\usepackage{siunitx}
\usepackage{array}
\newcolumntype{L}[1]{>{\raggedright\let\newline\\\arraybackslash\hspace{0pt}}p{#1}}
\newcolumntype{C}[1]{>{\centering\let\newline\\\arraybackslash\hspace{0pt}}p{#1}}
\newcolumntype{R}[1]{>{\raggedleft\let\newline\\\arraybackslash\hspace{0pt}}p{#1}}

\usepackage{enumitem}
\usepackage{hyperref}
%\usepackage{placeins} %PRovides \FloatBarrier to flush all floats before a certain point.
\usepackage{hanging}

\usepackage[sc]{titlesec}

\pagestyle{headandfoot}
\firstpageheader{\textsc{zo}\,478/678 Ichthyology}{}{}
\runningheader{}{}{\footnotesize{pg. \thepage}}
\footer{}{}{}
\runningheadrule

\title{External and internal anatomy}
\author{ZO 478 / 678}
\date{}                                           % Activate to display a given date or no date

\begin{document}

You must be able to identify the following external and internal morphological and anatomical structures for the first lab exam.

\noindent\begin{multicols}{3}
\columnbreak\textbf{External} \\
placoid scale \\
ganoid scale \\
cycloid scale \\
ctenoid scale \\
superior mouth  \\
terminal mouth  \\
subterminal mouth  \\
inferior mouth  \\
abdominal fin position \\
thoracic fin position \\
jugular fin position \\
homocercal tail \\
lunate tail \\
heterocercal tail \\
isocercal tail \\
spinous dorsal fin \\
soft dorsal fin \\
anal fin \\
caudal fin \\
caudal peduncle \\
pectoral fins \\
pelvic fins \\
adipose fin \\
barbels \\
operculum \\
lateral line \\

\columnbreak\textbf{Internal, non-skeletal} \\
gill rakers \\
gill arch \\
gill filaments \\
myomere \\
epaxial muscles \\
hypaxial muscles \\
horizontal septum \\
vertical septum \\
heart \\
sinus venosus \\
atrium \\
ventricle \\
bulbus/conus arteriosus \\
ventral aorta \\
dorsal aorta \\
swim bladder \\
gas gland \\ %rete mirabile \\
liver \\
gall bladder \\
kidney \\
stomach \\
pyloric caeca \\
intestine \\
anus \\
testis \\
ovary \\

\columnbreak\textbf{Internal, skeletal} \\
opercle \\
pre-opercle \\
premaxilla \\
maxilla \\
dentary \\
abdominal vertebra \\
caudal vertebra \\
neural spine \\
neural arch \\
neural canal \\
hemal spine \\
hemal arch \\
hemal canal \\
\end{multicols}

\end{document}  