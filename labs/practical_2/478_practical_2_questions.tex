%!TEX TS-program = lualatex
%!TEX encoding = UTF-8 Unicode

\documentclass{article}
\usepackage[left=0.75in,right=0.75in,top=0.75in, bottom=0.75in]{geometry}     

\usepackage{fontspec}
\def\mainfont{Linux Libertine O}
\setmainfont[Ligatures=TeX, Contextuals={NoAlternate}, BoldFont={* Bold}, ItalicFont={* Italic}, Numbers={Proportional}]{\mainfont}
\setsansfont[Scale=MatchLowercase]{Linux Biolinum O} 
\usepackage{microtype}

\pagenumbering{gobble}
 \setlength{\parindent}{0pt}

\begin{document}

% Station 1
{\Large 
1. Name the family.
\vspace{0.5\baselineskip}

2. Name the genus.
\vspace{0.5\baselineskip}

3. Name the species.
\vspace{0.5\baselineskip}

Extra Credit: Name the order.
\vspace{2\baselineskip}


% Station 2
4. Name the family.
\vspace{0.5\baselineskip}

5. Name the genus.
\vspace{0.5\baselineskip}

6. Name the species.
\vspace{2\baselineskip}


% Station 3
7. Name the class.
\vspace{0.5\baselineskip}

8. Name the family.
\vspace{0.5\baselineskip}

9. Name the genus.
\vspace{0.5\baselineskip}

Extra Credit: This species sees very well in the dim light. Name the reflective layer at the back of the eye that maximizes the ability of this species to capture available light.
\vspace{2\baselineskip}


% Station 4
10. Name the genus.
\vspace{0.5\baselineskip}

11. Name the species (top; smaller specimen).
\vspace{0.5\baselineskip}

12. Name the species (bottom; larger specimen).
\vspace{2\baselineskip}


% Station 5
13. Name the superclass.
\vspace{0.5\baselineskip}

14. Name the family.
\vspace{0.5\baselineskip}

15. Name the genus.
\vspace{2\baselineskip}

\newpage

% Station 6
16. Name the genus (left, smaller fish).
\vspace{0.5\baselineskip}

17. Name the species.
\vspace{0.5\baselineskip}

18. Name the subclass (right, larger fish).
\vspace{0.5\baselineskip}

19. Name the genus.
\vspace{2\baselineskip}


% Station 7
20. Name the genus (left, larger fish).
\vspace{0.5\baselineskip}

21. Name the species.
\vspace{0.5\baselineskip}

22. Name the genus (right, smaller fish).
\vspace{0.5\baselineskip}

23. Name the species.
\vspace{2\baselineskip}


% Sation 8
24. Name the family.
\vspace{0.5\baselineskip}

25. Name the genus.
\vspace{0.5\baselineskip}

26. Name the species.
\vspace{2\baselineskip}


% Station 9
27. Name the division.
\vspace{0.5\baselineskip}

28. Name the genus.
\vspace{0.5\baselineskip}

29. Name the species.
\vspace{2\baselineskip}


% Station 10
30. Name the genus (left, smaller fish).
\vspace{0.5\baselineskip}

31. Name the species.
\vspace{0.5\baselineskip}

32. Name the family (right, larger fish).
\vspace{0.5\baselineskip}

33. Name the genus
\vspace{0.5\baselineskip}

34. Name the species
\vspace{2\baselineskip}
%
\newpage

% Station 11
35. Name the family (left, smaller fish).
\vspace{0.5\baselineskip}

36. Name the genus.
\vspace{0.5\baselineskip}

37. Name the species.
\vspace{0.5\baselineskip}

38. Name the genus (right fish).
\vspace{0.5\baselineskip}

39. Name the species.
\vspace{2\baselineskip}



% Station 12
40. Name the subclass.
\vspace{0.5\baselineskip}

41. Name the genus.
\vspace{0.5\baselineskip}

42. Name the species.
\vspace{2\baselineskip}


% Station 13
43. Name the genus.
\vspace{0.5\baselineskip}

44. Name the species.
\vspace{0.5\baselineskip}

Extra Credit: Name the form/function of this fish.
\vspace{2\baselineskip}


% Station 14
45. Name the genus (left, larger fish).
\vspace{0.5\baselineskip}

46. Name the species.
\vspace{0.5\baselineskip}

47. Name the genus (right, smaller fish).
\vspace{0.5\baselineskip}

48. Name the species.
\vspace{2\baselineskip}


% Station 15
49. Name the genus.
\vspace{0.5\baselineskip}

50. Name the species.
\vspace{0.5\baselineskip}

Extra Credit: During this practical, you named two subclasses within the class of ray-finned fishes. Name the other subclass. 

}%End large font size
\end{document}