%!TEX TS-program = lualatex
%!TEX encoding = UTF-8 Unicode

\documentclass{exam}
\usepackage[left=0.75in,right=0.75in,top=0.75in, bottom=0.75in]{geometry}     

\usepackage{fontspec}
\def\mainfont{Linux Libertine O}
\setmainfont[Ligatures=TeX, Contextuals={NoAlternate}, BoldFont={* Bold}, ItalicFont={* Italic}, Numbers={Proportional}]{\mainfont}
\setmonofont[Scale=MatchLowercase]{Inconsolata} 
\setsansfont[Scale=MatchLowercase]{Linux Biolinum O} 
\usepackage{microtype}

\pagenumbering{gobble}
 \setlength{\parindent}{0pt}

\footer{}{}{}

\begin{document}
\begin{questions}

{\Large
\question Name the family.
\vspace{0.5\baselineskip}

\question Name the genus.
\vspace{0.5\baselineskip}

\question Name the species.
\vspace{0.5\baselineskip}

Extra Credit: Name the order.
\vspace{2\baselineskip}

%


\question Name the family.
\vspace{0.5\baselineskip}

\question Name the genus.
\vspace{0.5\baselineskip}

\question Name the species.
\vspace{2\baselineskip}
%


\question Name the class.
\vspace{0.5\baselineskip}

\question Name the family.
\vspace{0.5\baselineskip}

\question Name the genus.
\vspace{0.5\baselineskip}

Extra Credit: This species sees very well in the dim light. Name the membrane at the back of the eye that maximizes the ability of this species to capture available light.
\vspace{2\baselineskip}
%


\question Name the genus.
\vspace{0.5\baselineskip}

\question Name the species (top; smaller specimen).
\vspace{0.5\baselineskip}

\question Name the species (bottom; larger specimen).
\vspace{2\baselineskip}
%

%
\question Name the superclass.
\vspace{0.5\baselineskip}

\question Name the family.
\vspace{0.5\baselineskip}

\question Name the genus.
\vspace{2\baselineskip}
%

\question Name the genus (left fish).
\vspace{0.5\baselineskip}

\question Name the species.
\vspace{2\baselineskip}

\question Name the subclass (right fish).
\vspace{0.5\baselineskip}

\question Name the genus.
\vspace{2\baselineskip}
%

\question Name the genus (left fish).
\vspace{0.5\baselineskip}

\question Name the species.
\vspace{2\baselineskip}
%


\question Name the genus (right fish).
\vspace{0.5\baselineskip}

\question Name the species.
\vspace{2\baselineskip}
%

\question Name the family.
\vspace{0.5\baselineskip}

\question Name the genus.
\vspace{0.5\baselineskip}

\question Name the species.
\vspace{2\baselineskip}
%


\question Name the division.
\vspace{0.5\baselineskip}

\question Name the genus.
\vspace{0.5\baselineskip}

\question Name the species.
\vspace{2\baselineskip}
%


\question Name the genus (left fish).
\vspace{0.5\baselineskip}

\question Name the species.
\vspace{2\baselineskip}
%

\question Name the family (right fish).
\vspace{0.5\baselineskip}

\question Name the genus
\vspace{0.5\baselineskip}

\question Name the species
\vspace{2\baselineskip}
%

\question Name the family (left fish).
\vspace{0.5\baselineskip}

\question Name the genus.
\vspace{0.5\baselineskip}

\question Name the species.
\vspace{2\baselineskip}


\question Name the genus (right fish).
\vspace{0.5\baselineskip}

\question Name the species.
\vspace{2\baselineskip}
%

\question Name the subclass.
\vspace{0.5\baselineskip}

\question Name the genus.
\vspace{0.5\baselineskip}

\question Name the species.
\vspace{2\baselineskip}
%

\question Name the genus.
\vspace{0.5\baselineskip}

\question Name the species.
\vspace{0.5\baselineskip}

Extra Credit: Name the form/function of this fish.
\vspace{2\baselineskip}
%


\question Name the genus (left fish).
\vspace{0.5\baselineskip}

\question Name the species.
\vspace{2\baselineskip}
%

\question Name the genus (right fish).
\vspace{0.5\baselineskip}

\question Name the species.
\vspace{2\baselineskip}
%

\question Name the genus.
\vspace{0.5\baselineskip}

\question Name the species.
\vspace{0.5\baselineskip}

Extra Credit: During this practical, you named two subclasses within the class of ray-finned fishes. Name the other subclass. 
}%End large font size
\end{questions}
\end{document}