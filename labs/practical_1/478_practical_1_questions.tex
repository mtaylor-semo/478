%!TEX TS-program = lualatex
%!TEX encoding = UTF-8 Unicode

\documentclass{exam}
\usepackage[left=0.75in,right=0.75in,top=0.75in, bottom=0.75in]{geometry}     

\usepackage{fontspec}
\def\mainfont{Linux Libertine O}
\setmainfont[Ligatures=TeX, Contextuals={NoAlternate}, BoldFont={* Bold}, ItalicFont={* Italic}, Numbers={Proportional}]{\mainfont}
%\setmonofont[Scale=MatchLowercase]{Linux Libertine Mono O} 
%\setsansfont[Scale=MatchLowercase]{Linux Biolinum O} 
\usepackage{microtype}

\pagestyle{head}

%\pagenumbering{gobble}
 \setlength{\parindent}{0pt}

\begin{document}

\begin{questions}

{\Large 
%1
\question Name the family.

%2
\question Given the form, what is the likely function of this fish.
\vspace{2\baselineskip}


%3
\question Name the family.

%4
\question Given the form, what is the likely function of this fish.
\vspace{2\baselineskip}


% 
\question Name the long, coiled structure.

%6
\question What is the likely diet of this fish?
\vspace{2\baselineskip}


%7
\question \label{heart}Name the entire structure.

%8
\question Name the pinned structure.

%9
\question Name the blood vessel that passes forward from the structure you named in equestion~\ref{heart}.
\vspace{2\baselineskip}


%10
\question Name the family.

%11
\question Name the type of fin with the pin through it.
\vspace{2\baselineskip}


%12
\question Name the family.

%13
\question Name of the position for the paired fins.
\vspace{2\baselineskip}


%14
\question Name the family of the smaller fish.

%15
\question Name the family of the larger fish.

%16
\question Name the mouth position of the larger fish.
\vspace{2\baselineskip}

\newpage

%17
\question Name the family.

%18
\question Given the form, what is the likely function of this fish?
\vspace{2\baselineskip}


%19
\question Name the family of the smaller fish.

%20
\question Name the family of the larger fish.
\vspace{2\baselineskip}

%21
\question Name the pinned, darkish structure (not the bladder). 

%22
\question What is the sex of this fish? %What is the proper term for this type of gas bladder?
\vspace{2\baselineskip}

%23
%\question Name the structure pinned with the label A.

%23
\question Name the tooth-bearing bone of the lower jaw.

%24
%\question Name the type of scale next to the pin with the label B.

%24
\question Name the type of vertebra.
\vspace{2\baselineskip}

%25
\question Name the family of the smaller fish.
%\vspace{1\baselineskip}

%26
\question Name the family of the larger fish.
\vspace{2\baselineskip}

%27
\question Name of the larger bone that covers the gills.

%28
\question Name of this specific muscle mass (dorsal side of fish) labeled B.

%29. Name the family.
\question Name the membrane that separates the left and right muscle masses.
\vspace{2\baselineskip}

%30
\question Name the mouth position of this fish.

%30. 
%\question Name the long, whisker-like structures around this fishes mouth.

%31
\question \label{adipose} Name the small fin next to the pin.

%32
\bonusquestion \textsc{(ec)} Name another native Missouri family that has the fin type named in question~\ref{adipose}.
\vspace{2\baselineskip}

\newpage

%33
\question Name the family.

%34
\question Name the region of the body that extends forward from the end of the anal fin to the start of the caudal fin.
\vspace{2\baselineskip}

%35
%\bonusquestion \textsc{(ec)} There are three pairs of this type of structure in the inner ear of fishes.  %This is the large one.  Name it. You must be exact to receive credit.

%36
\bonusquestion \textsc{(ec)} Name another family of fishes that is native to Missouri but is \textit{not} one of the families that you are required to know for this practical. % Hiodontidae does not count.
\vspace{2\baselineskip}

} %End large font

\end{questions}


%\newpage



%}%End large font size


\end{document}