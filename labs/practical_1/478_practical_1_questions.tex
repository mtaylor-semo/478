%!TEX TS-program = lualatex
%!TEX encoding = UTF-8 Unicode

\documentclass{exam}
\usepackage[left=0.75in,right=0.75in,top=0.5in, bottom=0.5in]{geometry}     

\usepackage{fontspec}
\def\mainfont{Linux Libertine O}
\setmainfont[Ligatures=TeX, Contextuals={NoAlternate}, BoldFont={* Bold}, ItalicFont={* Italic}, Numbers={Proportional}]{\mainfont}
%\setmonofont[Scale=MatchLowercase]{Linux Libertine Mono O} 
%\setsansfont[Scale=MatchLowercase]{Linux Biolinum O} 
\usepackage{microtype}

\pagestyle{head}

%\pagenumbering{gobble}
% \setlength{\parindent}{0pt}

\renewcommand{\questionshook}{%
	\setlength{\leftmargin}{0pt}
	\setlength{\labelwidth}{\parindent}}


\begin{document}

\begin{questions}

{\Large 
%1
\question Name the type of vertebra.

%2
\question Name the hole in the spines that the thread passes through.
\vspace{2\baselineskip}


%3
\question Name the bone numbered 3 (on lower jaw).

%4
\question Name the bone numbered 4 (near back side of head).
\vspace{2\baselineskip}


%5 
\question Name the pinned, darkish structure (not the bladder).

%6
\question What is the term for this type of swim bladder?
\vspace{2\baselineskip}


%7
\question \label{heart}Name the entire structure.

%8
\question Name the pinned structure.

%9
\question Name the blood vessel that passes forward from the structure you named in question~\ref{heart}.
\vspace{2\baselineskip}


%10
\question Name the \emph{long, coiled} structure (a little tricky but you can do it!).

%11
\question What is the likely diet of this fish?

%12
\question Where does this fish feed?

%13
\question Proper term for this mouth position?
\vspace{2\baselineskip}


%14
\question Name the scale type under the microscope. You may adjust focus and magnification.

%15ec
\bonusquestion \textsc{(ec)} Name the scale type on the piece of skin not under the micrope.

\vspace{2\baselineskip}

%\newpage

%16
\question Name the region between the end of the anal fin and the caudal fin.

%17
\question Name the large bone at the distal end of the \textit{upper} lip. Ask if you are unsure which bone I mean.

%18
\question Given the form, what is the likely function of this fish?

\vspace{2\baselineskip}


%19
\question Name these paired structures.

%20ec
\bonusquestion \textsc{(ec)} What is the likely diet of this fish.

\vspace{2\baselineskip}

%\newpage

%21
\question What pair of fins is this?

%22
\question What is the term that describes the relative position of the paired fins to each other on this fish?

\vspace{2\baselineskip}


%23
\question What is the name of the individual muscle blocks?

%24
\question What is the name for the membrane that separates the epaxials from the hypaxials?

\vspace{2\baselineskip}

%25
\question What type of tail is found on the left fish? 

%26
\question Given the form, what is the likely function of the right fish? %What is the proper term for this type of gas bladder?

\vspace{2\baselineskip}

%27
%\question Name the bone that supports each of these fins.

%\vspace{2\baselineskip}

%27
\question What type of tail is found on this fish?

%28
\question What is the sex of this fish?

\vspace{2\baselineskip}

%30
%\question Name the bone held together by the clay.

%\vspace{1\baselineskip}

%29
\question Name the opening through which the pin passes.

%30
\question Name the large gray structure that is pinned.

%31
\question What sex is this fish? How do you know?

\vspace{2\baselineskip}
%\newpage


%31
%\question Name of the larger bone that covers the gills.

%34
%\question Name the bone under the tip of the light pointer.

%35
%\question Name the bone under the tip of the dark pointer.

%\vspace{2\baselineskip}

%32
\question What is the name for the reddish-brown part of this structure,\newline through which the pin passes? \label{filaments}

%33 
\question What is the name for the white structure \newline to which the reddish-brown structure attaches?

%34ec
\question \textsc{(ec)}
What is the name of the artery that exits the structure in question~\ref{filaments}?


\vspace{2\baselineskip}

%38
%\question Name the bone under the tip of the light pointer.

%39
%\question Name the bone under the tip of the dark pointer.

\question Name for this type of vertebra (individual vertebra, not all collectively).

\question Name for the spine on the ventral (lower) side  of each vertebra. Ventral is facing you when tail is to your left.

\vspace{2\baselineskip}

%37
\question Name the long, whisker-like structures around the mouth of this fish?

%38
\question Name the small, fleshy fin on the dorsal surface, near the caudal fin.


\bonusquestion \textsc{(ec)} Name this bone.
%\vspace{2\baselineskip}

} %End large font

\end{questions}


%\newpage



%}%End large font size


\end{document}