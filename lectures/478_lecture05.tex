 %!TEX TS-program = lualatex
%!TEX encoding = UTF-8 Unicode

 \documentclass[t]{beamer}

%%%% HANDOUTS For online Uncomment the following four lines for handout
%\documentclass[t,handout]{beamer}  %Use this for handouts.
%\includeonlylecture{student}
%\usepackage{handoutWithNotes}
%\pgfpagesuselayout{3 on 1 with notes}[letterpaper,border shrink=5mm]
%	\setbeamercolor{background canvas}{bg=black!5}


%%% Including only some slides for students.
%%% Uncomment the following line. For the slides,
%%% use the labels shown below the command.

%% For students, use \lecture{student}{student}
%% For mine, use \lecture{instructor}{instructor}


%\usepackage{pgf,pgfpages}
%\pgfpagesuselayout{4 on 1}[letterpaper,border shrink=5mm]

% FONTS
\usepackage{fontspec}
\def\mainfont{Linux Biolinum O}
\setmainfont[Ligatures=TeX, Contextuals={NoAlternate}, BoldFont={* Bold}, ItalicFont={* Italic}, Numbers={Proportional}]{\mainfont}
%\setmonofont[Scale=MatchLowercase]{Inconsolata} 
\setsansfont[Scale=MatchLowercase]{Linux Biolinum O} 
\usepackage{microtype}

\usepackage{graphicx}
	\graphicspath{%
	{/Users/goby/Pictures/teach/478/lectures/}%
	{/Users/goby/Pictures/teach/common/}} % set of paths to search for images

\usepackage{amsmath,amssymb}

%\usepackage{units}

\usepackage{booktabs}
\usepackage{array}
\newcolumntype{L}[1]{>{\raggedright\let\newline\\\arraybackslash\hspace{0pt}}p{#1}}
\newcolumntype{C}[1]{>{\centering\let\newline\\\arraybackslash\hspace{0pt}}p{#1}}
\newcolumntype{R}[1]{>{\raggedleft\let\newline\\\arraybackslash\hspace{0pt}}p{#1}}

\usepackage{multicol}
%	\setlength{\columnsep=1em}

\usepackage{textcomp}
\usepackage{setspace}
\usepackage{tikz}
	\tikzstyle{every picture}+=[remember picture,overlay]

\mode<presentation>
{
  \usetheme{Lecture}
  \setbeamercovered{invisible}
  \setbeamertemplate{items}[default]
}

\usepackage{calc}
\usepackage{hyperref}


\begin{document}
%\lecture{instructor}{instructor}
%\lecture{student}{student}

{
\usebackgroundtemplate{\includegraphics[width=\paperwidth]{sensory_how}}
\begin{frame}

\vfill

\tinyfill\textcolor{white}{Carcharhiniformes: Carcharhinidae: \textit{Galeocerdo cuvier,} \href{https://commons.wikimedia.org/w/index.php?curid=7737291}{Albert kok, \ccbysa{3.0}.}}

\end{frame}
}

{
\usebackgroundtemplate{\includegraphics[width=\paperwidth]{sensory_mechanoreception}}
\begin{frame}[t]{Fishes use \highlight{mechanoreception} to detect vibrations via \highlight{sensory hair cells}.}


\vfilll

\tinyfill Figs.~6.1 \& 6.2 \textcopyright\,Helfman et al.~2009

\end{frame}
}


\begin{frame}[t]{Fishes can have superficial and canal \highlight{neuromasts.}}

\includegraphics[width=\linewidth]{sensory_neuromasts}

\end{frame}


\begin{frame}[t]{Placement of lateralis system reduces interference from fins, maximizes signal from environment.}

\includegraphics[width=\linewidth]{sensory_lateralis_placement}

\vfilll

\tinyfill Fig.~10.7 Moyle and Cech, Jr.~2004
\end{frame}


\begin{frame}[t]{\highlight{Otoliths} (ear stones) in the inner ear aid hearing and balance.}

\vspace{-\baselineskip}
\begin{multicols}{2}

\includegraphics[width=0.9\linewidth]{sensory_sound}

\columnbreak

The \highlight{lapillus} otolith used primary for balance. The \highlight{sagitta} and \highlight{astericus} used primarily for hearing.

\vspace{3\baselineskip}

Swimbladder amplifies sound.

\vspace{3\baselineskip}

\highlight{Weberian apparatus} composed of bony ossicles that transfer sound to inner ear of otophysan fishes.

\end{multicols}

\end{frame}

\lecture{instructor}{instructor}
{
\setbeamercolor{background canvas}{bg=black}
\begin{frame}
\centering

\includegraphics[height=0.9\textheight]{sensory_coryphaenoides_inner_ear}

\vfilll

\tiny \textcolor{white}{Brain and inner ears, Gadiformes: Macrouridae: \textit{Coryphaenoides armata} \hfill A.~Popper, Laboratory of Bioacoustics, Univ.~Maryland}
\end{frame}
}

{
\setbeamercolor{background canvas}{bg=black}
\begin{frame}
\centering

\includegraphics[height=0.9\textheight]{sensory_left_ear}

\vfilll

\tiny \textcolor{white}{Left ear, Gadiformes: Moridae: \textit{Antimora rostrata} \hfill A.~Popper, Laboratory of Bioacoustics, Univ.~Maryland}
\end{frame}
}


{
\setbeamercolor{background canvas}{bg=black}
\begin{frame}
\centering

\includegraphics[height=0.9\textheight]{sensory_right_ear}

\vfilll

\tiny \textcolor{white}{Right ear, Gadiformes: Moridae: \textit{Antimora rostrata} \hfill A.~Popper, Laboratory of Bioacoustics, Univ.~Maryland}
\end{frame}
}

%


\lecture{student}{student}

\begin{frame}[t,plain]{Fishes use \highlight{chemoreception} to find food and habitat, detect predators, and communicate with conspecifics.}

\vspace{-0.5\baselineskip}

\begin{multicols}{2}

\includegraphics[width=0.9\linewidth]{sensory_olfaction}

\reflectbox{\includegraphics[width=0.9\linewidth]{sensory_barbels}}

\columnbreak

\highlight{Olfaction:} olfactory gland in nostrils detects chemical cues in the water.

\vspace{5\baselineskip}

\highlight{Gustation:} taste receptor cells and buds can occur in the mouth, lips, barbels, skin, and fins.
\end{multicols}


\vfilll

\tiny Top: Lagler et al.~1977 \hfill Bottom: Perciformes: Mullidae: \textit{Upeneichthys lineatus,} \href{https://www.flickr.com/photos/71925535@N03/33984018120}{John Turnbull, \ccbyncsa{2.0}} 

\end{frame}

%

\begin{frame}[t]{Ostariophysan fishes produce a substance (\highlight{schreckstoff}) that triggers alarm response in other individuals.}
\centering
\includegraphics[height=0.75\textheight]{sensory_schreckstoff_response}

\vfilll

\tiny Gonorhynchiformes, Cypriniformes, Siluriformes, Gymnotiformes, Characiformes \hfill Mathuru et al.~2012, Curr Biol 22: 538
\end{frame}


%
{
\usebackgroundtemplate{\includegraphics[width=\paperwidth]{sensory_ampullary}}
\begin{frame}[t]{\highlight{Ampullary receptors} passively detect electrical signals.}

\vfill

\tinyfill \href{ https://www.flickr.com/photos/27077560@N05/49203331708}{Chimaeriformes: Chimaeridae: \textit{Hydrolagus} sp., NOAA Ocean Exploration, \ccbysa{2.0}.}

\end{frame}
}
%
{
\usebackgroundtemplate{\includegraphics[width=\paperwidth]{sensory_tuberous}}
\begin{frame}[t]{\highlight{Tuberous receptors} actively detect electrical signals.}

\vfill

\tinyfill \textcolor{white}{ \href{https://www.science.org/content/article/how-ghost-knifefish-became-fastest-electrical-discharger-animal-kingdom}{Gymnotiformes: Apteronotidae: \textit{Apteronotus} sp. \textcopyright\,Blickwinkel/Alamy stock photo.}}

\end{frame}
}
%
{
\usebackgroundtemplate{\includegraphics[width=\paperwidth]{sensory_weakly_electric}}
\begin{frame}[t]{Weakly electric fishes use electricity for communication.}

\vfill

\tinyfill Fortune and Chacron 2011, Encyc.~Fish Physiol.~1: 366.

\end{frame}
}
%
{
\usebackgroundtemplate{\includegraphics[width=\paperwidth]{sensory_strongly_electric}}
\begin{frame}[t]{Some fishes can use electricity for attack or defense.}

\vfill

\tinyfill \href{https://commons.wikimedia.org/w/index.php?curid=17940959}{Torpediniformes: Narcinidae: \textit{Narcine bancroftii,} NOAA Photo Library, Public Domain}

\end{frame}
}

%

\begin{frame}[t]{Most fishes have \highlight{chorioid glands} behind the retina to meet oxygen demands.}

\vspace{-\baselineskip}

\begin{multicols}{2}
\includegraphics[width=\linewidth]{sensory_vision}

\columnbreak

What physiological process do you think is used to build-up high O\textsubscript{2} concentrations?

\vspace{2\baselineskip}

\includegraphics[width=\linewidth]{sensory_tapetum}

\vspace{0.2\baselineskip}

\highlight{Tapetum lucidum} in some fishes enhances visual sensitivity in low light.

\end{multicols}

\vfilll

\tinyfill Collin~2018, Clin.~Exp.~Optom.~101: 624

\end{frame}

%

\begin{frame}[t]{Color can be important for visual communication.}

\vspace{-\baselineskip}

\begin{multicols}{2}

\includegraphics[width=\linewidth]{sensory_color1}

\includegraphics[width=\linewidth]{sensory_color2}

\includegraphics[width=\linewidth]{sensory_color3}
\columnbreak

Color is an evolutionary trade-off between mating and predation.

\vspace{9\baselineskip}

Environment can affect visual mate choice.

\end{multicols}

\end{frame}

\begin{frame}[t]{Bright yellows and blues can serve as disruptive coloration for some coral reef fishes.}

\vspace{-\baselineskip}

\begin{multicols}{2}

\includegraphics[width=\linewidth]{sensory_camoflage1}

\columnbreak

\includegraphics[width=\linewidth]{sensory_camoflage2}

\end{multicols}

\end{frame}

%

{
\usebackgroundtemplate{\includegraphics[width=\paperwidth]{misc_intro}}
\begin{frame}

\vfill

\tinyfill\textcolor{white}{Holocentriformes: Holocentridae: \textit{Sargocentron xantherythrum,} \href{https://en.wikipedia.org/wiki/File:Red_Fish_at_Papah\%C4\%81naumoku\%C4\%81kea_(cropped).jpg}{James Watt/NOAA, Public Domain.}}

\end{frame}
}

\begin{frame}[t]{Migration is long-term movements that affect ecology.}

\vspace{-\baselineskip}

\begin{multicols}{2}

\includegraphics[width=\linewidth]{migration_tuna}

Feeding

\columnbreak

\includegraphics[width=\linewidth]{migration_salmon}

Spawning


%\vspace{8\baselineskip}


\end{multicols}

\vfilll

\tiny Top: \href{https://www.fisheries.noaa.gov/feature-story/how-we-safeguard-atlantic-tunas}{Jeff Muir/NOAA, Public Domain} \hfill Bottom: \href{https://www.fws.gov/media/coho-and-sockeye-migration-russian-river}{Ryan Hagerty/USFWS, Public Domain}
\end{frame}

%

\begin{frame}[t]{\highlight{Diadromy} is the movement of fishes between fresh and salt water.}

\vspace{-\baselineskip}

\begin{multicols}{2}

\includegraphics[width=\linewidth]{diadromy_eel}

\reflectbox{\includegraphics[width=\linewidth]{diadromy_salmon}}

\columnbreak

\highlight{Catadromy:} fresh- to saltwater to spawn.

\highlight{Anadromy:} salt- to freshwater to spawn.

\highlight{Amphidromy:} either way, \emph{not} to spawn.

\vspace{\baselineskip}

\includegraphics[width=\linewidth]{diadromy_lentipes}\\
\tiny \textit{Lentipes concolor} \href{http://hbs.bishopmuseum.org/waipio/Critter\%20pages/lentipes.html}{Bishop Museum, Hawaii.}  \href{https://www.youtube.com/watch?v=mmwM7dwtvKA}{Link to video} 

\end{multicols}


\vfilll

\tiny Top left: European Eel, \href{https://commons.wikimedia.org/wiki/File:Anguilla-anguilla_1.jpg}{Dmitriy Konstantinov \ccbysa{3.0}} \hfill Bottom left:  \href{https://www.fisheries.noaa.gov/species/sockeye-salmon}{\textit{Oncorhynchus nerka}, NOAA Fisheries, Public Domain}

\end{frame}

%

\begin{frame}[t]{Fishes aggregate in unorganized \highlight{shoals} or organized, polarized \highlight{schools.}}

\vspace{-\baselineskip}

\begin{multicols}{2}

\includegraphics[width=\linewidth]{shoaling_shoal2}\\
Shoal

\columnbreak

\includegraphics[width=0.96\linewidth]{shoaling_baitball}\\
School


%\reflectbox{\includegraphics[width=0.96\linewidth]{shoaling_school_jacks}}

\end{multicols}

\vspace{\baselineskip}

\begin{tabular}{L{0.465\textwidth}L{0.4\textwidth}}
Hydrodynamic efficiency & Predator avoidance \tabularnewline

Foraging efficiency & Reproductive efficiency \tabularnewline

\end{tabular}

\vfilll

C4%81kea_(cropped).jpg
\tiny Right:  \href{https://www.flickr.com/photos/42988059@N02/6868820995/in/album-72157629279028961/}{\textcopyright\,Critidoc, Flickr} \hfill Left: \href{https://en.wikipedia.org/wiki/File:Red_Fish_at_Papah\%C4\%81naumoku\%}{James Watt } %\href{https://oceanexplorer.noaa.gov/explorations/17cuba-reefs/logs/may30/may30.html}{NOAA Ocean Explorer} 
\end{frame}


\end{document}
