%!TEX TS-program = lualatex
%!TEX encoding = UTF-8 Unicode

%\documentclass[t]{beamer}

%%%% HANDOUTS For online Uncomment the following four lines for handout
\documentclass[t,handout]{beamer}  %Use this for handouts.
\usepackage{handoutWithNotes}
\pgfpagesuselayout{3 on 1 with notes}[letterpaper,border shrink=5mm]
	\setbeamercolor{background canvas}{bg=black!5}


%%% Including only some slides for students.
%%% Uncomment the following line. For the slides,
%%% use the labels shown below the command.
%\includeonlylecture{student}

%% For students, use \lecture{student}{student}
%% For mine, use \lecture{instructor}{instructor}


%\usepackage{pgf,pgfpages}
%\pgfpagesuselayout{4 on 1}[letterpaper,border shrink=5mm]

% FONTS
\usepackage{fontspec}
\def\mainfont{Linux Biolinum O}
\setmainfont[Ligatures=TeX, Contextuals={NoAlternate}, BoldFont={* Bold}, ItalicFont={* Italic}, Numbers={Proportional}]{\mainfont}
\setmonofont[Scale=MatchLowercase]{Inconsolata} 
\setsansfont[Scale=MatchLowercase]{Linux Biolinum O} 
\usepackage{microtype}

\usepackage{graphicx}
	\graphicspath{%
	{/Users/goby/Pictures/teach/478/lectures/}%
	{/Users/goby/Pictures/teach/common/}} % set of paths to search for images

\usepackage{amsmath,amssymb}

%\usepackage{units}

\usepackage{booktabs}
\usepackage{multicol}
%	\setlength{\columnsep=1em}

\usepackage{textcomp}
\usepackage{setspace}
\usepackage{tikz}
	\tikzstyle{every picture}+=[remember picture,overlay]

\mode<presentation>
{
  \usetheme{Lecture}
  \setbeamercovered{invisible}
  \setbeamertemplate{items}[square]
}

\usepackage{calc}
\usepackage{hyperref}


\begin{document}
%\lecture{instructor}{instructor}
%\lecture{student}{student}

{
\usebackgroundtemplate{\includegraphics[width=\paperwidth]{buoyancy_intro}}
\begin{frame}[t,plain]

\vspace{3\baselineskip}
\hangpara\LARGE\hspace{9em}\textcolor{white}{How do fishes}\\%
	\LARGE\hspace{9em}\textcolor{white}{control buoyancy?}
\vskip0pt plus 1filll
\tiny\textcolor{white!80!black}{School of fishes, Red Sea. Michael Aston, Flickr Creative Commons.}
\end{frame}
}


{
\usebackgroundtemplate{\includegraphics[width=\paperwidth]{buoyancy_squalene}}
\begin{frame}[b,plain]

\hfill\tiny\textcolor{white!80!black}{\textit{Prionace glauca} (Blue Shark, Carcharhinidae: Carcharhiniformes). National Marine Fisheries Service, Wikimedia Commons.}

\end{frame}
}


{
\usebackgroundtemplate{\includegraphics[width=\paperwidth]{buoyancy_head_fin_lift}}
\begin{frame}[b,plain]

\tiny\textcolor{white!80!black}{\textit{Aliopus vulpinus} (Thresher Shark, Alopiidae: Lamniformes). Raven\_Denmark, Flickr Creative Commons.}

\end{frame}
}

{
\usebackgroundtemplate{\includegraphics[width=\paperwidth]{buoyancy_reduce_tissue_density}}
\begin{frame}[b,plain]{Bathypelagic fishes (and deeper) reduce the density and amount of heavy tissues.}

\hfill\tiny\textcolor{white!20!black}{\textit{Psychrolutes phrictus} (Blob Sculpin, Psychrolutidae: Scorpaeniformes). NOAA, Wikimedia Commons.}

\end{frame}
}


\begin{frame}[t]{A \highlight{physostomous} gas bladder connects to the gut for air exchange.}

\vspace{\baselineskip}
\centering
\includegraphics[width=\textwidth]{buoyancy_physostomous_bladder}

\begin{tikzpicture}
	\draw [ultra thick] (-2.5,2.5) circle (0.4cm);
	\draw [thick] (-4,0) -- (-2.7,2.2);
	\node [below] at (-4,0) {\highlight{pneumatic duct}};
	
	\draw [->,thick] (-0.7,0) -- (-1.2,1);
	\node [below] at (-0.7,0) {gut};
	
	\draw [->,thick] (2,0) -- (1.5,2.2);
	\node [below] at (2,0) {gas bladder};
	
	\pause
	\node [right,text width=4cm] at (-5.5,5.25){How is the bladder filled and emptied?};
\end{tikzpicture}

\end{frame}

\begin{frame}[t]{A \highlight{physoclistous} gas bladder is sealed and separate from the gut.}

\vspace{\baselineskip}
\centering
\includegraphics[width=\textwidth]{buoyancy_physoclistous_bladder}

\begin{tikzpicture}
	\node [right] at (-5.5,-0.5){How is the bladder filled and emptied?};
\end{tikzpicture}

\end{frame}

\begin{frame}[c]{The \highlight{rete mirabile} and the \highlight{gas gland} are  used to fill both types of gas bladder.}

\centering\includegraphics{buoyancy_rete_gas_gland_structure}

\end{frame}
%%
\begin{frame}[t]{\highlight{Root effect} and \highlight{countercurrent exchange} processes fill the gas bladder.}

\vspace{\baselineskip}
\centering
\includegraphics[width=0.8\textwidth]{buoyancy_rete_gas_gland_blank}

\begin{tikzpicture}[%
	textNode/.style={font=\footnotesize},%
	scriptNode/.style={font=\scriptsize}]

	\node [textNode, rotate=-90] at (5.1,4.4){gas bladder lumen};
	\node [textNode, right, text width=1.4cm] at (4.6,1.5){cell of gas gland};
	\draw [->,thick, above left] (4.6,1.5) -- (4.1,2.3);

	% First pause for blood moving through vessel with HbO2/Hb
	%\pause
	\node [textNode, left] at (-4.9, 5.2){HbO$_2$};
	\draw [->,thick] (-4.9, 5) -- (-4.3, 4.4);
	\node [textNode, left] at (-4.9, 2.9){Hb};
	\draw [->,thick] (-4.3, 3.4) -- (-4.9, 2.9);
	
	%Second pause for glycolysis
	\pause
	\node [scriptNode, left] at (3.1, 5.1) {glucose};
	\node [scriptNode, left] at (3.1, 4.5) {lactic acid};
	\node [scriptNode, rotate=-90,above] at (3.9,4.75){glycolysis};
	\draw [->, thick] (3.1, 5.1) -- (3.9, 5.1);
	\draw [->, thick] (3.9, 4.5) -- (3.1, 4.5);
	
	% Third pause for root effect
	\pause
	\node [scriptNode, right, text width=1cm] at (2.5, 3.25) {\highlight{Root effect}};	
	\draw [->] (3.0, 4.4) .. controls (3.1,3.95) .. (3.0,3.5);

	% Fourth pause to results of root effect
	\pause
	\node [textNode] at (2.5,3.75){HbO$_2$};
	\node [textNode] at (2.5, 2.65){Hb + O$_2$};
	\draw [->,thick] (2.5,3.52) -- (2.5,2.88);
	
	\draw [thick] (2.85, 2.65) circle (0.25cm); 
	\draw [->,ultra thick] (3.1, 2.65) -- (5.3, 2.65);
	\node [textNode, right] at (5.3, 2.65){O$_2$};
	
	% Fifth pause to begin CCE
	\pause
	\node [scriptNode, text width=2cm] at (-1.5, 3.85){\highlight{countercurrent exchange}};
	\node [scriptNode] at (-3, 3.5){O$_2$};
	\node [scriptNode] at (0, 3.5){lactic acid};
	\draw [->, thick] (-3, 3.7) -- (-3, 4.2); 
	\draw [->, thick] (0, 3.7) -- (0, 4.2); 
	
	% Last pause for concentrations
	\pause
	\node [textNode, text width=1.5cm] at (-3, 5.1) {pH = 7.8\\ pO$_2$ = 0.05};
	\node [textNode, text width=1.5cm] at (0, 5.1) {pH = 7.3\\ pO$_2$ = 0.37};
	\node [textNode, text width=1.5cm] at (-3, 2.7) {pH = 7.6\\ pO$_2$ = 0.06};
	\node [textNode, text width=1.5cm] at (0, 2.7) {pH = 7.1\\ pO$_2$ = 0.39};
	
	
\end{tikzpicture}

\end{frame}

\begin{frame}[c]{The \highlight{oval} has a muscular sphincter to control release of gas from the bladder.}
\centering
\includegraphics[width=0.75\textwidth]{buoyancy_oval}

\end{frame}

\end{document}
