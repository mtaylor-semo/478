 %!TEX TS-program = lualatex
%!TEX encoding = UTF-8 Unicode

\label{key}\documentclass[t]{beamer}

%%%% HANDOUTS For online Uncomment the following four lines for handout
%\documentclass[t,handout]{beamer}  %Use this for handouts.
%\includeonlylecture{student}
%\usepackage{handoutWithNotes}
%\pgfpagesuselayout{3 on 1 with notes}[letterpaper,border shrink=5mm]
%	\setbeamercolor{background canvas}{bg=black!5}


%%% Including only some slides for students.
%%% Uncomment the following line. For the slides,
%%% use the labels shown below the command.

%% For students, use \lecture{student}{student}
%% For mine, use \lecture{instructor}{instructor}


%\usepackage{pgf,pgfpages}
%\pgfpagesuselayout{4 on 1}[letterpaper,border shrink=5mm]

% FONTS
\usepackage{fontspec}
\def\mainfont{Linux Biolinum O}
\setmainfont[Ligatures=TeX, Contextuals={NoAlternate}, BoldFont={* Bold}, ItalicFont={* Italic}, Numbers={Proportional}]{\mainfont}
%\setmonofont[Scale=MatchLowercase]{Inconsolata} 
\setsansfont[Scale=MatchLowercase]{Linux Biolinum O} 
\usepackage{microtype}

\usepackage{graphicx}
	\graphicspath{%
	{/Users/goby/Pictures/teach/478/lectures/}%
	{/Users/goby/Pictures/teach/common/}} % set of paths to search for images

\usepackage{amsmath,amssymb}

%\usepackage{units}

\usepackage{booktabs}
\usepackage{multicol}
%	\setlength{\columnsep=1em}

\usepackage{textcomp}
\usepackage{setspace}
\usepackage{tikz}
	\tikzstyle{every picture}+=[remember picture,overlay]

\mode<presentation>
{
  \usetheme{Lecture}
  \setbeamercovered{invisible}
  \setbeamertemplate{items}[default]
}

\usepackage{calc}
\usepackage{hyperref}


\begin{document}
%\lecture{instructor}{instructor}
%\lecture{student}{student}

{
\usebackgroundtemplate{\includegraphics[width=\paperwidth]{sensory_how}}
\begin{frame}

\vfill

\tinyfill\textcolor{white}{Carcharhiniformes: Carcharhinidae: \textit{Galeocerdo cuvier,} \href{https://commons.wikimedia.org/w/index.php?curid=7737291}{Albert kok, \ccbysa{3.0}.}}

\end{frame}
}

{
\usebackgroundtemplate{\includegraphics[width=\paperwidth]{sensory_mechanoreception}}
\begin{frame}[t]{Fishes use \highlight{mechanoreception} to detect vibrations via \highlight{sensory hair cells}.}


\vfilll

\tinyfill Figs.~6.1 \& 6.2 \textcopyright\,Helfman et al.~2009

\end{frame}
}


\begin{frame}[t]{Fishes can have superficial and canal \highlight{neuromasts.}}

\includegraphics[width=\linewidth]{sensory_neuromasts}

\end{frame}


\begin{frame}[t]{Placement of lateralis system reduces interference from fins, maximizes signal from environment.}

\includegraphics[width=\linewidth]{sensory_lateralis_placement}

\vfilll

\tinyfill Fig.~10.7 Moyle and Cech, Jr.~2004
\end{frame}


\begin{frame}[t]{\highlight{Otoliths} (ear stones) in the inner ear aid hearing and balance.}

\vspace{-\baselineskip}
\begin{multicols}{2}

\includegraphics[width=0.9\linewidth]{sensory_sound}

\columnbreak

The \highlight{lapillus} otolith used primary for balance. The \highlight{sagitta} and \highlight{astericus} used primarily for hearing.

\vspace{3\baselineskip}

Swimbladder amplifies sound.

\vspace{3\baselineskip}

\highlight{Weberian apparatus} composed of bony ossicles that transfer sound to inner ear of otophysan fishes.

\end{multicols}

\end{frame}

\lecture{instructor}{instructor}
{
\setbeamercolor{background canvas}{bg=black}
\begin{frame}
\centering

\includegraphics[height=0.9\textheight]{sensory_coryphaenoides_inner_ear}

\vfilll

\tiny \textcolor{white}{Brain and inner ears, Gadiformes: Macrouridae: \textit{Coryphaenoides armata} \hfill A.~Popper, Laboratory of Bioacoustics, Univ.~Maryland}
\end{frame}
}

{
\setbeamercolor{background canvas}{bg=black}
\begin{frame}
\centering

\includegraphics[height=0.9\textheight]{sensory_left_ear}

\vfilll

\tiny \textcolor{white}{Left ear, Gadiformes: Moridae: \textit{Antimora rostrata} \hfill A.~Popper, Laboratory of Bioacoustics, Univ.~Maryland}
\end{frame}
}


{
\setbeamercolor{background canvas}{bg=black}
\begin{frame}
\centering

\includegraphics[height=0.9\textheight]{sensory_right_ear}

\vfilll

\tiny \textcolor{white}{Right ear, Gadiformes: Moridae: \textit{Antimora rostrata} \hfill A.~Popper, Laboratory of Bioacoustics, Univ.~Maryland}
\end{frame}
}




\lecture{student}{student}

\begin{frame}[t,plain]{Fishes use \highlight{chemoreception} to find food and habitat, detect predators, and communicate with conspecifics.}

\vspace{-0.5\baselineskip}

\begin{multicols}{2}

\includegraphics[width=0.9\linewidth]{sensory_olfaction}

\reflectbox{\includegraphics[width=0.9\linewidth]{sensory_barbels}}

\columnbreak

\highlight{Olfaction:} olfactory gland in nostrils detects chemical cues in the water.

\vspace{5\baselineskip}

\highlight{Gustation:} taste receptor cells and buds can occur in the mouth, lips, barbels, skin, and fins.
\end{multicols}


\vfilll

\tiny Top: Lagler et al.~1977 \hfill Bottom: Perciformes: Mullidae: \textit{Upeneichthys lineatus,} \href{https://www.flickr.com/photos/71925535@N03/33984018120}{John Turnbull, \ccbyncsa{2.0}} 


\end{frame}


\end{document}
