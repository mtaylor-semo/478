
%!TEX TS-program = lualatex
%!TEX encoding = UTF-8 Unicode

%\documentclass[t]{beamer}

%%%% HANDOUTS For online Uncomment the following four lines for handout
\documentclass[t,handout]{beamer}  %Use this for handouts.
\includeonlylecture{student}
\usepackage{handoutWithNotes}
\pgfpagesuselayout{3 on 1 with notes}[letterpaper,border shrink=5mm]


%%% Including only some slides for students.
%%% Uncomment the following line. For the slides,
%%% use the labels shown below the command.

%% For students, use \lecture{student}{student}
%% For mine, use \lecture{instructor}{instructor}


% FONTS
\usepackage{fontspec}
\def\mainfont{Linux Biolinum O}
\setmainfont[Ligatures=TeX, Contextuals={NoAlternate}, BoldFont={* Bold}, ItalicFont={* Italic}, Numbers={Proportional}]{\mainfont}
\setmonofont[Scale=MatchLowercase]{Linux Libertine Mono O} 
\setsansfont[Scale=MatchLowercase]{Linux Biolinum O} 
\usepackage{microtype}

\usepackage{graphicx}
	\graphicspath{%
	{/Users/goby/Pictures/teach/478/lectures/}%
	{/Users/goby/Pictures/teach/common/}} % set of paths to search for images

\usepackage{amsmath,amssymb}

%\usepackage{units}

\usepackage{booktabs}
\usepackage{multicol}
%	\setlength{\columnsep=1em}

\usepackage{textcomp}
\usepackage{setspace}
\usepackage{tikz}
	\tikzstyle{every picture}+=[remember picture,overlay]

\mode<presentation>
{
  \usetheme{Lecture}
  \setbeamercovered{invisible}
  \setbeamertemplate{items}[square]
}

\usepackage{calc}
\usepackage{hyperref}


\begin{document}
%\lecture{instructor}{instructor}
%\lecture{student}{student}

{
\usebackgroundtemplate{\includegraphics[width=\paperwidth]{locomotion_how}}
\begin{frame}[b,plain]
\tiny\textcolor{white}{\textit{Lutjanus griseus} (gray snapper), Mike Malz, Flickr Creative Commons.}

\end{frame}
}

\begin{frame}[c]{Fishes swim by contraction and relaxation of \highlight{myomeres.}}
	\centering\includegraphics[width=0.9\textwidth]{locomotion_salmon_muscle}
\end{frame}

\begin{frame}[c]{Myomeres are arranged in a series along the body.}
	\centering\includegraphics[width=\textwidth]{locomotion_myomere_arrangement}
\end{frame}

{
\usebackgroundtemplate{\includegraphics[width=\paperwidth]{locomotion_myomere_contraction}}
\begin{frame}[c]{\highlight{Contraction} of the myomere pulls the head and caudal fin toward each other.}
	%\pause
	\begin{tikzpicture}
		\draw[->, ultra thick] (3.7, 1.4) -- (5.0, 1.4);
		\draw[->, ultra thick] (8.2, 2.4) -- (6.9, 2.4); 
		\draw[->, ultra thick] (8.2, 0.4) -- (6.9, 0.4); 
	
		\draw[->, line width=3, white] (3.0,-2.4) -- (5.0, -2.4); 
		\draw[->, line width=3, white] (9.0,-2.4) -- (7.0, -2.4); 
	
	\end{tikzpicture}	
	
\end{frame}
}

{
\usebackgroundtemplate{\includegraphics[width=\paperwidth]{locomotion_myomere_contraction}}
\begin{frame}[c]{\highlight{Relaxation} of the myomere allows the head and caudal fin to be pulled away from each other.}
	%\pause
	\begin{tikzpicture}
		\draw[->, ultra thick] (5.0, 1.4) -- (3.7, 1.4);
		\draw[->, ultra thick] (6.9, 2.4) -- (8.2, 2.4); 
		\draw[->, ultra thick] (6.9, 0.4) -- (8.2, 0.4); 
	
		\draw[->, line width=3, white] (5.0,-2.4) -- (3.0, -2.4); 
		\draw[->, line width=3, white] (7.0,-2.4) -- (9.0, -2.4); 
	
	\end{tikzpicture}	
	
\end{frame}
}




{
\usebackgroundtemplate{\includegraphics[width=\paperwidth]{locomotion_overview}}
\begin{frame}[b,plain]{Fishes swim by \highlight{undulation} or \highlight{oscillation.}}

\hspace{3em}%
	\parbox{\widthof{Anguilliform}}{\centering\highlight{undulation}\\ %
		{\small \href{http://www.youtube.com/watch?v=LDrvbr_CbhE}{(video)}}}\hfill%
		%
	\parbox{\widthof{Carangiform}}{\centering\highlight{undulation}\\ %
		{\small \href{http://www.youtube.com/watch?v=OYrdFlr2vU8}{(video)}}}\hfill%
		%
	\parbox{\widthof{Ostraciform}}{\centering\highlight{oscillation}\\ %
		{\small \href{http://www.youtube.com/watch?v=tsJzGycQMcE}{(video)}}}%
\hspace{3em}

\vspace{1\baselineskip}

\end{frame}
}

{
\usebackgroundtemplate{\includegraphics[width=\paperwidth]{locomotion_bcf}}
\begin{frame}[b,plain]{Undulation of body and caudal fin (\highlight{BCF}) has four forms.}

\end{frame}
}

{
\usebackgroundtemplate{\includegraphics[width=\paperwidth]{locomotion_bcf_top_view}}
\begin{frame}[b,plain]

\end{frame}
}

{
\usebackgroundtemplate{\includegraphics[width=\paperwidth]{locomotion_mpf}}
\begin{frame}[b,plain]{Some fishes swim by undulation or oscillation of median or paired fins (\highlight{MPF}).}

\end{frame}
}

{
\usebackgroundtemplate{\includegraphics[width=\paperwidth]{locomotion_review}}
\begin{frame}[b,plain]

\end{frame}
}



{
\usebackgroundtemplate{\includegraphics[width=\paperwidth]{respiration_how}}
\begin{frame}[b,plain]
\hfill\tiny\textcolor{white!40!orange}{\textit{Cyprinus carpio} (koi, a glorified carp), Ksionic, Flickr Creative Commons.}
\end{frame}
}


\begin{frame}[t,plain]{Identify two factors that determine [O$_2$] in water.}
	\centering
	\includegraphics[width=\textwidth]{oxygen_content}
\end{frame}

{
\usebackgroundtemplate{\includegraphics[width=\paperwidth]{gill_structure}}
\begin{frame}[t,plain]
\hangpara\hspace{17em} \highlight{Gills} are the primary\\ 
	\hspace{17em} respiratory organ of fishes.

\vskip0pt plus 1filll
\hfill\tiny\textcolor{gray}{Tuna gills (right), Wikimedia Commons.}
\end{frame}
}

{
\usebackgroundtemplate{\includegraphics[width=\paperwidth]{gill_function_one}}
\begin{frame}[t,plain]{Gill structure maximizes surface area exposed to water.}

\end{frame}
}

{
\usebackgroundtemplate{\includegraphics[width=\paperwidth]{gill_function_two}}
\begin{frame}[t,plain]{\highlight{Countercurrent exchange} maximizes O$_2$ uptake.}

\end{frame}
}

{
\usebackgroundtemplate{\includegraphics[width=\paperwidth]{gill_surface_area_activity_one}}
\begin{frame}[t,plain]{Gill surface area increases with activity.}

\end{frame}
}



\begin{frame}[t,plain]
\begin{minipage}{0.55\textwidth}
	\includegraphics[width=\textwidth]{gill_surface_area_activity_two}
\end{minipage}\hfill
\begin{minipage}{0.40\textwidth}
	\raggedright
	{\parbox{\widthof{active rover-predators}}{\footnotesize\highlight{active} rover-predators\\\highlight{large} surface area}}

	\vspace{5em}

	{\hangpara\hspace{17em} Gill surface area corresponds to form and function.}

	\vspace{7em}

	{\parbox{\widthof{sluggish rover-predators}}{\footnotesize\highlight{sluggish} bottom dwellers\\\highlight{small} surface area}}

\end{minipage}
\end{frame}

{
\usebackgroundtemplate{\includegraphics[width=\paperwidth]{gill_ventilation}}
\begin{frame}[t,plain]{\highlight{Two-phase ventilation} maintains constant water flow over gills.}

\hangpara\hspace{1.5em}\parbox{2cm}{Phase 1:\\opercular\\ suction\\ pump}
\pause\hspace{8.2em}\parbox{2cm}{Phase 2:\\buccal\\ pressure\\ pump}

\end{frame}
}

{
\usebackgroundtemplate{\includegraphics[width=\paperwidth]{gill_ram_ventilation_tuna}}
\begin{frame}[b,plain]{\textcolor{white}{High speed swimmers can use} \textcolor{orange7}{ram-gill ventilation.}}
\tiny\textcolor{white!20!black}{Swimming tunas by TheAnimalDay.org, Flickr Creative Commons}
\end{frame}
}

\begin{frame}[c,plain]{Identify three ways that fishes can adjust to hypoxic conditions.}

	\hangpara 1. \rule{6cm}{0.4pt}

	\vspace{2\baselineskip}

	\hangpara 2. \rule{6cm}{0.4pt}

	\vspace{2\baselineskip}

	\hangpara 3. \rule{6cm}{0.4pt}

	\vspace{2\baselineskip}

\end{frame}

{
\usebackgroundtemplate{\includegraphics[width=\paperwidth]{respiration_regulator}}
\begin{frame}[b,plain]{\textcolor{orange7}{Oxygen regulators} increase the volume of water over gills.}
\hfill\tiny\textcolor{white!70!black}{\textit{Aphyosemion elberti} (West African killifish), Montykillies, Wikimedia Commons.}
\end{frame}
}

{
\usebackgroundtemplate{\includegraphics[width=\paperwidth]{respiration_conformer}}
\begin{frame}[b,plain]{\highlight{Oxygen conformers} decrease oxygen use.}
\hfill\tiny\textcolor{white}{\textit{Thalassothia cirrhosis} (Red Sea Toadfish), Silke Baron, Flickr Creative Commons.}
\end{frame}
}

{
\usebackgroundtemplate{\includegraphics[width=\paperwidth]{respiration_vascular_skin}}
\begin{frame}[b,plain]

\vspace{5\baselineskip}

\hangpara\Large\hspace{64mm}\highlight{Air breathers} extract O$_2$\\
\hspace{64mm}from the air.

\vskip0pt plus 1filll
\hfill\tiny\textit{Periophthalmus gracilis} (slender mudskipper),\\
\hfill PacificKlaus, Flickr Creative Commons.
\end{frame}
}

{
\usebackgroundtemplate{\includegraphics[width=\paperwidth]{respiration_vascular_mouth}}
\begin{frame}[b,plain]
\hfill\tiny\textcolor{white}{\textit{Electrophorus electricus} (electric eel), Sibylie Stofer, Flickr Creative Commons.}
\end{frame}
}

{
\usebackgroundtemplate{\includegraphics[width=\paperwidth]{respiration_vascular_gut}}
\begin{frame}[b,plain]
\tiny\textcolor{white}{\textit{Hypancistrus zebra} (loricariid catfish), Birger A, Wikimedia Commons.}
\end{frame}
}

{
\usebackgroundtemplate{\includegraphics[width=\paperwidth]{respiration_vascular_bladder}}
\begin{frame}[b,plain]
\tiny\textcolor{white}{\textit{Lepisosteus oculatus} (spotted gar), Brian Gratwicke, Wikimedia Commons.}
\end{frame}
}


{
\usebackgroundtemplate{\includegraphics[width=\paperwidth]{respiration_suprabranchial_organ}}
\begin{frame}[b,plain]
\hfill\tiny\textcolor{white}{\textit{Clarias} catfish (and carp), Takeaway, Wikimedia Commons.}
\end{frame}
}

{
\usebackgroundtemplate{\includegraphics[width=\paperwidth]{respiration_suprabranchial_organ_example}}
\begin{frame}[b,plain]{Suprabranchial organ of a clariid catfish.}
\tiny\textcolor{white}{\textit{Clarias gariepinus} (African Sharptooth Catifhsh, Clariidae: Siluriformes) \copyright Dr. Dominique Adriaens, Universiteit Ghent.}
\end{frame}
}

{
\usebackgroundtemplate{\includegraphics[width=\paperwidth]{respiration_labyrinth_organ}}
\begin{frame}[b,plain]
\hfill\tiny\textit{Trichopodus pectoralis} (snakeskin gourami), Bochr, Wikimedia Commons.
\end{frame}
}

{
\usebackgroundtemplate{\includegraphics[width=\paperwidth]{respiration_lungs}}
\begin{frame}[b,plain]
\tiny\textcolor{white}{\textit{Protopterus aethiopicus} (marbled lungfish), Joel Abroad, Flickr Creative Commons.}
\end{frame}
}


{
	\usebackgroundtemplate{\includegraphics[width=\paperwidth]{circulation_intro}}
	\begin{frame}[b,plain]{The circulatory system in fishes.}
	
	\hfill\parbox{3.2cm}{\raggedright\tiny\textcolor{white}{\textit{Chionodraco hamatus} (an icefish, Channichthyidae: Perciformes), Marrabbio2, Wikimedia Commons.}}
	
\end{frame}
}


\begin{frame}[c,plain]{Fishes have a \highlight{single pump, single circuit} circulatory system.}
\centering
\includegraphics[width=\textwidth]{circulation_system}

\end{frame}

\begin{frame}[c,plain]{Fishes have a \highlight{four-chambered} heart.}

\includegraphics[width=\textwidth]{circulation_heart_structure}

\end{frame}

\begin{frame}[c,plain]{Blood flows from ventral to dorsal through the gills.}

\includegraphics[width=\textwidth]{circulation_gills_teleost}

\phantom{\includegraphics[width=\textwidth]{circulation_gills_lungfish}}

\end{frame}

\begin{frame}[c,plain]{Lungfishes divert some blood from gills through the lungs.}

\includegraphics[width=\textwidth]{circulation_gills_teleost}

\includegraphics[width=\textwidth]{circulation_gills_lungfish}

\end{frame}


\begin{frame}[c,plain]{Hemoglobin is monomeric in hagfishes and lampreys.}

\vspace{\baselineskip}

\centering
\includegraphics[width=0.7\textwidth]{circulation_hemoglobin_monomer}

\vskip0pt plus 1filll
\hfill\tiny Modified from Hemogloblin diagram by OpenStax College, Wikimedia Commons.
\end{frame}


\begin{frame}[c,plain]{Hemoglobin is tetrameric in all other fishes.}

\vspace{\baselineskip}

\centering
\includegraphics[width=0.7\textwidth]{circulation_hemoglobin_structure}

Hb + O$_2 \Longleftrightarrow $ HbO$_2$

\vskip0pt plus 1filll
\hfill\tiny Hemogloblin diagram by OpenStax College, Wikimedia Commons.
\end{frame}

\begin{frame}[c,plain]{\highlight{Oxygen affinity} describes how easily Hb binds to and releases O$_2$.}

\centering
\includegraphics{circulation_hb_affinity}

\pause

\begin{tikzpicture}

\draw [dashed,thick] (-4.8,4) -- (3,4);
\node [right] at (3,4) {P50};

\end{tikzpicture}

\end{frame}



{
\usebackgroundtemplate{\includegraphics[width=\paperwidth]{circulation_affects_affinity}}
\begin{frame}[t,plain]

\vspace{2\baselineskip}

\hangpara What factors affect\\  oxygen affinity?

\vspace{4em}

\hangpara Do you know\\ the bicarbonate\\ buffering system?

\vskip0pt plus 1filll
\hfill\tiny\textit{Gorgasia preclara} (Splendid Garden Eel, Congridae: Anguilliformes),\\
\hfill\tiny  Opencage.info, Creative Commons.
\end{frame}
}


\begin{frame}[c,plain]{\highlight{Bohr effect} describes how pH affects oxygen \highlight{affinity.}}

\centering
\includegraphics[width=\textwidth]{circulation_bohr_root}
\pause

\begin{tikzpicture}
%% Bohr shift
\draw [->, thick] (-3.2,5.2) -- (-0.4,5.2);
\draw [->, thick] (0.1,5.2) -- (2.3,5.2);
\node [above] at (-1.8,5.2) {Bohr Effect};

\end{tikzpicture}

\end{frame}


\begin{frame}[c,plain]{\highlight{Root effect} describes how pH affects oxygen \highlight{capacity.}}

\centering
\includegraphics[width=\textwidth]{circulation_bohr_root}
\pause

\begin{tikzpicture}
% Root shift	
\draw [->, thick] (4.4,8.1) -- (4.4,6.4);
\draw [->, thick] (4.4,6.2) -- (4.4,5.38);
\node [left, text width=1cm] at (4.4,7.25) {Root Effect};

\end{tikzpicture}

\end{frame}


\begin{frame}[c,plain]{Compare and contrast O$_2$ loading and unloading in the gills and the body.}

\vspace{\baselineskip}
\includegraphics[width=\textwidth]{circulation_loading_unloading}
\vskip0pt plus 1filll

\hfill\tiny\textit{Enoplosus armatus} (Old Wife, Enoplosidae:Perciformes), Richard Ling, Wikimedia Commons.
\end{frame}

\end{document}
