%!TEX TS-program = lualatex
%!TEX encoding = UTF-8 Unicode

\documentclass[t]{beamer}

%%%% HANDOUTS For online Uncomment the following four lines for handout
%\documentclass[t,handout]{beamer}  %Use this for handouts.
%\usepackage{handoutWithNotes}
%\pgfpagesuselayout{3 on 1 with notes}[letterpaper,border shrink=5mm]
%	\setbeamercolor{background canvas}{bg=black!5}


%%% Including only some slides for students.
%%% Uncomment the following line. For the slides,
%%% use the labels shown below the command.
%\includeonlylecture{student}

%% For students, use \lecture{student}{student}
%% For mine, use \lecture{instructor}{instructor}


%\usepackage{pgf,pgfpages}
%\pgfpagesuselayout{4 on 1}[letterpaper,border shrink=5mm]

% FONTS
\usepackage{fontspec}
\def\mainfont{Linux Biolinum O}
\setmainfont[Ligatures=TeX, Contextuals={NoAlternate}, BoldFont={* Bold}, ItalicFont={* Italic}, Numbers={Proportional}]{\mainfont}
\setmonofont[Scale=MatchLowercase]{Inconsolata} 
\setsansfont[Scale=MatchLowercase]{Linux Biolinum O} 
\usepackage{microtype}

\usepackage{graphicx}
	\graphicspath{%
	{/Users/goby/Pictures/teach/478/lectures/}%
	{/Users/goby/Pictures/teach/common/}} % set of paths to search for images

\usepackage{amsmath,amssymb}

%\usepackage{units}

\usepackage{booktabs}
\usepackage{multicol}
%	\setlength{\columnsep=1em}


\usepackage{textcomp}
\usepackage{setspace}
\usepackage{tikz}
	\tikzstyle{every picture}+=[remember picture,overlay]

\mode<presentation>
{
  \usetheme{Lecture}
  \setbeamercovered{invisible}
  \setbeamertemplate{items}[square]
}

\begin{document}
%\lecture{instructor}{instructor}
%\lecture{student}{student}


%% External anatomy
{
\usebackgroundtemplate{\includegraphics[width=\paperwidth]{external_anatomy}}
\begin{frame}[b,plain]
\end{frame}
}

\begin{frame}[t,plain]{Fishes have one of four \highlight{scale types.}}
	\begin{minipage}{0.3\textwidth}
	\raggedright
	
	\highlight{Placoid}\\{\small Chondrichthys}
	
	\vspace{2em}

	\highlight{Ganoid}\\ {\small some primitive fishes}
	
	\vspace{2em}
	
	\highlight{Ctenoid}\\ {\small Teleostei}
	
	\vspace{2em}
	
	\highlight{Cycloid} \\{\small Teleostei}

	\vspace{2em}
	
	\end{minipage}\hfill\begin{minipage}{0.6\textwidth}

		\includegraphics[width=1\textwidth]{scale_types}
	\end{minipage}
\end{frame}

{
\usebackgroundtemplate{\includegraphics[width=\paperwidth]{mouth_position}}
\begin{frame}[b,plain]{Fishes have one of four \highlight{mouth positions.}}
\end{frame}
}

{
\usebackgroundtemplate{\includegraphics[width=\paperwidth]{fin_position}}
\begin{frame}[b,plain]{Paired fins are distributed in one of three \highlight{fin positions.}}
\end{frame}
}

{
\usebackgroundtemplate{\includegraphics[width=\paperwidth]{caudal_shape}}
\begin{frame}[b,plain]{Fishes have one of four \highlight{caudal fin shapes.}}
\end{frame}
}

\begin{frame}[t,plain]{The body \highlight{form} of fishes is tied to the ecological \highlight{function.}}

\hangpara Rover-Predator

\hangpara Lie-in-Wait Predator

\hangpara Surface-Oriented

\hangpara Deep-Bodied

\hangpara Eel-Like

\hangpara Bottom-Oriented

\end{frame}

{
\usebackgroundtemplate{\includegraphics[width=\paperwidth]{centropomus_rover_predator}}
\begin{frame}[b,plain]{\textcolor{white}{Rover-predators have a} \textcolor{orange6}{fusiform} \textcolor{white}{body and terminal mouth.}}
\hfill\tiny\textcolor{black}{\textit{Centropomus undecimalis} (snook), Matthew Hoelscher, Flickr Creative Commons.}
\end{frame}
}

{
\usebackgroundtemplate{\includegraphics[width=\paperwidth]{barracuda_lie_wait}}
\begin{frame}[b,plain]{\textcolor{white}{Lie-in-wait predators have a} \textcolor{orange6}{strongly fusiform} \textcolor{white}{body, and large terminal mouth.}}
\hfill\tiny\textcolor{white}{\textit{Sphyraena barracuda} (great barracuda), Wikipedia Creative Commons.}
\end{frame}
}

{
\usebackgroundtemplate{\includegraphics[width=\paperwidth]{anableps_surface_oriented}}
\begin{frame}[b,plain]{\textcolor{white}{Surface-oriented fishes have a flattened head and superior mouth.}}
\hfill\tiny\textcolor{white}{\textit{Anableps anableps} (largescape foureye), Cayambe, Wikipedia Creative Commons.}
\end{frame}
}

{
\usebackgroundtemplate{\includegraphics[width=\paperwidth]{butterflyfish_deepbody}}
\begin{frame}[b,plain]{Deep-bodied fishes have a flexible \textcolor{orange6}{compressiform} body.}
\hfill\tiny\textcolor{white}{\textit{Chaetodon melannotus}, Leonard Low, Flickr Creative Commons.}
\end{frame}
}

{
\usebackgroundtemplate{\includegraphics[width=\paperwidth]{form_eel_like}}
\begin{frame}[b,plain]{\textcolor{white}{Eel-like fishes have reduced or absent paired fins. Median fins may be confluent with caudal fin.}}
\hfill\tiny\textcolor{orange7}{Moray eel, Muraenidae: Anguilliformes. annie\_stru, Flickr Creative Commons.}
\end{frame}
}



{
\usebackgroundtemplate{\includegraphics[width=\paperwidth]{goatfish_bottom_rover}}
\begin{frame}[b,plain]{\textcolor{white}{Bottom-rovers are ventrally flattened with terminal to inferior mouths.}}
\hfill\tiny\textcolor{white}{\textit{File:Upeneichthys lineatus} (blue-lined goatfish, Ian Skipworth, Wikimedia Commons.}
\end{frame}
}

{
\usebackgroundtemplate{\includegraphics[width=\paperwidth]{goby_bottom_clinger}}
\begin{frame}[b,plain]{\textcolor{white}{Bottom clingers are small with fused pelvic fins.}}
\hfill\tiny\textcolor{white}{\textit{Cryptocentrus cinctus} (yellow prawn-goby), Nicolas Keller, Flickr Creative Commons.}
\end{frame}
}

{
\usebackgroundtemplate{\includegraphics[width=\paperwidth]{blenny_bottom_hider}}
\begin{frame}[t,plain]{}

\hangpara\Large\textcolor{white}{Bottom hiders are small \\
and live in crevices.}

\vskip0pt plus 1filll
\tiny\textcolor{white!80!black}{\textit{Acanthemblemaria spinosa} (spiny headed blenny), nhobgood, Wikimedia Commons.}
\end{frame}
}

{
\usebackgroundtemplate{\includegraphics[width=\paperwidth]{stingray_flatfish}}
\begin{frame}[b,plain]{Flatfishes are \highlight{compressiform} with eyes on top of head.}
\tiny\textit{Taeniura lymma} (bluespotted ribbontail ray)\hfill Jens Petersen, Wikimedia Commons.
\end{frame}
}


{
\usebackgroundtemplate{\includegraphics[width=\paperwidth]{nezumia_rattail}}
\begin{frame}[b,plain]{\textcolor{white}{Rattails have large heads and tapered tails.}}
\tiny\textcolor{white!80!black}{\textit{Nezumia} sp. (rattail), NOAA Photo Library.}
\end{frame}
}




\end{document}
