%!TEX TS-program = lualatex
%!TEX encoding = UTF-8 Unicode

\documentclass[t]{beamer}

%%%% HANDOUTS For online Uncomment the following four lines for handout
%\documentclass[t,handout]{beamer}  %Use this for handouts.
%\usepackage{handoutWithNotes}
%\pgfpagesuselayout{3 on 1 with notes}[letterpaper,border shrink=5mm]
%	\setbeamercolor{background canvas}{bg=black!5}


%%% Including only some slides for students.
%%% Uncomment the following line. For the slides,
%%% use the labels shown below the command.
%\includeonlylecture{student}

%% For students, use \lecture{student}{student}
%% For mine, use \lecture{instructor}{instructor}


%\usepackage{pgf,pgfpages}
%\pgfpagesuselayout{4 on 1}[letterpaper,border shrink=5mm]

% FONTS
\usepackage{fontspec}
\def\mainfont{Linux Biolinum O}
\setmainfont[Ligatures=TeX, Contextuals={NoAlternate}, BoldFont={* Bold}, ItalicFont={* Italic}, Numbers={Proportional}]{\mainfont}
\setmonofont[Scale=MatchLowercase]{Inconsolatazi4} 
\setsansfont[Scale=MatchLowercase]{Linux Biolinum O} 
\usepackage{microtype}

\usepackage{graphicx}
	\graphicspath{%
	{/Users/goby/Pictures/teach/478/lectures/}%
	{/Users/goby/Pictures/teach/common/}} % set of paths to search for images

\usepackage{amsmath,amssymb}

%\usepackage{units}

\usepackage{booktabs}
\usepackage{multicol}
%	\setlength{\columnsep=1em}

\usepackage{textcomp}
\usepackage{setspace}
\usepackage{tikz}
	\tikzstyle{every picture}+=[remember picture,overlay]

\mode<presentation>
{
  \usetheme{Lecture}
  \setbeamercovered{invisible}
  \setbeamertemplate{items}[square]
}

\usepackage{calc}
\usepackage{hyperref}

\newcommand\HiddenWord[1]{%
	\alt<handout>{\rule{\widthof{#1}}{\fboxrule}}{#1}%
}



\begin{document}
%\lecture{instructor}{instructor}
%\lecture{student}{student}

{
\usebackgroundtemplate{\includegraphics[width=\paperwidth]{thermoregulate_how}}
\begin{frame}[b,plain]

\tiny\textcolor{white!80!black}{\textit{Thunnus orientalis} (Pacific Bluefin Tuna, Scombridae: Perciformes), aes256, Wikimedia Commons.}
\end{frame}
}


{
\usebackgroundtemplate{\includegraphics[width=\paperwidth]{thermoregulate_teleost}}
\begin{frame}[b,plain]{The dorsal aorta and pseudocardinal vein run along the midline of most teleost fishes.}

\end{frame}
}

{
\usebackgroundtemplate{\includegraphics[width=\paperwidth]{thermoregulate_heterothermic}}
\begin{frame}[b,plain]{\highlight{Heterothermic} fishes use countercurrent exchange to elevate body temperature.}

\end{frame}
}

{
\usebackgroundtemplate{\includegraphics[width=\paperwidth]{osmoregulate_how}}
\begin{frame}[b,plain]

\tiny\textcolor{white!20!black}{\textit{Inimicus didactylus} (Demon Stinger, Synanceiidae: Scorpaeniformes), Bjoertvedt, Wikimedia Commons.}
\end{frame}
}

{
\usebackgroundtemplate{\includegraphics[width=\paperwidth]{osmoregulate_euryhaline}}
\begin{frame}[b,plain]{\textcolor{orange5}{Euryhaline} \textcolor{white}{fishes tolerate a wide range of salinity.}}

\hfill\tiny\textcolor{white!80!black}{\textit{Epiplatys annulatus} (Clown Killifish, Synanceiidae: Scorpaeniformes), \copyright Hristo Hristov, \url{seriouslyfish.com}.}
\end{frame}
}

{
\usebackgroundtemplate{\includegraphics[width=\paperwidth]{osmoregulate_stenohaline}}
\begin{frame}[b,plain]{\textcolor{orange5}{Stenohaline} \textcolor{white}{fishes tolerate a narrow range of salinity.}}

\hfill\tiny\textcolor{white!80!black}{\textit{Chrosomus cumberlandensis} (Blackside Dace, Cyprinidae: Cypriniformes), Dick Biggins, USFWS, Wikimedia Commons.}
\end{frame}
}

{
\usebackgroundtemplate{\includegraphics[width=\paperwidth]{osmoregulate_osmoconformer}}
\begin{frame}[b,plain]{The internal salinity of \highlight{osmoconformers} is about equal to seawater.}

\hfill\tiny\textcolor{white}{\textit{Eptatretus stouti}? (Pacific Hagfish, Myxinidae: Myxiniformes), NOAA Photo Library, Flickr Creative Commons.}
\end{frame}
}

\begin{frame}[c]{Plasma solute concentrations (mMol/L) of representative freshwater and marine fishes.}

\centering\begin{tabular}{lrrrrr}
\toprule
Species	&	Na+	&	Ca2+	&	K+	&	Urea	&	Total Salts \\
\midrule
\textbf{Freshwater}	&	< 1	&	< 1	&	< 1	&		&	1--10 \\
\hspace{1ex}Bass	&	120	&	3	&	3	&		&	 \\
\hspace{1ex}Whitefish	&	141	&	3	&	4	&	1	&	 \\
\hspace{1ex}Lamprey	&	120	&	2	&	3	&		&	270 \\
\textbf{Marine}	&	\textasciitilde450	&	\textasciitilde20	&	10	&		&	1000 \\
\hspace{1ex}Hagfish	&	549	&	5	&	11	&		&	1152 \\
\hspace{1ex}Anglerfish	&	198	&	2	&	3	&		&	 \\
\hspace{1ex}Moral eel	&	212	&	4	&	2	&		&	 \\
\hspace{1ex}Dogfish	&	263	&	7	&	4	&	357	&	1007 \\
\bottomrule
\end{tabular}

\end{frame}
	
{
\usebackgroundtemplate{\includegraphics[width=\paperwidth]{osmoregulate_sharks_urea}}
\begin{frame}[b,plain]{\textcolor{white}{Elasmobranchs maintain osmotic balance with organic compounds like urea and TMAO.}}

\hfill\tiny\textcolor{white!80!black}{\textit{Carcharhinus longimanus} (Oceanic Whitetip Shark, Carcharhinidae: Carcharhiniformes), Michael Aston, Flickr Creative Commons.}
\end{frame}
}

{
\usebackgroundtemplate{\includegraphics[width=\paperwidth]{osmoregulate_rectal_gland}}
\begin{frame}[b,plain]{Sharks use a \highlight{rectal gland} to eliminate excess salts.}

\hfill\tiny\textcolor{white}{Rectal gland of Atlantic Sharpnose Shark. \copyright Andy Murch, \url{elasmodiver.com}. All Rights Reserved. Used with permission.}

\end{frame}
}

%% Begin marine fish osmoregulation: Water and passive diffusion
{
\usebackgroundtemplate{\includegraphics[width=\paperwidth]{osmoregulate_marine_fish_outline}}
\begin{frame}[b,plain]{Marine fishes are \highlight{\HiddenWord{hypotonic}} relative to the environment.}

\begin{tikzpicture}
	
	\draw [thick, dashed] (0, 1.5) -- (1, 1.5);
	\node [right] at (1, 1.5) {\small{passive diffusion}};
	\draw [thick] (0, 1) -- (1, 1);
	\node [right] at (1, 1) {\small{active transport}};

	\pause %% Water and salts
	\node [right, orange6] at (5.7, 6.1) {\footnotesize{H$_2$O}};
	\draw [->, thick, dashed, orange6] (5, 5.5) -- (5.7, 6.1);

	\node [right, orange6] at (5.7, 5.7) {\footnotesize{Na$^+$},\,Cl$^-$};
	\draw [->, thick, dashed, orange6] (5.7, 5.7) -- (5.2, 5.3);

	\pause  %% Drink Sea water
	\node [right, orange6] at (0, 4.5){seawater};
	\draw [->, thick, orange6] (1.7, 4.5) -- (3, 4.5);

	% Alpha Chloride
%	\draw [->, thick] (4.9, 4) -- (5.9, 4.2);
%	\node [right] at (5.8, 4.25) {\footnotesize{Na$^+$},\,Cl$^-$,\,NH$^-_4$};
	
	% Kidneys
%	\draw [->, thick] (7.3, 3.8) -- (7.55, 2.65);
%	\node [right] at (7.4, 2.5) {\footnotesize{Mg$^{2+}$},\,SO$^{2-}_4$};

\end{tikzpicture}

\end{frame}
}

%% Marine fish: alpha cloride cells
{
\usebackgroundtemplate{\includegraphics[width=\paperwidth]{osmoregulate_marine_fish_outline}}
\begin{frame}[b,plain]{\highlight{Alpha chloride cells} in the gills eliminate monovalent ions.}

\begin{tikzpicture}
	
	\draw [thick, dashed] (0, 1.5) -- (1, 1.5);
	\node [right] at (1, 1.5) {\small{passive diffusion}};
	\draw [thick] (0, 1) -- (1, 1);
	\node [right] at (1, 1) {\small{active transport}};

	% \pause %% Water and salts
	\node [right] at (5.7, 6.1) {\footnotesize{H$_2$O}};
	\draw [->, thick, dashed] (5, 5.5) -- (5.7, 6.1);

	\node [right] at (5.7, 5.7) {\footnotesize{Na$^+$},\,Cl$^-$};
	\draw [->, thick, dashed] (5.7, 5.7) -- (5.2, 5.3);

	%\pause  %% Drink Sea water
	\node [right] at (0, 4.5){seawater};
	\draw [->, thick] (1.7, 4.5) -- (3, 4.5);

	% Alpha Chloride
	\draw [->, thick, orange6] (4.9, 4) -- (5.9, 4.2);
	\node [right] at (5.8, 4.25) {\highlight{\footnotesize{Na$^+$},\,Cl$^-$,\,NH$^-_4$}};
	
	% Kidneys
%	\draw [->, thick] (7.3, 3.8) -- (7.55, 2.65);
%	\node [right] at (7.4, 2.5) {\footnotesize{Mg$^{2+}$},\,SO$^{2-}_4$};

\end{tikzpicture}

\end{frame}
}


%% Marine fish: kidneys
{
\usebackgroundtemplate{\includegraphics[width=\paperwidth]{osmoregulate_marine_fish_outline}}
\begin{frame}[b,plain]{Kidneys excrete divalent ions.}

\begin{tikzpicture}
	
	\draw [thick, dashed] (0, 1.5) -- (1, 1.5);
	\node [right] at (1, 1.5) {\small{passive diffusion}};
	\draw [thick] (0, 1) -- (1, 1);
	\node [right] at (1, 1) {\small{active transport}};

	% \pause %% Water and salts
	\node [right] at (5.7, 6.1) {\footnotesize{H$_2$O}};
	\draw [->, thick, dashed] (5, 5.5) -- (5.7, 6.1);

	\node [right] at (5.7, 5.7) {\footnotesize{Na$^+$},\,Cl$^-$};
	\draw [->, thick, dashed] (5.7, 5.7) -- (5.2, 5.3);

	%\pause  %% Drink Sea water
	\node [right] at (0, 4.5){seawater};
	\draw [->, thick] (1.7, 4.5) -- (3, 4.5);

	% Alpha Chloride
	\draw [->, thick] (4.9, 4) -- (5.9, 4.2);
	\node [right] at (5.8, 4.25) {\footnotesize{Na$^+$},\,Cl$^-$,\,NH$^-_4$};
	
	% Kidneys
	\draw [->, thick, orange6] (7.3, 3.8) -- (7.55, 2.65);
	\node [right, orange6] at (7.4, 2.5) {\footnotesize{Mg$^{2+}$},\,SO$^{2-}_4$};

\end{tikzpicture}

\end{frame}
}

%% Begin Osmoregulate Freshwater Fishes
{
\usebackgroundtemplate{\includegraphics[width=\paperwidth]{osmoregulate_freshwater_fish_outline}}
\begin{frame}[b,plain]{Freshwater fishes are \highlight{\HiddenWord{hypertonic}} relative to the environment.}

\begin{tikzpicture}
	
	\draw [thick, dashed] (0, 1.5) -- (1, 1.5);
	\node [right] at (1, 1.5) {\small{passive diffusion}};
	\draw [thick] (0, 1) -- (1, 1);
	\node [right] at (1, 1) {\small{active transport}};

	\pause %% water and salts
	\node [right, orange6] at (5.8, 6.4) {\footnotesize{H$_2$O}};
	\draw [->, thick, dashed, orange6] (5.8, 6.3) -- (5.35, 5.45) ;

	\node [right, orange6] at (5.8, 6.0) {\footnotesize{Na$^+$}, Cl$^-$};
	\draw [->, thick, dashed, orange6] (5.4, 5.3) -- (5.9, 5.8) ;

	% Beta Chloride
%	\draw [->, thick, orange6] (5.9, 3.85) -- (5.0, 4.1) ;
%	\node [right] at (5.8, 3.85) {\footnotesize{Na$^+$},\,Cl$^-$,\,Ca$^{2+}$};
	
	% Kidneys
%	\draw [->, thick] (7.3, 3.34) -- (7.55, 2.55);
%	\node [right] at (7.4, 2.4) {\footnotesize{H$_2$O, salts}};

\end{tikzpicture}
\end{frame}
}

%%  Freshwater : Beta chloride cells
{
\usebackgroundtemplate{\includegraphics[width=\paperwidth]{osmoregulate_freshwater_fish_outline}}
\begin{frame}[b,plain]{\highlight{Beta chloride cells} import sodium and chloride ions.}

\begin{tikzpicture}
	
	\draw [thick, dashed] (0, 1.5) -- (1, 1.5);
	\node [right] at (1, 1.5) {\small{passive diffusion}};
	\draw [thick] (0, 1) -- (1, 1);
	\node [right] at (1, 1) {\small{active transport}};

	% \pause %% Food and salts
	\node [right] at (5.8, 6.4) {\footnotesize{H$_2$O}};
	\draw [->, thick, dashed] (5.8, 6.3) -- (5.35, 5.45) ;

	\node [right] at (5.8, 6.0) {\footnotesize{Na$^+$}, Cl$^-$};
	\draw [->, thick, dashed] (5.4, 5.3) -- (5.9, 5.8) ;

	% Beta Chloride
	\draw [->, thick, orange6] (5.9, 3.85) -- (5.0, 4.1) ;
	\node [right, orange6] at (5.8, 3.85) {\footnotesize{Na$^+$},\,Cl$^-$,\,Ca$^{2+}$};
	
	% Kidneys
%	\draw [->, thick] (7.3, 3.34) -- (7.55, 2.55);
%	\node [right] at (7.4, 2.4) {\footnotesize{H$_2$O, salts}};

\end{tikzpicture}
\end{frame}
}

%%  Freshwater : Kidneys
{
\usebackgroundtemplate{\includegraphics[width=\paperwidth]{osmoregulate_freshwater_fish_outline}}
\begin{frame}[b,plain]{Kidneys resorb divalent ions.}

\begin{tikzpicture}
	
	\draw [thick, dashed] (0, 1.5) -- (1, 1.5);
	\node [right] at (1, 1.5) {\small{passive diffusion}};
	\draw [thick] (0, 1) -- (1, 1);
	\node [right] at (1, 1) {\small{active transport}};

	% \pause %% Food and salts
	\node [right] at (5.8, 6.4) {\footnotesize{H$_2$O}};
	\draw [->, thick, dashed] (5.8, 6.3) -- (5.35, 5.45) ;

	\node [right] at (5.8, 6.0) {\footnotesize{Na$^+$}, Cl$^-$};
	\draw [->, thick, dashed] (5.4, 5.3) -- (5.9, 5.8) ;

	% Beta Chloride
	\draw [->, thick] (5.9, 3.85) -- (5.0, 4.1) ;
	\node [right] at (5.8, 3.85) {\footnotesize{Na$^+$},\,Cl$^-$,\,Ca$^{2+}$};
	
	% Kidneys
	\draw [->, thick, orange6] (7.3, 3.34) -- (7.55, 2.55);
	\node [right, orange6] at (7.4, 2.4) {\footnotesize{H$_2$O}};

\end{tikzpicture}
\end{frame}
}


{
\usebackgroundtemplate{\includegraphics[width=\paperwidth]{osmoregulate_euryhaline_fishes}}
\begin{frame}[b,plain]{\textcolor{white}{How do euryhaline fishes adjust to changing salinities?}}

\hfill\tiny\textcolor{white!80!black}{\textit{Neostethus lankesteri} (Phallostethidae: Atheriniformes), Mike Noren, Wikimedia Commons.}
\end{frame}
}


\end{document}
