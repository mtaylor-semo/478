%!TEX TS-program = lualatex
%!TEX encoding = UTF-8 Unicode

%\documentclass[t]{beamer}

%%%% HANDOUTS For online Uncomment the following four lines for handout
\documentclass[t,handout]{beamer}  %Use this for handouts.
\usepackage{handoutWithNotes}
\pgfpagesuselayout{3 on 1 with notes}[letterpaper,border shrink=5mm]
	\setbeamercolor{background canvas}{bg=black!5}


%%% Including only some slides for students.
%%% Uncomment the following line. For the slides,
%%% use the labels shown below the command.
%\includeonlylecture{student}

%% For students, use \lecture{student}{student}
%% For mine, use \lecture{instructor}{instructor}


%\usepackage{pgf,pgfpages}
%\pgfpagesuselayout{4 on 1}[letterpaper,border shrink=5mm]

% FONTS
\usepackage{fontspec}
\def\mainfont{Linux Biolinum O}
\setmainfont[Ligatures=TeX, Contextuals={NoAlternate}, BoldFont={* Bold}, ItalicFont={* Italic}, Numbers={Proportional}]{\mainfont}
\setmonofont[Scale=MatchLowercase]{Inconsolata} 
\setsansfont[Scale=MatchLowercase]{Linux Biolinum O} 
\usepackage{microtype}

\usepackage{graphicx}
	\graphicspath{%
	{/Users/goby/Pictures/teach/478/lectures/}%
	{/Users/goby/Pictures/teach/common/}} % set of paths to search for images

\usepackage{amsmath,amssymb}

%\usepackage{units}

\usepackage{booktabs}
\usepackage{multicol}
%	\setlength{\columnsep=1em}

\usepackage{textcomp}
\usepackage{setspace}
\usepackage{tikz}
	\tikzstyle{every picture}+=[remember picture,overlay]

\mode<presentation>
{
  \usetheme{Lecture}
  \setbeamercovered{invisible}
  \setbeamertemplate{items}[square]
}

\usepackage{calc}
\usepackage{hyperref}


\begin{document}
%\lecture{instructor}{instructor}
%\lecture{student}{student}

{
\usebackgroundtemplate{\includegraphics[width=\paperwidth]{circulation_intro}}
\begin{frame}[b,plain]{The circulatory system in fishes.}

\hfill\parbox{3.2cm}{\raggedright\tiny\textcolor{white}{\textit{Chionodraco hamatus} (an icefish, Channichthyidae: Perciformes), Marrabbio2, Wikimedia Commons.}}

\end{frame}
}


\begin{frame}[c,plain]{Fishes have a \highlight{single pump, single circuit} circulatory system.}
	\centering
	\includegraphics[width=\textwidth]{circulation_system}
	
\end{frame}

\begin{frame}[c,plain]{Fishes have a \highlight{four-chambered} heart.}

	\includegraphics[width=\textwidth]{circulation_heart_structure}
	
\end{frame}

\begin{frame}[c,plain]{Blood flows from ventral to dorsal through the gills.}

	\includegraphics[width=\textwidth]{circulation_gills_teleost}

	\phantom{\includegraphics[width=\textwidth]{circulation_gills_lungfish}}
	
\end{frame}

\begin{frame}[c,plain]{Lungfishes divert some blood from gills through the lungs.}

	\includegraphics[width=\textwidth]{circulation_gills_teleost}

	\includegraphics[width=\textwidth]{circulation_gills_lungfish}
	
\end{frame}


\begin{frame}[c,plain]{Hemoglobin is monomeric in hagfishes and lampreys.}

\vspace{\baselineskip}

	\centering
	\includegraphics[width=0.7\textwidth]{circulation_hemoglobin_monomer}

\vskip0pt plus 1filll
\hfill\tiny Modified from Hemogloblin diagram by OpenStax College, Wikimedia Commons.
\end{frame}


\begin{frame}[c,plain]{Hemoglobin is tetrameric in all other fishes.}

\vspace{\baselineskip}

	\centering
	\includegraphics[width=0.7\textwidth]{circulation_hemoglobin_structure}
	
	Hb + O$_2 \Longleftrightarrow $ HbO$_2$
	
\vskip0pt plus 1filll
\hfill\tiny Hemogloblin diagram by OpenStax College, Wikimedia Commons.
\end{frame}

\begin{frame}[c,plain]{\highlight{Oxygen affinity} describes how easily Hb binds to and releases O$_2$.}

\centering
\includegraphics{circulation_hb_affinity}

\pause

\begin{tikzpicture}

	\draw [dashed,thick] (-4.8,4) -- (3,4);
	\node [right] at (3,4) {P50};
	
\end{tikzpicture}
	
\end{frame}



{
\usebackgroundtemplate{\includegraphics[width=\paperwidth]{circulation_affects_affinity}}
\begin{frame}[t,plain]

	\vspace{2\baselineskip}
	
	\hangpara What factors affect\\  oxygen affinity?
	
	\vspace{4em}
	
	\hangpara Do you know\\ the bicarbonate\\ buffering system?
	
\vskip0pt plus 1filll
\hfill\tiny\textit{Gorgasia preclara} (Splendid Garden Eel, Congridae: Anguilliformes),\\
\hfill\tiny  Opencage.info, Creative Commons.
\end{frame}
}


\begin{frame}[c,plain]{\highlight{Bohr effect} describes how pH affects oxygen \highlight{affinity.}}

	\centering
	\includegraphics[width=\textwidth]{circulation_bohr_root}
	\pause
	
\begin{tikzpicture}
	%% Bohr shift
	\draw [->, thick] (-3.2,5.2) -- (-0.4,5.2);
	\draw [->, thick] (0.1,5.2) -- (2.3,5.2);
	\node [above] at (-1.8,5.2) {Bohr Effect};

\end{tikzpicture}

\end{frame}


\begin{frame}[c,plain]{\highlight{Root effect} describes how pH affects oxygen \highlight{capacity.}}

	\centering
	\includegraphics[width=\textwidth]{circulation_bohr_root}
	\pause
	
\begin{tikzpicture}
	% Root shift	
	\draw [->, thick] (4.4,8.1) -- (4.4,6.4);
	\draw [->, thick] (4.4,6.2) -- (4.4,5.38);
	\node [left, text width=1cm] at (4.4,7.25) {Root Effect};
	
\end{tikzpicture}

\end{frame}


\begin{frame}[c,plain]{Compare and contrast O$_2$ loading and unloading in the gills and the body.}

	\vspace{\baselineskip}
	\includegraphics[width=\textwidth]{circulation_loading_unloading}
	\vskip0pt plus 1filll

\hfill\tiny\textit{Enoplosus armatus} (Old Wife, Enoplosidae:Perciformes), Richard Ling, Wikimedia Commons.
\end{frame}

\end{document}
