%!TEX TS-program = lualatex
%!TEX encoding = UTF-8 Unicode

\documentclass[t]{beamer}

%%%% HANDOUTS For online Uncomment the following four lines for handout
%\documentclass[t,handout]{beamer}  %Use this for handouts.
%\includeonlylecture{student}
%\usepackage{handoutWithNotes}
%\pgfpagesuselayout{3 on 1 with notes}[letterpaper,border shrink=5mm]


%% For students, use \lecture{student}{student}
%% For mine, use \lecture{instructor}{instructor}


% FONTS
\usepackage{fontspec}
\def\mainfont{Linux Biolinum O}
\setmainfont[Ligatures={Common,TeX}, Contextuals={NoAlternate}, BoldFont={* Bold}, ItalicFont={* Italic}, Numbers={OldStyle}]{\mainfont}
\setsansfont[Ligatures={Common,TeX}, Scale=MatchLowercase, Numbers=OldStyle]{Linux Biolinum O} 
\usepackage{microtype}

\usepackage{graphicx}
	\graphicspath{%
	{/Users/goby/Pictures/teach/478/lectures/}%
	{/Users/goby/Pictures/teach/common/}} % set of paths to search for images

\usepackage{amsmath,amssymb}

%\usepackage{units}

\usepackage{booktabs}
\usepackage{multicol}
%	\setlength{\columnsep=1em}


\usepackage{textcomp}
\usepackage{setspace}
\usepackage{tikz}
	\tikzstyle{every picture}+=[remember picture,overlay]

\mode<presentation>
{
  \usetheme{Lecture}
  \setbeamercovered{invisible}
  \setbeamertemplate{items}[square]
}

\begin{document}
%\lecture{instructor}{instructor}
%\lecture{student}{student}

%% Intro

{
\usebackgroundtemplate{\includegraphics[width=\paperwidth]{intro}}
\begin{frame}[b,plain]
	\hfill\tiny\textcolor{gray}{Chimaera by NOAA Ocean Explorer, Flickr Creative Commons.}
\end{frame}
}

%% Contact Info
{
\usebackgroundtemplate{\includegraphics[width=\paperwidth]{mike_fish}}
\begin{frame}[t,plain]{I will be your host for this class.}
%\large
	\vspace{5ex}
	\hangpara\hspace{17em} Mike Taylor

	\hangpara\hspace{17em} RH 217

	\hangpara\hspace{17em} mtaylor@semo.edu

\end{frame}
}

%% Grades
\begin{frame}[t,plain]{You earn your grade with }

	\hangpara \highlight{Online assignments} @ 10 points, 

	\hspace{2em} Given approximately every second lecture.

	\hangpara \highlight{Three exams} @ 100/125 points, 

	\hangpara \highlight{Three lab practicals} @ 100 points,
	
	\hangpara \highlight{Ecological life history paper} @ 100 points,
	
	\hangpara \highlight{Presentation or evaluation} @10/50 points
	
	\hspace{2em} Grads will present their life history paper (50 points),
	
	\hspace{2em} Undergrads will evaluate (10 points).

\end{frame}

%% Ecological Life History Paper
\begin{frame}[t,plain]{Each of you will write an \highlight{ecological life history paper. }}

	\hangpara You choose your species.
	
	\hspace{2em} Research your choice in advance, and then choose wisely! 

	\hangpara Your paper will cover ecology, taxonomy, and life history.

	\hangpara \highlight{Due dates are}
	
	\hspace{2em} Fish choice: 19 September,
	
	\hspace{2em} Bibliography: 17 October,
	
	\hspace{2em} Paper: 14 November.
	
	\hangpara Graduate students will present during final exam period.

\end{frame}

%% Field Trips
\begin{frame}[t,plain]{We will take \highlight{three Saturday field trips.}}

	\hangpara Ozark upland stream: TBD %6 September (alternate 13 Sep). 
	
	\hangpara Lowland ditch: TBD %27 Sep (alternate ?).

	\hangpara Mississippi River: TBD %4 Oct (alternate ?). 


	\hangpara
	
	\hangpara The Ozark and Lowland trips will depart from campus at 8am. Return times unknown.
	
	\hangpara You are \highlight{required} to participate on two trips for 25 points each.\\ If you attend all three trips, you earn 10  extra credit points.
	
\end{frame}

%% Textbooks
\begin{frame}[t,plain]{These two texts are for this course.}
\begin{multicols}{2}
	\begin{center}
		\includegraphics[width=0.9\columnwidth]{helfman_cover} \\
		\includegraphics[width=0.9\columnwidth]{pflieger_cover} \\
	\end{center}
\end{multicols}
\end{frame}


%% Defining a fish
{
\usebackgroundtemplate{\includegraphics[width=\paperwidth]{what_is_fish01}}
\begin{frame}[t,plain]{What is a fish?}
\end{frame}
}


\begin{frame}[t,plain]{A fish is\dots}

	\hangpara A poikilothermic, aquatic chordate with appendages (when present) developed as fins, whose chief respiratory organs are gills, and whose body is usually covered with scales (Berra 2001).

	\hangpara An aquatic vertebrate with gills and with limbs in the shape of fins (Nelson 2006)

\end{frame}

%% How many fishes are there?
{
\usebackgroundtemplate{\includegraphics[width=\paperwidth]{mola}}
\begin{frame}[b,plain]{\textcolor{white}{How many species of fishes are there?}}
	\hfill\tiny\textcolor{gray}{\textit{Mola mola} by Ilse Reijs and Jan-Noud Hutten, Flickr Creative Commons.}
\end{frame}
}

\begin{frame}[t,plain]{Fishes outnumber all other vertebrates combined.}
\begin{multicols}{2}
	\begin{center}
		\includegraphics[width=0.75\columnwidth]{nelson_fishes_world} \\
	\end{center}
	
	\columnbreak
	
	\hangpara 62 orders

	\hangpara 515 families
		
	\hangpara 4494 genera
		
	\hangpara 27,977 species
	
	\hangpara 34,000 species listed in {\small\highlight{\url{fishbase.org}}}
		
\end{multicols}
\end{frame}

%% Why study fishes?
{
\usebackgroundtemplate{\includegraphics[width=\paperwidth]{why_study_fishes}}
\begin{frame}[t,plain]{Why should we study fishes?}
\end{frame}
}

%% Classes of fishes
\begin{frame}[t,plain]{Your text lists five classes of fishes. {\small See pages 4 and 231.}}
	\hangpara{\small Domain Eukarya
	
	\hspace{1em} Kingdom Metazoa
	
	\hspace{2em} Phylum Chordata
	
	\hspace{3em} Subphylum Craniata}

	\hspace{5em} Class Myxini (hagfishes)
	
	\hspace{4em} {\small Infraphylum Vertebrata}

	\hspace{5em} Class Petromyzontida (lampreys)

	\hspace{5em} Class Chondrichthys (sharks and rays)

	\hspace{5em} Class Actinopterygii (ray-finned fishes)

	\hspace{5em} Class Sarcopterygii (lobe-finned fishes and tetrapods)
	
	\hangpara\highlight{Does this classification mean that Myxini are not vertebrates?}
	
\end{frame}

{
\usebackgroundtemplate{\includegraphics[width=\paperwidth]{cyclostome_miRNA_tree}}
\begin{frame}[b,plain]{Molecular evidence supports \highlight{monophyly} of hagfishes and lampreys.}
	\hfill\textcolor{gray}{Heimberg et al. 2010. Proc. Natl. Acad. Sci. 107:  19379}
\end{frame}
}


\begin{frame}[t,plain]{We will follow this classification scheme.}
	\hangpara{\small Domain Eukarya
	
	\hspace{1em} Kingdom Metazoa
	
	\hspace{2em} Phylum Chordata
	
	\hspace{3em} Subphylum Craniata
	
	\hspace{4em} Infraphylum Vertebrata}

	\hspace{5em} Superclass Cyclostomata (jawless vertebrates)

	\hspace{6em} Class Myxini (hagfishes)

	\hspace{6em} Class Petromyzontida (lampreys)

	\hspace{5em} Superclass Gnathostomata (jawed vertebrates)

	\hspace{6em} Class Chondrichthys (sharks and rays)

	\hspace{6em} Class Actinopterygii (ray-finned fishes)

	\hspace{6em} Class Sarcopterygii (lobe-finned fishes and tetrapods)
	
\end{frame}

%% Classification levels
\begin{frame}[t,plain]{You must know the suffixes for higher classification levels.}

	\hangpara Order: \highlight{-iformes}
	
	\hspace{2em} Saccopharyngiformes
	
	\hspace{2em} saccopharyngiform (informal. Note change in capitalization.)

	\hangpara Suborder: \highlight{-oidei}
	
	\hspace{2em} Anabantoidei
	
	\hspace{2em} anabantoid (informal)
	
	\hangpara Family: \highlight{-idae}
	
	\hspace{2em} Salmonidae
	
	\hspace{2em} salmonid (informal)
	
\end{frame}

%% Look up the names
\begin{frame}[t,plain]{Look up the names used in this course.}
\begin{multicols}{2}
	\begin{center}
		\includegraphics[width=0.9\columnwidth]{gulper} \\
		Saccopharyngiformes\\
		\columnbreak
		\includegraphics[width=0.9\columnwidth]{colisa_lalia} \\
		anabantoid\\
	\end{center}
\end{multicols}
	\vfilll
	\tiny\textcolor{gray}{\textit{Eurypharynx pelecanoides} (left) by Claf Hong, Flickr Creative Commons.\hfill\textit{Colisa lalia} (right), public domain, Wikimedia Commons.}
\end{frame}


%% External anatomy
{
	\usebackgroundtemplate{\includegraphics[width=\paperwidth]{external_anatomy}}
	\begin{frame}[b,plain]
\end{frame}
}

\begin{frame}[t,plain]{Fishes have one of four \highlight{scale types.}}
\begin{minipage}{0.3\textwidth}
\raggedright

\highlight{Placoid}\\{\small Chondrichthys}

\vspace{2em}

\highlight{Ganoid}\\ {\small some primitive fishes}

\vspace{2em}

\highlight{Ctenoid}\\ {\small Teleostei}

\vspace{2em}

\highlight{Cycloid} \\{\small Teleostei}

\vspace{2em}

\end{minipage}\hfill\begin{minipage}{0.6\textwidth}

\includegraphics[width=1\textwidth]{scale_types}
\end{minipage}
\end{frame}

{
\usebackgroundtemplate{\includegraphics[width=\paperwidth]{mouth_position}}
\begin{frame}[b,plain]{Fishes have one of four \highlight{mouth positions.}}
\end{frame}
}

{
\usebackgroundtemplate{\includegraphics[width=\paperwidth]{fin_position}}
\begin{frame}[b,plain]{Paired fins are distributed in one of three \highlight{fin positions.}}
\end{frame}
}

{
\usebackgroundtemplate{\includegraphics[width=\paperwidth]{caudal_shape}}
\begin{frame}[b,plain]{Fishes have one of four \highlight{caudal fin shapes.}}
\end{frame}
}

\begin{frame}[t,plain]{The body \highlight{form} of fishes is tied to the ecological \highlight{function.}}

\hangpara Rover-Predator

\hangpara Lie-in-Wait Predator

\hangpara Surface-Oriented

\hangpara Deep-Bodied

\hangpara Eel-Like

\hangpara Bottom-Oriented

\end{frame}

{
\usebackgroundtemplate{\includegraphics[width=\paperwidth]{centropomus_rover_predator}}
\begin{frame}[b,plain]{\textcolor{white}{Rover-predators have a} \textcolor{orange6}{fusiform} \textcolor{white}{body and terminal mouth.}}
\hfill\tiny\textcolor{black}{\textit{Centropomus undecimalis} (snook), Matthew Hoelscher, Flickr Creative Commons.}
\end{frame}
}

{
\usebackgroundtemplate{\includegraphics[width=\paperwidth]{barracuda_lie_wait}}
\begin{frame}[b,plain]{\textcolor{white}{Lie-in-wait predators have a} \textcolor{orange6}{strongly fusiform} \textcolor{white}{body and large terminal mouth.}}
\hfill\tiny\textcolor{white}{\textit{Sphyraena barracuda} (great barracuda), Wikipedia Creative Commons.}
\end{frame}
}

{
\usebackgroundtemplate{\includegraphics[width=\paperwidth]{anableps_surface_oriented}}
\begin{frame}[b,plain]{\textcolor{white}{Surface-oriented fishes have a flattened head and superior mouth.}}
\hfill\tiny\textcolor{white}{\textit{Anableps anableps} (largescape foureye), Cayambe, Wikipedia Creative Commons.}
\end{frame}
}

{
\usebackgroundtemplate{\includegraphics[width=\paperwidth]{butterflyfish_deepbody}}
\begin{frame}[b,plain]{Deep-bodied fishes have a flexible \textcolor{orange6}{compressiform} body.}
\hfill\tiny\textcolor{white}{\textit{Chaetodon melannotus}, Leonard Low, Flickr Creative Commons.}
\end{frame}
}

{
\usebackgroundtemplate{\includegraphics[width=\paperwidth]{form_eel_like}}
\begin{frame}[b,plain]{\textcolor{white}{Eel-like fishes have reduced or absent paired fins. Median fins may be confluent with caudal fin.}}
\hfill\tiny\textcolor{orange7}{Moray eel, Muraenidae: Anguilliformes. annie\_stru, Flickr Creative Commons.}
\end{frame}
}



{
\usebackgroundtemplate{\includegraphics[width=\paperwidth]{goatfish_bottom_rover}}
\begin{frame}[b,plain]{\textcolor{white}{Bottom-rovers are ventrally flattened with terminal to inferior mouths.}}
\hfill\tiny\textcolor{white}{\textit{File:Upeneichthys lineatus} (blue-lined goatfish, Ian Skipworth, Wikimedia Commons.}
\end{frame}
}

{
\usebackgroundtemplate{\includegraphics[width=\paperwidth]{goby_bottom_clinger}}
\begin{frame}[b,plain]{\textcolor{white}{Bottom clingers are small with fused pelvic fins.}}
\hfill\tiny\textcolor{white}{\textit{Cryptocentrus cinctus} (yellow prawn-goby), Nicolas Keller, Flickr Creative Commons.}
\end{frame}
}

{
\usebackgroundtemplate{\includegraphics[width=\paperwidth]{blenny_bottom_hider}}
\begin{frame}[t,plain]{}

\hangpara\Large\textcolor{white}{Bottom hiders are small \\
and live in crevices.}

\vskip0pt plus 1filll
\tiny\textcolor{white!80!black}{\textit{Acanthemblemaria spinosa} (spiny headed blenny), nhobgood, Wikimedia Commons.}
\end{frame}
}

{
\usebackgroundtemplate{\includegraphics[width=\paperwidth]{stingray_flatfish}}
\begin{frame}[b,plain]{Flatfishes are \highlight{compressiform} with eyes on top of head.}
\tiny\textit{Taeniura lymma} (bluespotted ribbontail ray)\hfill Jens Petersen, Wikimedia Commons.
\end{frame}
}


{
\usebackgroundtemplate{\includegraphics[width=\paperwidth]{nezumia_rattail}}
\begin{frame}[b,plain]{\textcolor{white}{Rattails have large heads and tapered tails.}}
\tiny\textcolor{white!80!black}{\textit{Nezumia} sp. (rattail), NOAA Photo Library.}
\end{frame}
}


\end{document}
