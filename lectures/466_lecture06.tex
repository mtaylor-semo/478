%!TEX TS-program = lualatex
%!TEX encoding = UTF-8 Unicode

\documentclass[t]{beamer}

%%%% HANDOUTS For online Uncomment the following four lines for handout
%\documentclass[t,handout]{beamer}  %Use this for handouts.
%\includeonlylecture{student}
%\usepackage{handoutWithNotes}
%\pgfpagesuselayout{3 on 1 with notes}[letterpaper,border shrink=5mm]
%	\setbeamercolor{background canvas}{bg=black!5}


%%% Including only some slides for students.
%%% Uncomment the following line. For the slides,
%%% use the labels shown below the command.

%% For students, use \lecture{student}{student}
%% For mine, use \lecture{instructor}{instructor}

% FONTS
\usepackage{fontspec}
\def\mainfont{Linux Biolinum O}
\setmainfont[Ligatures=TeX, Contextuals={NoAlternate}, BoldFont={* Bold}, ItalicFont={* Italic}, Numbers={Proportional, OldStyle}]{\mainfont}
\setsansfont[Scale=MatchLowercase]{Linux Biolinum O} 
\usepackage{microtype}

\usepackage{graphicx}
	\graphicspath{%
	{/Users/goby/Pictures/teach/466/lectures/}}%
%	{/Users/goby/Pictures/teach/common/}} % set of paths to search for images

\usepackage{amsmath,amssymb}

%\usepackage{units}

\usepackage{booktabs}
\usepackage{multicol}
%	\setlength{\columnsep=1em}

%\usepackage{textcomp}
%\usepackage{setspace}
\usepackage{tikz}
	\tikzstyle{every picture}+=[remember picture,overlay]

\mode<presentation>
{
  \usetheme{Lecture}
  \setbeamercovered{invisible}
  \setbeamertemplate{items}[square]
}

%\usepackage{calc}
\usepackage{hyperref}


\newcommand{\cornell}[1]{Fig.~#1~Lovette and Fitzpatrick, 2016. 3rd ed.}

\newcommand{\backskip}{\vspace{-0.5\baselineskip}}

\begin{document}
%\lecture{instructor}{instructor}

\lecture{student}{student}
{
\usebackgroundtemplate{\includegraphics[width=\paperwidth]{mating_intro}}
\begin{frame}[b,plain]
	
	\tinyfill \textcolor{white}{White-throated Kingfisher by Akshay Charegaonkar, \href{https://www.flickr.com/photos/27286306@N07/7616501040}{Flickr}, \ccbyncsa{2}} 
\end{frame}
}

%%

\begin{frame}[t]{Overview of mating and mating systems}

\begin{multicols}{2}
\hangpara Territories for breeding

\hangpara Choosing a mate\\
	\quad Female choice \\
	\quad Male choice?

\hangpara Attracting a mate

\hangpara Mating systems

\columnbreak

\noindent\includegraphics[width=\linewidth]{mating_snail_kites}

\end{multicols}

\vfilll

\tinyfill \href{https://www.flickr.com/photos/66020093@N03/27898298881}{Snail Kites, Andy Morfew, Flickr, \ccby{2}}
\end{frame}

%%

\begin{frame}[t]{\highlight{Resource-based territories} are defended for mating, nesting, and feeding.}

\vspace{-0.5\baselineskip}
\begin{multicols}{2}
\centering
\noindent\includegraphics[width=0.7\linewidth]{mating_northern_goshawk}

%\hangpara Size depends on resource needs of birds and habitat quality.

%\hangpara May be defended by one or both sexes.

\columnbreak

\noindent\includegraphics[width=\linewidth]{mating_territory_resource}
\end{multicols}

\vfilll

\tiny Northern Goshawk, Norbert Kenntner, \href{https://commons.wikimedia.org/wiki/File:Northern_Goshawk_ad_M2.jpg}{Wikimedia}, \ccbysa{3} \hfill \cornell{9.26}

\end{frame}

%%

\begin{frame}[t]{\highlight{Mating territories} are defended for mating and nesting.}

\vspace{-0.5\baselineskip}

\centering
\includegraphics[width=0.93\linewidth]{mating_territory_mating}

\vfilll

\tinyfill \cornell{9.17}

\end{frame}

%%

\begin{frame}[t]{\highlight{Nesting territories} are defended for nesting.}

\vspace{-0.5\baselineskip}

\centering

\includegraphics[width=\linewidth]{mating_territory_nesting}

\vfilll

\tinyfill \href{https://www.flickr.com/photos/27345927@N07/4542373073}{Gannets by russellstreet, Flickr}, \ccbysa{2}

\end{frame}

%%

\begin{frame}[t]{\highlight{Lek territories} are defended for mating.}
\vspace{-0.5\baselineskip}

\begin{multicols}{2}

\noindent\includegraphics[width=\linewidth]{mating_lek_grouse}

\hangpara Some lekking species have \highlight{exploded leks.}

\columnbreak

\noindent\includegraphics[width=\linewidth]{mating_territory_lek}

\end{multicols}

\vfilll

\tiny \href{https://www.flickr.com/photos/54430347@N04/6948123054}{Greater Sage-Grouse, \textsc{usfws, pacific southwest}, Flickr}, \ccby{2} \hfill \href{https://en.wikipedia.org/wiki/File:Lek-diagram.jpg}{Sadi Carnot, Wikimedia}, \ccbysa{3}

\end{frame}

\begin{frame}[t]{\highlight{Floaters} move in if a territory-holder disappears.}
\vspace{-0.5\baselineskip}
\begin{multicols}{2}

\noindent\includegraphics[width=\linewidth]{mating_territory_floaters}

\vfilll

{\tiny
Above: Zack and Stutchbury 1992. Behaviour 123:194.\\
Right, top: Sergio et al.~2009. J.~Anim.~Ecol.~78:109.\\
Right, bottom: \href{https://commons.wikimedia.org/wiki/File:Black_kite_(Milvus_migrans_migrans)_in_flight.jpg}{Black Kite, Charles Sharp, \ccbysa{4}}\\
}

\columnbreak

\noindent\includegraphics[width=\linewidth]{mating_floater_age_groups}

\noindent\includegraphics[width=\linewidth]{mating_floater_black_kite}

\end{multicols}
\end{frame}

%%

\begin{frame}[t]{Why are females choosy about their mates?}

\begin{multicols}{2}

\hangpara \highlight{Genetic benefits}

\hangpara \quad Sexy sons

\hangpara \quad Good genes

\columnbreak

\highlight{Material benefits}

\end{multicols}

\hangpara These hypotheses stem from \highlight{sexual conflict} between males and females.

\end{frame}

%%

\begin{frame}[t]{\highlight{Sexy sons:} females choose best males so their sons have increased chance of reproductive success.}

\vspace{-0.5\baselineskip}
\centering
\includegraphics[width=0.85\linewidth]{mating_rwbl_male_female}

\vfilll

\tinyfill Original source unknown.

\end{frame}

%%

\begin{frame}[t]{\highlight{Good genes:} females choose best males so their offspring .}

\vspace{-0.5\baselineskip}
\centering
\includegraphics[width=\linewidth]{mating_good_genes_rnph}

\vfilll

\tinyfill \href{https://www.flickr.com/photos/36542741@N03/4454127334}{Ring-necked Pheasant, Hiyashi Haka, Flickr, \ccbyncsa{2}}

\end{frame}

%%

\begin{frame}[t]{\highlight{Honest indicators} tell female about male's ability to provide material resources.}
\vspace{-0.5\baselineskip}
\centering
\includegraphics[width=\linewidth]{mating_hofi_honest_indicator}

\vfilll

\tiny \href{https://www.flickr.com/photos/28156071@N00/4104955050}{Don DeBold, Flickr, \ccby{2}} \hfill \href{https://www.audubon.org/news/house-finch-or-purple-finch-heres-how-tell-them-apart}{Michele Black, Audubon Photography}


\end{frame}

%%

\lecture{instructor}{instructor}
{
\usebackgroundtemplate{\includegraphics[width=\paperwidth]{mating_vogelkop_bowerbird}}
\begin{frame}[b,plain]
	
	\tinyfill \textcolor{white}{ \href{https://macaulaylibrary.org/asset/188495711}{Vogelkop Bowerbird, \textcopyright\,Arco Huang, Macaulay Library \textsc{ml188495711}}} 
\end{frame}
}

%%

\lecture{student}{student}

\begin{frame}[t]{Do males choose among females?}

\vspace{-0.5\baselineskip}
\includegraphics[width=\linewidth]{mating_male_choice_epfl}

\hangpara \hfill Males prefer females with white foreheads.

\vfilll

\tinyfill \cornell{9.06}

\end{frame}

%%

\begin{frame}[t]


\begin{multicols}{2}

\noindent\includegraphics[width=\linewidth]{mating_uv_blue_tit_design}

\columnbreak

\centering
\noindent\includegraphics[width=0.9\linewidth]{mating_uv_tit_preference}
\end{multicols}

\vfilll

\tiny Withgott 2000.~Bioscience 50:854. \hfill Hunt et al.~1999. Anim.~Behav.~58:809.
\end{frame}

%%

\begin{frame}[t]{Males attract mates with exaggerated \highlight{secondary sexual traits.}}

\vspace{-0.5\baselineskip}
\includegraphics[width=\linewidth]{mating_satin_bowerbird}

\vfilll

\tinyfill \cornell{9.B1.02}

\end{frame}

{
\usebackgroundtemplate{\includegraphics[width=\paperwidth]{mating_frigatebird_display}}
\begin{frame}[b,plain]{Ornaments, displays, and vocalizations can all be used to attract females.}
	
	\vfilll
	
	\tiny \colorbox{black!40!lime}{\href{https://commons.wikimedia.org/wiki/File:Male_greater_frigate_bird_displaying.jpg}{Greater Frigatebird, Charles Sharp, Wikimedia, \ccbysa{3}} \href{https://www.youtube.com/watch?v=RbVJD7R_VzU}{Link to Video}}
\end{frame}
}

%%

{
\usebackgroundtemplate{\includegraphics[width=\paperwidth]{mating_flame_bowerbird}}
\begin{frame}[b,plain]
	
	\vfilll
	
	\tinyfill \colorbox{brown}{\href{https://macaulaylibrary.org/asset/185096241}{Flame Bowerbird, Chris Wiley, \textcopyright\,Macaulay Library, \textsc{ml185096241}} \href{https://www.youtube.com/watch?v=1XkPeN3AWIE}{Link to Video}}
\end{frame}
}

%%

{
\usebackgroundtemplate{\includegraphics[width=\paperwidth]{mating_bird_of_paradise}}
\begin{frame}[b,plain]
	
	\vfilll
	
	\tinyfill  \textcolor{white}{\href{https://www.macaulaylibrary.org/2017/06/30/dance-moves-support-evidence-for-new-bird-of-paradise-species/}{Vogelkop Lophorina, \textcopyright\,Tim Laman, Macaulay Library, \textsc{ml62128001}} \href{https://www.youtube.com/watch?v=rX40mBb8bkU}{Link to Video}}
\end{frame}
}

%%

{
\usebackgroundtemplate{\includegraphics[width=\paperwidth]{mating_blue_manakin}}
\begin{frame}[b,plain]
	
	\vfilll
	
	\tiny  \textcolor{black}{\href{https://macaulaylibrary.org/asset/45272931}{Swallow-tailed Manakin, \textcopyright\,Nigel Voaden, Macaulay Library, \textsc{ml45272931}} \href{https://www.youtube.com/watch?v=1zxJPQlFFTI}{Link to Video}}
\end{frame}
}

%%

\begin{frame}[t]{Birds have a wide range of \highlight{mating systems.}}

\vspace{-0.5\baselineskip}


\includegraphics[width=\linewidth]{mating_systems}

\vfilll

\tinyfill \cornell{9.B5.01}

\end{frame}

%%

\begin{frame}[t]{Mating system categories (for reference only).}

{\raggedbottom
\begin{multicols}{3}

\textbf{Social Monogamy}

\vspace{0.5\baselineskip}

\highlight{mate guarding}

mate assistance

female enforced

\columnbreak

\textbf{Polyandry}

\vspace{0.5\baselineskip}

sequential 

male enforced

convenience

material benefits

fertility insurance

genetic benefits

\columnbreak

\textbf{Polygyny}

\vspace{0.5\baselineskip}

deception based

female defense

\highlight{resource defense}

\highlight{lek}

\end{multicols}}

\end{frame}

%%

\begin{frame}[t]{True \highlight{genetic monogamy} is rare.}
\vspace{-0.5\baselineskip}
\centering

\includegraphics[height=0.85\textheight]{mating_genetic_monogamy}

\vfilll
\tinyfill \cornell{9.13}
\end{frame}

%%

\begin{frame}[t]{About 90\% of all species have \highlight{extra-pair copulations.}}

\backskip

\includegraphics[width=\linewidth]{mating_epc_eabl}

\vfilll

\tinyfill \href{https://www.flickr.com/photos/bugbait/7973909138}{Eastern Bluebird \textcopyright\,Larry McGahey, Flickr.}

\end{frame}

%%

\begin{frame}[t]{Males may \highlight{mate guard} to prevent female \textsc{epc.}}

\backskip
\centering
\includegraphics[width=0.95\linewidth]{mating_mate_guarding}

\vfilll

\tiny Seychelles Warbler \hfill \cornell{9.12}
\end{frame}

%%

\begin{frame}[t]{\highlight{Polyandry} might evolve when territory space is limited.}
\backskip

\includegraphics[width=\linewidth]{mating_polyandry_galapagos_hawk}

\vfilll

\tinyfill Galapagos Hawks, photographer unknown.
\end{frame}

%%

\begin{frame}[t]{\highlight{Sequential polyandry} allows females to mate more often.}

\backskip

\includegraphics[width=\linewidth]{mating_sequential_polyandry}

\vfilll

\tinyfill Red-necked Phalarope, photographer unknown.
\end{frame}

%%

\begin{frame}[t]{Polyandry can result from \highlight{forced copulations.}}

\backskip

\includegraphics[width=\linewidth]{mating_forced_copulation}

\vfilll

\tinyfill \cornell{9.23}
\end{frame}

%%

\begin{frame}[t]{Females may be able to retain male choice.}

\backskip

\begin{multicols}{2}
\centering

\noindent\includegraphics[width=0.78\linewidth]{mating_stiff_tailed_duck}

\columnbreak

\noindent \includegraphics[width=\linewidth]{mating_interomittant_organs}

\vfilll

\tinyfill McCracken et al.~2001.~Nature 413: 128\newline 
\tinyfill Brennan et al.`2007.~PLoS ONE 2(5):e418. 

\end{multicols}

\end{frame}


%% Female RWBL prefer to mate with unmated male (no female in territory.
%% But, they will chose a mated male if a nest is already over water,
%% which is safer.
\begin{frame}[t]{\highlight{Resource-defense polygyny} may occur when males defend high-quality habitat.}
\backskip
\centering
\includegraphics[height=0.77\textheight]{mating_polygyny_rwbl}

\vfilll

\tiny See Pribil and Searcy 2001.~Proc.~R.~Soc.~Lond.~268:1643. \hfill \href{https://i0.wp.com/auduboncnc.org/wp-content/uploads/2018/06/Redwinged-blackbird-nest.jpg?ssl=1}{\textcopyright\,Audubon Community Nature Center}
\end{frame}

%%

\begin{frame}[t]{\highlight{Leks} are aggregations of males that engage in competitive mating displays. }

\backskip

\includegraphics[width=\linewidth]{mating_lek_lpch}

\backskip
\hangpara \highlight{Exploded leks} are diffuse. Males might not be within sight of each other.
\vfilll

\tiny \href{https://www.youtube.com/watch?v=OPzDtTyzGIg}{Link to Video} \hfill \href{https://photofeathers.wordpress.com/tag/lesser-prairie-chicken-festival/}{Lesser Prairie-Chicken, \textcopyright\,Linda Rockwell}

\end{frame}

%%


\end{document}


