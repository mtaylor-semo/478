%!TEX TS-program = lualatex
%!TEX encoding = UTF-8 Unicode

\documentclass[t]{beamer}

%%%% HANDOUTS For online Uncomment the following four lines for handout
%\documentclass[t,handout]{beamer}  %Use this for handouts.
%\includeonlylecture{student}
%\usepackage{handoutWithNotes}
%\pgfpagesuselayout{3 on 1 with notes}[letterpaper,border shrink=5mm]
%	\setbeamercolor{background canvas}{bg=black!5}


%%% Including only some slides for students.
%%% Uncomment the following line. For the slides,
%%% use the labels shown below the command.

%% For students, use \lecture{student}{student}
%% For mine, use \lecture{instructor}{instructor}


%\usepackage{pgf,pgfpages}
%\pgfpagesuselayout{4 on 1}[letterpaper,border shrink=5mm]

% FONTS
\usepackage{fontspec}
\def\mainfont{Linux Biolinum O}
\setmainfont[Ligatures=TeX, Contextuals={NoAlternate}, BoldFont={* Bold}, ItalicFont={* Italic}, Numbers={Proportional}]{\mainfont}
%\setmonofont[Scale=MatchLowercase]{Inconsolata} 
\setsansfont[Scale=MatchLowercase]{Linux Biolinum O} 
\usepackage{microtype}

\usepackage{graphicx}
	\graphicspath{%
	{/Users/goby/Pictures/teach/466/lectures/}%
	{img/}%
	{/Users/goby/Pictures/teach/common/}} % set of paths to search for images

\usepackage{amsmath,amssymb}

%\usepackage{units}

\usepackage{booktabs}
\usepackage{multicol}
%	\setlength{\columnsep=1em}

\usepackage{textcomp}
\usepackage{setspace}
\usepackage{tikz}
	\tikzstyle{every picture}+=[remember picture,overlay]

\mode<presentation>
{
  \usetheme{Lecture}
  \setbeamercovered{invisible}
  \setbeamertemplate{items}[square]
}

\usepackage{calc}
\usepackage{hyperref}

\newcommand\HiddenWord[1]{%
	\alt<handout>{\rule{\widthof{#1}}{\fboxrule}}{#1}%
}

\newcommand\imagecredit[1]{%
	\vskip0pt plus 1filll \tiny #1}%
	

\begin{document}


\lecture{student}{student}
{
\usebackgroundtemplate{\includegraphics[width=\paperwidth]{horned_lark} }
\begin{frame}[t,plain]{}
\imagecredit{Kenneth Cole Schneider, Flickr Creative Commons.}
\end{frame}
}

{
\usebackgroundtemplate{\includegraphics[width=\paperwidth]{population_flock} }
\begin{frame}[t,plain]{\textcolor{white}{What are the two models of population growth?}}
\imagecredit{\hfill\textcolor{white}{Photographer unknown, Wikimedia Commons.}}
\end{frame}
}

{
\usebackgroundtemplate{\includegraphics[width=\paperwidth]{house_finch} }
\begin{frame}[t,plain]{}
\imagecredit{\hfill John Benson, Wikimedia Commons.}
\end{frame}
}

\begin{frame}[t,plain]{Exponential population growth for House Finches.}
	\begin{center}
		\includegraphics[height=0.8\textheight]{house_finch_population_growth}
	\end{center}	

	\imagecredit{\hfill Fig. 18--2 of textbook.}
\end{frame}


{
\usebackgroundtemplate{\includegraphics[width=\paperwidth]{monk_parakeet_chicago} }
\begin{frame}[t,plain]{}
\imagecredit{\textcolor{white}{John W. Iwanski, Flickr Creative Commons.}}
\end{frame}
}

\begin{frame}[t,plain]{Exponential population growth for Monk Parakeets.}
	\begin{center}
		\includegraphics[height=0.8\textheight]{monk_parakeet_population_growth}
	\end{center}	

	\imagecredit{\hfill Pruett-Jones and Tarvin 1998, Proc. 18th Vertebr. Pest Conf. Paper 67.}
\end{frame}

{
\usebackgroundtemplate{\includegraphics[width=\paperwidth]{cattle_egret} }
\begin{frame}[t,plain]{}
\imagecredit{\textcolor{white}{Changhua Coast Conservation Action, Flickr Creative Commons.}}
\end{frame}
}

\begin{frame}[t,plain]{Cattle Egret had logistic population growth.}
	\begin{center}
		\includegraphics[width=1\textwidth]{cattle_egret_population_growth}
	\end{center}	

	\imagecredit{\hfill Fig. 18--3 of textbook.}
\end{frame}

\begin{frame}[t,plain]{Take a moment to answer these two questions. }

	\hangpara What is density-dependent population regulation?
	
	\hangpara What is negative feedback?
	
\end{frame}

{
\usebackgroundtemplate{\includegraphics[width=\paperwidth]{density_dependent_song_sparrow} }
\begin{frame}[t,plain]{}
\imagecredit{\textcolor{white}{Keith, Wikimedia Commons.}}
\end{frame}
}

{
\usebackgroundtemplate{\includegraphics[width=\paperwidth]{density_dependent_sparrow_nest} }
\begin{frame}[t,plain]{}
\imagecredit{\textcolor{white}{K.P. McFarland, Flickr Creative Commons.}}
\end{frame}
}


\begin{frame}[t,plain]{How does clutch size change? Why?}
	\begin{center}
		\includegraphics[height=0.8\textheight]{density_dependent_clutchsize}
	\end{center}	

	\imagecredit{Arcese and Smith 1988, J. Animal Ecology 57: 119.}
\end{frame}

\begin{frame}[t,plain]{How does nest failure change? Why?}
	\begin{center}
		\includegraphics[height=0.8\textheight]{density_dependent_failure}
	\end{center}	

	\imagecredit{Arcese and Smith 1988, J. Animal Ecology 57: 119.}
\end{frame}

{
\usebackgroundtemplate{\includegraphics[width=\paperwidth]{BTBW} }
\begin{frame}[t,plain]{}
	\imagecredit{\hfill\textcolor{white}{cuatrok77, Wikimedia Commons.}}
\end{frame}
}

\begin{frame}[t,plain]{\emph{r} decreases when \emph{N} increases.}
	\centering
		\includegraphics[height=0.85\textheight]{BTBW_population_size2}

	\imagecredit{\hfill Fig. 18--13 of textbook.}
\end{frame}

\begin{frame}[t,plain]{Fledging success decreases with increasing density.}
	\begin{center}
		\includegraphics[height=0.7\textheight]{BTBW_fledgling_success}
	\end{center}	

	\imagecredit{\hfill Fig. 18--13 of textbook.}
\end{frame}

{
\usebackgroundtemplate{\includegraphics[width=\paperwidth]{BTBW_migration_mortality} }
\begin{frame}[t,plain]{}
\imagecredit{\hfill\colorbox{black}{\color{white} cuatrok77, Wikimedia Commons.}}
\end{frame}
}


{
\usebackgroundtemplate{\includegraphics[width=\paperwidth]{density_dependent_territory} }
\begin{frame}[t,plain]{}
\imagecredit{\hfill\colorbox{black}{\color{white} Foliash, Wikimedia Commons.}}
\end{frame}
}

\begin{frame}[t,plain]{Disease can cause local population decline.}
	\centering
		\includegraphics[width=1\textwidth]{population_decline_house_finch}

	\imagecredit{\hfill Figure 18--9 of textbook.}
\end{frame}

{
\usebackgroundtemplate{\includegraphics[width=\paperwidth]{h5n1_genealogy} }
\begin{frame}[t,plain]{}
\imagecredit{\hfill Li et al. 2014. Emerging Infectious Diseases 20: 1287.}
\end{frame}
}

{
\usebackgroundtemplate{\includegraphics[width=\paperwidth]{h5n1_spread} }
\begin{frame}[t,plain]{Siberia is a critical “hub” for the spread of H5N1.}
\imagecredit{\hfill Li et al. 2014. Emerging Infectious Diseases 20: 1287.}
\end{frame}
}

{
\usebackgroundtemplate{\includegraphics[width=\paperwidth]{h5n1_flyways} }
\begin{frame}[t,plain]{Most eastern hemisphere flyways overlap in Siberia.}
\imagecredit{Li et al. 2014. Emerging Infectious Diseases 20: 1287.}
\end{frame}
}

{
\usebackgroundtemplate{\includegraphics[width=\paperwidth]{migrating_duck_flock} }
\begin{frame}[t,plain]{}
\imagecredit{\textcolor{white}{Gary Kramer, USFWS, Public Domain.}}
\end{frame}
}

{
\begin{frame}[t,plain]{Some ducks have the highest incidence of avian flu.}
	\includegraphics[height=0.85\textheight]{h5n1_anatidae}
	
	\imagecredit{\hfill Olsen et al. 2006. Science 312: 384.}
\end{frame}
}

{
\begin{frame}[t,plain]{Some migratory birds use Nearctic and Palearctic flyways.}
	\includegraphics[width=1\textwidth]{h5n1_shared_flyways}
	
	\imagecredit{\hfill Peterson et al. 2007. PLoS ONE 2: e261.}
\end{frame}
}

{
\usebackgroundtemplate{\includegraphics[width=\paperwidth]{eastern_meadowlark}}
\begin{frame}[t,plain]{}
    	\imagecredit{\colorbox{white!80!gray}{Kenneth Cole Schneider, Flickr Creative Commons.}}
\end{frame}
}

{
\usebackgroundtemplate{\includegraphics[width=\paperwidth]{grasshopper_sparrow}}
\begin{frame}[t,plain]{}
	\imagecredit{\hfill Kenneth Cole Schneider, Flickr Creative Commons.}
\end{frame}
}

{
\usebackgroundtemplate{\includegraphics[width=\paperwidth]{dickcissel}}
\begin{frame}[t,plain]{}
	\imagecredit{\textcolor{white}{US Fish \& Wildlife Service, Public Domain.}}
\end{frame}
}

{
\usebackgroundtemplate{\includegraphics[width=\paperwidth]{henslows_sparrow}}
\begin{frame}[t,plain]{}
	\imagecredit{\hfill Amy Evenstad, Flickr Creative Commons.}
\end{frame}
}

{
\usebackgroundtemplate{\includegraphics[width=\paperwidth]{bobolink}}
\begin{frame}[t,plain]{}
	\imagecredit{\hfill Amy Evenstad, Flickr Creative Commons.}
\end{frame}
}

{
\begin{frame}[t,plain]{What do these bird species have in common?}

	\hangpara{Eastern Meadowlark}

	\hangpara{Grasshopper Sparrow}
	
	\hangpara{Dickcissel}
	
	\hangpara{Henslow's Sparrow}
	
	\hangpara{Bobolink}
	
\end{frame}
}

{
\begin{frame}[t,plain]{Grassland birds have declined rapidly.  Why?}
	\begin{center}
		\includegraphics[width=1\textwidth]{grassland_bird_decline}
	\end{center}

	\imagecredit{\hfill Gregory et al. 2005. Phil. Trans. Roy. Soc. B. 360: 269. See also Fig. 21--2 of textbook.}
\end{frame}
}

{
\usebackgroundtemplate{\includegraphics[width=\paperwidth]{midwest_prairie}}
\begin{frame}[t,plain]{The Midwest was once covered by prairies.}
	\imagecredit{\hfill\textcolor{white}{US Fish \& Wildlife Service, Public Domain}}
\end{frame}
}

{
\usebackgroundtemplate{\includegraphics[width=\paperwidth]{taberville_prairie_ca}}
\begin{frame}[t,plain]{\textcolor{white}{Taberville Prairie is a fragment of tallgrass prairie surrounded by agricultural land.}}
\end{frame}
}

{
\begin{frame}[t,plain]{Grassland bird richness increases with habitat area.}
	\centering
		\includegraphics[width=0.85\textwidth]{grassland_fragment_size}
%	\end{center}

	\imagecredit{\hfill Herkert 1994. Ecol. Appl. 4: 461.}
\end{frame}
}

{
\usebackgroundtemplate{\includegraphics[width=\paperwidth]{controlled_burn}}
\begin{frame}[t,plain]{What is the purpose of controlled burning?}
	\imagecredit{\hfill\textcolor{white}{Erik, Flickr Creative Commons.}}
\end{frame}
}

{
\begin{frame}[t,plain]{How does density of dead and live matter respond?}
	\begin{center}
		\includegraphics[width=0.9\textwidth]{grassland_burn_habitat_diversity2}
	\end{center}

	\imagecredit{\hfill Hovick et al. 2014. Ecosphere 5: Article 62.}
\end{frame}
}

{
\begin{frame}[t,plain]{How do vegetation types respond to controlled burns?}
	\begin{center}
		\includegraphics[width=0.9\textwidth]{grassland_burn_habitat_diversity1}
	\end{center}

	\imagecredit{\hfill Hovick et al. 2014. Ecosphere 5: Article 62.}
\end{frame}
}

{
\begin{frame}[t,plain]{How do individual species respond?}
	\begin{center}
		\includegraphics[width=0.9\textwidth]{grassland_burn_bird_response1}
	\end{center}

	\imagecredit{\hfill Hovick et al. 2014. Ecosphere 5: Article 62.}
\end{frame}
}

{
\begin{frame}[t,plain]{Which post-burn stage supports the most bird diversity?}
	\begin{center}
		\includegraphics[width=0.82\textwidth]{grassland_burn_bird_response2}
	\end{center}

	\imagecredit{\hfill Hovick et al. 2014. Ecosphere 5: Article 62.}
\end{frame}
}

{
\begin{frame}[t,plain]{Population size of the Grasshopper Sparrow is decreasing.}
	\centering
		\includegraphics[height=0.85\textheight]{GRSP_population_decline}

	\imagecredit{\hfill Figure 21--6 of textbook.}
\end{frame}
}

{
\begin{frame}[t,plain]{Grasshopper Sparrow populations respond to management.}
	\centering
		\includegraphics[width=0.9\linewidth]{GRSP_population_recovery}

	\imagecredit{\hfill Figure 21--6 of textbook.}
\end{frame}
}


{
\usebackgroundtemplate{\includegraphics[width=\paperwidth]{iiwi_extinction}}
\begin{frame}[t,plain]{}
	\imagecredit{\hfill\textcolor{white}{Norman Kaleomokuokanalu Chock, Flickr Creative Commons.}}
\end{frame}
}

{
\usebackgroundtemplate{\includegraphics[width=\paperwidth]{hawaii_endangered}}
\begin{frame}[t,plain]{}
	\imagecredit{\hfill Brenda Zaun, US Fish \& Wildlife Service, Flickr Creative Commons.}
\end{frame}
}

{
\begin{frame}[t,plain]{Habitat loss is the greatest risk for threatened species.}
	\begin{center}
		\includegraphics[width=1\textwidth]{extinction_risk}
	\end{center}
	\imagecredit{\hfill Owens and Bennett, 2000. Proc. Natl. Acad. Sci. USA 97: 12144}
\end{frame}
}

{
\begin{frame}[t,plain]{Habitat loss affects different types of species.}
	\begin{center}
		\includegraphics[width=1\textwidth]{extinction_risk_habitat_loss}
	\end{center}

	\imagecredit{\hfill Owens and Bennett, 2000. Proc. Natl. Acad. Sci. USA 97: 12144}
\end{frame}
}

{
\begin{frame}[t,plain]{Persecution affects different types of species.}
	\begin{center}
		\includegraphics[width=1\textwidth]{extinction_risk_persecution}
	\end{center}
	\imagecredit{\hfill Owens and Bennett, 2000. Proc. Natl. Acad. Sci. USA 97: 12144}
\end{frame}
}



\end{document}
