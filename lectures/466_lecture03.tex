%!TEX TS-program = lualatex
%!TEX encoding = UTF-8 Unicode

\documentclass[t]{beamer}

%%%% HANDOUTS For online Uncomment the following four lines for handout
%\documentclass[t,handout]{beamer}  %Use this for handouts.
%\includeonlylecture{student}
%\usepackage{handoutWithNotes}
%\pgfpagesuselayout{3 on 1 with notes}[letterpaper,border shrink=5mm]
%	\setbeamercolor{background canvas}{bg=black!5}


%%% Including only some slides for students.
%%% Uncomment the following line. For the slides,
%%% use the labels shown below the command.

%% For students, use \lecture{student}{student}
%% For mine, use \lecture{instructor}{instructor}

% FONTS
\usepackage{fontspec}
\def\mainfont{Linux Biolinum O}
\setmainfont[Ligatures=TeX, Contextuals={NoAlternate}, BoldFont={* Bold}, ItalicFont={* Italic}, Numbers={Proportional, OldStyle}]{\mainfont}
\setsansfont[Scale=MatchLowercase]{Linux Biolinum O} 
\usepackage{microtype}

\usepackage{graphicx}
	\graphicspath{%
	{/Users/goby/Pictures/teach/466/lectures/}}%
%	{/Users/goby/Pictures/teach/common/}} % set of paths to search for images

\usepackage{amsmath,amssymb}

%\usepackage{units}

\usepackage{booktabs}
\usepackage{multicol}
%	\setlength{\columnsep=1em}

%\usepackage{textcomp}
%\usepackage{setspace}
\usepackage{tikz}
	\tikzstyle{every picture}+=[remember picture,overlay]

\mode<presentation>
{
  \usetheme{Lecture}
  \setbeamercovered{invisible}
  \setbeamertemplate{items}[square]
}

%\usepackage{calc}
\usepackage{hyperref}

\newcommand\HiddenWord[1]{%
	\alt<handout>{\rule{\widthof{#1}}{\fboxrule}}{#1}%
}


\begin{document}
%\lecture{instructor}{instructor}

\lecture{student}{student}
{
\usebackgroundtemplate{\includegraphics[width=\paperwidth]{anatomy_intro}}
\begin{frame}[b,plain]{Anatomy and physiology}
	\tiny \href{https://www.flickr.com/photos/148768555@N05/31605481882}{Red Knot by \textsc{grid-a}rendal, Flickr Creative Commons, \ccbyncsa{2}}
\end{frame}
}

%% Skeletal fusion
\begin{frame}[t,plain]{Fusion of skeletal structures increases strength and rigidity for flight.}

	\vspace{-0.5\baselineskip}
	\centering 
	\includegraphics[height=0.8\textheight]{anatomy_skeleton_fused}
	
	\vfilll
	
	\tiny \hfill Lovette and Fitzpatrick, \textit{Handbook of Bird Biology} \textcopyright\,Cornell University
\end{frame}

%% Carina
\begin{frame}[t,plain]{The \highlight{carina} provides large surface area for muscle attachment.}

	\vspace{-0.5\baselineskip}
	\centering 
	\includegraphics[height=0.8\textheight]{anatomy_carina}
	
	\vfilll
	
	\tiny \hfill Lovette and Fitzpatrick, \textit{Handbook of Bird Biology} \textcopyright\,Cornell University
\end{frame}

%% Flight muscles
\begin{frame}[t,plain]{\highlight{Pectoralis} powers the downstroke. \highlight{Supracoracoideus} powers the upstroke.}

	\vspace{-0.5\baselineskip}
	\centering 
	\includegraphics[width=\linewidth]{anatomy_flight_muscles}
	
	\vfilll
	
	\tiny \hfill Lovette and Fitzpatrick, \textit{Handbook of Bird Biology} \textcopyright\,Cornell University
\end{frame}



%Pneumatic bone
\begin{frame}[t,plain]{Flying birds have strong, hollow  \highlight{pneumatic bone.}}

	\includegraphics[width=\linewidth]{anatomy_pneumatic_bone}
	
	\hangpara The air spaces connect directly to lungs or air sacs.
	
	\vfilll
	
	\tiny \hfill Lovette and Fitzpatrick, \textit{Handbook of Bird Biology} \textcopyright\,Cornell University
\end{frame}


{
\begin{frame}[t,plain]{Nine air sacs facilitate air flow through the lungs.}

	\vspace{-0.5\baselineskip}
	
	\centering
	
	\includegraphics[height=0.87\textheight]{anatomy_air_sacs}
	
	\vfilll
	
	\tiny \hfill Lovette and Fitzpatrick, \textit{Handbook of Bird Biology} \textcopyright\,Cornell University
\end{frame}
}
%
%inhalation exhalation
\begin{frame}[t,plain]{A complete respiratory cycle has two inhalations and two exhalations.}

	\includegraphics[width=\linewidth]{anatomy_inhalation_exhalation}
		
	\vfilll
	
	\tiny \href{https://www.youtube.com/watch?v=kWMmyVu1ueY}{Video} \hfill Lovette and Fitzpatrick, \textit{Handbook of Bird Biology} \textcopyright\,Cornell University
\end{frame}
%%
\begin{frame}[t,plain]{Each \highlight{parabronchus} has air capillaries that align perpendicular to blood capillaries.}

	\includegraphics[width=\linewidth]{anatomy_parabronchus_flow}
		
	\vfilll
	
	\tiny \hfill Lovette and Fitzpatrick, \textit{Handbook of Bird Biology} \textcopyright\,Cornell University

\end{frame}

%% Cross-current exchange
\begin{frame}[t,plain]{\highlight{Cross-current exchange} increases efficiency of passive air diffusion into blood.}

	\centering
	
	\includegraphics[width=0.9\linewidth]{anatomy_cross_current_exchange}
		
	\vfilll
	
	\tiny \hfill Diagram by Dustin Siegel. Used with permission.
\end{frame}


%% Crop and Gizzard
\begin{frame}[t]{The \highlight{crop} acts as a holding chamber. The \highlight{gizzard} grinds and softens food.}

	\vspace{-\baselineskip}
	
	\centering
	
	\includegraphics[height=0.82\textheight]{anatomy_crop_gizzard}
		
	\vfilll
	
	\tiny \hfill Lovette and Fitzpatrick, \textit{Handbook of Bird Biology} \textcopyright\,Cornell University

\end{frame}

%%Brain and skull
\begin{frame}[t,plain]{The brain is restricted to the posterior skull.}

	
	\includegraphics[width=\linewidth]{anatomy_brain_skull}
		
	\vfilll
	
	\tiny \hfill Lovette and Fitzpatrick, \textit{Handbook of Bird Biology} \textcopyright\,Cornell University

\end{frame}

%%Brain size
\begin{frame}[t,plain]{Brain size is comparable to mammals of similar size.\newline Intelligent birds have larger brains.}

	
	\includegraphics[width=\linewidth]{anatomy_brain_size}
		
	\vfilll
	
	\tiny \hfill Lovette and Fitzpatrick, \textit{Handbook of Bird Biology} \textcopyright\,Cornell University

\end{frame}

%% Eye position

{
\usebackgroundtemplate{\includegraphics[width=\paperwidth]{anatomy_eye_position}}
\begin{frame}[b,plain]{\textcolor{white}{Eye position determines field of view.}}
	\tiny
	
	\textcolor{white}{Pigeon photo, Rex Shutterstock, fair use.\hfill \href{https://www.flickr.com/photos/60401146@N04/9271244216}{Hawk by \textsc{wcvm} Today, Flickr Creative Commons, \ccbyncsa{2}}}
\end{frame}
}

%% Field of view

%%Brain size
\begin{frame}[t,plain]{Forward-facing eyes creates binocular vision.}

	\vspace{-\baselineskip}
	
	\begin{multicols}{2}
	
	\centering
	\includegraphics[width=0.9\linewidth]{anatomy_view_field_2d} 
	
	\columnbreak
	
	\includegraphics[width=0.9\linewidth]{anatomy_view_field_3d}
	
	\end{multicols}

	\vfilll
	
	\tiny \hfill Lovette and Fitzpatrick, \textit{Handbook of Bird Biology} \textcopyright\,Cornell University

\end{frame}

%%Brain size
\begin{frame}[t,plain]{Cocking and bobbing heads simulate stereo vision.}

	\vspace{-0.5\baselineskip}
	
	\centering
	\includegraphics[height=0.85\textheight]{anatomy_cocked_head}
		
	\vfilll
	
	\tiny \href{https://www.youtube.com/watch?v=bG139rclp_E}{Pigeon walk} |  \href{https://www.youtube.com/watch?v=9WV5DfDeCZI}{Slo-mo} \hfill \href{https://www.youtube.com/watch?v=BsNjIuDpJjM&t=30s}{American Coots swimming} \hfill European Robin by Francis C. Franklin, \href{https://en.wikipedia.org/wiki/File:Erithacus_rubecula_with_cocked_head.jpg}{Wikimedia Commons}, \ccbysa{3}
\end{frame}



%%thermoregulation
\begin{frame}[t,plain]{Metabolism outside the \highlight{thermal neutral zone} requires energy.}

	\vspace{-\baselineskip}
	
	\begin{multicols}{2}
	
	\centering
	\includegraphics[width=0.95\linewidth]{anatomy_thermal_neutral_zone} 
	
	\columnbreak
	
	\includegraphics[width=0.95\linewidth]{anatomy_tuvu1_3ca}
	
	\end{multicols}

	\vfilll
	
	\tiny Lovette and Fitzpatrick,Cornell University \hfill \href{https://www.youtube.com/watch?v=NggVoJ7mOos}{Gular Fluttering} \hfill Turkey Vulture by Michael S. Taylor, \ccbyncsa{3}

\end{frame}

%% counter-current exchange
\begin{frame}[t,plain]{Counter-current heat exchange retains heat in the body.}
	
	\centering
	\includegraphics[width=\linewidth]{anatomy_counter_current_exchange} 

	\vfilll
	
	\tiny \hfill Lovette and Fitzpatrick, \textit{Handbook of Bird Biology} \textcopyright\,Cornell University

\end{frame}

%% Metabolism
%%thermoregulation
\begin{frame}[t,plain]{Large birds need more O\textsubscript{2} but small birds need it faster.}

	\vspace{-\baselineskip}
	
	\begin{multicols}{2}
	
	\centering
	\includegraphics[width=0.95\linewidth]{anatomy_field_metabolism} 
	
	\columnbreak
	
	\includegraphics[width=0.95\linewidth]{anatomy_mass_specific_metabolism}
	
	\end{multicols}

	\vfilll
	
	\tiny \hfill Lovette and Fitzpatrick, \textit{Handbook of Bird Biology} \textcopyright\,Cornell University

\end{frame}

%% Fats and sugars
{
\usebackgroundtemplate{\includegraphics[width=\paperwidth]{anatomy_oriole}}
\begin{frame}[b,plain]{\textcolor{white}{Fats and carbs are main energy sources.}}
	\tiny \hfill\textcolor{white}{Baltimore Oriole by Michael S.~Taylor, \ccbyncsa{3}}
\end{frame}
}



\end{document}

