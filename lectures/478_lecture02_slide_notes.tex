%!TEX TS-program = lualatex
%!TEX encoding = UTF-8 Unicode

\documentclass[11pt]{article}
%\usepackage{graphicx}
%	\graphicspath{{/Users/goby/Pictures/teach/153/lab/}} % set of paths to search for images

\usepackage{geometry}
\geometry{letterpaper}                   
\geometry{bottom=1in}
%\geometry{landscape}                % Activate for for rotated page geometry
%\usepackage[parfill]{parskip}    % Activate to begin paragraphs with an empty line rather than an indent
%\usepackage{amssymb}
%\usepackage{mathtools}
%	\everymath{\displaystyle}

%\pagenumbering{gobble}

\usepackage{fontspec}
\setmainfont[Ligatures={Common}, BoldFont={* Bold}, ItalicFont={* Italic}, Numbers={Proportional}]{Linux Libertine O}
\setsansfont[Scale=MatchLowercase,Ligatures=TeX]{Linux Biolinum O}
\setmonofont[Scale=MatchLowercase]{Inconsolata}
\usepackage{microtype}

\usepackage{unicode-math}
\setmathfont[Scale=MatchLowercase]{Asana-Math.otf}
%\setmathfont{XITS Math}

% To define fonts for particular uses within a document. For example, 
% This sets the Libertine font to use tabular number format for tables.
%\newfontfamily{\tablenumbers}[Numbers={Monospaced}]{Linux Libertine O}
%\newfontfamily{\libertinedisplay}{Linux Libertine Display O}


%\usepackage{booktabs}
%\usepackage{tabularx}
%\usepackage{longtable}
%\usepackage{siunitx}
%\usepackage[justification=raggedright, singlelinecheck=off]{caption}
%\captionsetup{labelsep=period} % Removes colon following figure / table number.
%\captionsetup{tablewithin=none}  % Sequential numbering of tables and figures instead of
%\captionsetup{figurewithin=none} % resetting numbers within each chapter (Intro, M&M, etc.)
%\captionsetup[table]{skip=0pt}

%\usepackage{array}
%\newcolumntype{L}[1]{>{\raggedright\let\newline\\\arraybackslash\hspace{0pt}}p{#1}}
%\newcolumntype{C}[1]{>{\centering\let\newline\\\arraybackslash\hspace{0pt}}p{#1}}
%\newcolumntype{R}[1]{>{\raggedleft\let\newline\\\arraybackslash\hspace{0pt}}p{#1}}

\usepackage{enumitem}
\usepackage{hyperref}
%\usepackage{placeins} %PRovides \FloatBarrier to flush all floats before a certain point.
\usepackage{hanging}

\usepackage{titling}
\setlength{\droptitle}{-60pt}
\posttitle{\par\end{center}}
\predate{}\postdate{}

%\usepackage{fancyhdr}
%\fancyhf{}
%\pagestyle{fancy}
%\lhead{}
%\chead{}
%\rhead{Name: \rule{5cm}{0.4pt} }
%\renewcommand{\headrulewidth}{0pt}

\newcommand{\VSpace}{\vspace{\baselineskip}}

\title{Instructor Notes}
\author{ZO 478 Lecture 02: Form and Function}
\date{}                                           % Activate to display a given date or no date

\begin{document}
\maketitle
%\thispagestyle{fancy}

These notes accompany the slides for ZO 478 Lecture 02 on form and function. These are key points to be sure to make when discussing each form/function type.

\subsection*{Rover-Predator}

    \begin{itemize}
    	\item fusiform body
    	\item terminal mouth
    	\item narrow peduncle, isocercal to lunate tail
    	\item fins provide balance and maneuverability 
    \end{itemize}

\subsection*{Lie in Wait predator}

    \begin{itemize}
    	\item strongly fusiform body
    	\item large terminal mouth with many teeth
    	\item large caudal fin. Median fins set well back near caudal.
    	\item cryptically colored 
    \end{itemize}

\subsection*{Surface oriented}

    \begin{itemize}
    	\item Flattened head with eyes on top or directed upward.  
    	\item Superior mouth. 
    	\item Surface feeder.
    	\item Live in oxygen-rich environment at the air-water interface.
    \end{itemize}


\subsection*{Deep-Bodied}

    \begin{itemize}
    	\item Compressiform body
    	\item Body depth is $>\frac{1}{3}$ body length  
    	\item Flexibility and maneuverability
    	\item Terminal mouth is often protrusible
    \end{itemize}

\subsection*{Eel-like}
    \begin{itemize}
    	\item Blunt to pointed head, with terminal mouth
    	\item Pelvic and pectoral fins reduced or absent.
    	\item Dorsal and anal fins long, often confluent with caudal.
    	\item bottom associated
    \end{itemize}

\subsection*{Bottom Rover}
	\begin{itemize}
		\item Ventrally flattend.
		\item Mouth terminal to inferior.
		\item Prey on benthic organisms.
		\item large pectoral fins for maneuvering.
	\end{itemize}

\subsection*{Bottom clingers and hiders.}

	\begin{itemize}
		\item small, cryptic color to blend with habitat.
		\item Clingers typically have fused pelvic fins
		\item Hiders down in crevices
	\end{itemize}

\subsection*{Flatfish}

	\begin{itemize}
		\item Dorsoventrally flattened (Still compressiform)
		\item eyes on top of head
		\item inferior mouth, sometimes terminal (flounder)
		\item Cryptic coloration
	\end{itemize}

\subsection*{Rattails.}

	\begin{itemize}
		\item Large head, often large eyes
		\item tapered tail
		\item large pectoral fins for maneuverability
		\item deep-sea
	\end{itemize}

\end{document}  