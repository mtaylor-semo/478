%!TEX TS-program = lualatex
%!TEX encoding = UTF-8 Unicode

\documentclass[nofonts, letterpaper]{tufte-handout}

%\geometry{showframe} % display margins for debugging page layout

\usepackage{graphicx} % allow embedded images
  \setkeys{Gin}{width=\linewidth,totalheight=\textheight,keepaspectratio}
  \graphicspath{{img/}} % set of paths to search for images
  
\usepackage{fontspec}
  \setmainfont[Ligatures=TeX,Numbers={Proportional}]{Linux Libertine O}
  \setsansfont{Linux Biolinum O}
\usepackage{microtype}
\usepackage{enumitem}
\usepackage{multicol} % multiple column layout facilities
%\usepackage{hyperref}
%\usepackage{fancyvrb} % extended verbatim environments
%  \fvset{fontsize=\normalsize}% default font size for fancy-verbatim environments

% Change the header to shift the title to the left side of the page. 
% Replaced \quad with \hfill.  See \plaintitle in tufte-common.def
{\fancyhead[RE,RO]{\scshape{\newlinetospace{\plaintitle}}\hfill\thepage}}

\makeatletter
% Paragraph indentation and separation for normal text
\renewcommand{\@tufte@reset@par}{%
  \setlength{\RaggedRightParindent}{1.0pc}%
  \setlength{\JustifyingParindent}{1.0pc}%
  \setlength{\parindent}{1pc}%
  \setlength{\parskip}{0pt}%
}
\@tufte@reset@par

% Paragraph indentation and separation for marginal text
\renewcommand{\@tufte@margin@par}{%
  \setlength{\RaggedRightParindent}{0pt}%
  \setlength{\JustifyingParindent}{0.5pc}%
  \setlength{\parindent}{0.5pc}%
  \setlength{\parskip}{0pt}%
}

\makeatother

\title{Study Guide 04}
\author{Circulation}

\date{} % without \date command, current date is supplied

\begin{document}

\maketitle	% this prints the handout title, author, and date

%\printclassoptions

\section{Vocabulary}\marginnote{\textbf{Study:} Pgs. 45--47, 64--66.} 
\vspace{-1\baselineskip}
\begin{multicols}{2}
\textbf{From lecture} \\
sinus venosus \\
atrium \\
ventricle \\
conus arteriosus \\
bulbus arteriosus \\
ventral aorta \\
dorsal aorta \\
afferent branchial arteries \\
efferent branchial arteries \\
common cardinal vein \\
hemoglobin \\
tetrameric \\ \hspace{2ex}hemoglobin \\
oxygen dissociation curve \\
P$_{50}$ \\
Bohr effect$^1$ \\
Root effect$^1$ \\
\columnbreak\textbf{Anatomy from labs}\\
unpaired fins \\
paired fins \\
dorsal fin \\
spinous dorsal \\
soft dorsal \\
pectoral and pelvic fins \\
caudal fin \\
adipose fin \\
gills \\
operculum \\
lateral line system \\
placoid scale \\
ganoid scale \\
cycloid scale \\
ctenoid scale \\
ceratortrichia \\
lepidotrichia \\
spines and rays
\end{multicols}
\marginnote[-4\baselineskip]{$^1$ Remember ABCR: Affinity\,=\,Bohr, Capacity\,=\,Root.}
\section{Concepts}

\begin{enumerate}
	\item Identify and name external anatomical features. Recognize the four types of scales.

	\item Which chamber of the heart provides most acceleration of the blood? 

	\item What are the functions of the sinus venosus, and the conus / bulbus arteriosus?

	\item Describe and illustrate how the circulatory/respiratory system differs between lungfishes and teleost fishes.

	\item How is the heart of the lungfish structured to keep oxygenated and deoxygenated blood separate from each other?

	\item Interpret and draw blood oxygen equilibrium curves (oxygen affinity curves).  Compare oxygen affinity between an active fish in normoxic water vs a sluggish fish in hypoxic water.

	\item Describe the effects that pH and concentration of dissolved CO$_2$ have on the oxygen affinity of hemoglobin.

	\item Explain how the Bohr Effect and the oxygen affinity of hemoglobin influence the loading and unloading of oxygen at the gills and in body tissue.
	
	\item Be able to write and interpret the bicarbonate buffering equilibrium equation. Explain how this system (the equilibrium equation) contributes to the loading and unloading of oxygen at the gills and in body tissue.  
	
\end{enumerate}


\end{document}