%!TEX TS-program = lualatex
%!TEX encoding = UTF-8 Unicode

\documentclass[10pt]{article}  
\usepackage[left=0.75in,right=0.75in,top=1in,bottom=0.5in]{geometry} 
\geometry{letterpaper}                   		% 
\usepackage[parfill]{parskip}    		% Activate to begin paragraphs with an empty line rather than an indent
\setlength{\parindent}{0pt}

\usepackage{graphicx}
\graphicspath{%
	{/Users/goby/Pictures/teach/466/lab/}}
	
% FONTS
\usepackage{fontspec}
\def\mainfont{Linux Libertine O}
%\defaultfontfeatures{Mapping=tex} % converts LaTeX specials (``quotes'' --- dashes etc.) to unicode
\setmainfont[Ligatures={Common, TeX}, BoldFont={* Bold}, ItalicFont={* Italic}, Numbers={Proportional, OldStyle}]{\mainfont}
%\setmonofont[Scale=MatchLowercase]{Inconsolata} 
%\setsansfont[Scale=MatchLowercase]{Linux Biolinum O} 
\usepackage{microtype}



\usepackage[singlelinecheck=false]{caption}
\usepackage{array}
\newcolumntype{L}[1]{>{\raggedright\let\newline\\\arraybackslash\hspace{0pt}}m{#1}}
\newcolumntype{C}[1]{>{\centering\let\newline\\\arraybackslash\hspace{0pt}}p{#1}}
\newcolumntype{R}[1]{>{\raggedleft\let\newline\\\arraybackslash\hspace{0pt}}m{#1}}


%\pagenumbering{gobble}
%\usepackage{pdflscape}
\usepackage{longtable}
\usepackage[frenchlinks]{hyperref}
\hypersetup{
	frenchlinks=true,
	pdfborder={0 0 0}}
\usepackage{booktabs}
\usepackage{multicol}
%\usepackage{amssymb}
%\usepackage{enumitem}
%\setlist{noitemsep}
%\setlist[description]{style=multiline, leftmargin=1.25cm, parsep=1ex}

\usepackage[sc]{titlesec}

\usepackage{fancyhdr}
\fancyhf{}
\pagestyle{fancy}
\lhead{}
\chead{}
\rhead{\footnotesize pg.~\thepage }
\renewcommand{\headrulewidth}{0.4pt}

\fancypagestyle{plain}{%
	\fancyhf{}
	\lhead{\textsc{bi}~466/666: Ornithology}
	\rhead{Name: \enspace \makebox[2.5in]{\hrulefill}}
	\renewcommand{\headrulewidth}{0pt}
}


\begin{document}
\thispagestyle{plain}
%\begin{landscape}

\subsection*{Skeleton and external topology\footnote{Adapted from
		Pettingill, O.S., Jr. \textit{Ornithology in Laboratory and Field,} 
		5th ed.}}

	This exercise will help you learn skeletal elements specific 
	to birds. You will also learn the external topology of birds to 
	identify them accurately. 


\subsubsection*{Skeleton}

	Below is a labeled drawing of a pigeon skeleton. You are 
	responsible for knowing the following structures. Find 
	the structures on the drawing below and identify them 
	on one of the pigeon skeletons provided in lab.

\begin{multicols}{3}
	alular digit\\
	carina (extension of the sternum)\\
	carpometacarpus\\
	coracoid\\
	femur\\
	furcula (“wishbone”)\\
	hallux (toe I)\\
	humerous\\
	major and minor digits (wing)\\
	% mandible, lower\\
	% mandible, upper\\
	%pygostyle\\
	radius\\
	sternum\\
	tarsometatarsus\\
	tibiotarsus\\
	%tomium, upper and lower\\
	toes (toes II–IV)\\
	ulna
\end{multicols}


\begin{center}
	\includegraphics[width=0.8\linewidth]{skeleton_pigeon}
\end{center}

\newpage

Here are close-up illustrations of the wing and leg skeletons.  In the wing, the \textbf{manus} is composed of the major and minor digits, the carpometacarpus, and the alular digits. The \textbf{antebranchium} or forearm is composed of the radius and ulna, between the elbow and wrist. The \textbf{branchium} consists of the humerous and extends from the body to the elbow.

\includegraphics[width=0.94\linewidth]{skeleton_wing_pigeon}

\vfill

\begin{center}
	\includegraphics[width=0.7\linewidth]{skeleton_leg_pigeon}
\end{center}

\textsc{Note:} toes are numbered from the midline outward. For comparison, hold you arms down by your side, hand open, palms directed backwards. Your thumb would be the hallux (toe I), your index finger would be toe II, and so on.

\newpage

\subsubsection*{External topology}

Use the provided images or the bird topology section of your Sibley field guide (page xvii) to identify and label the following parts on the drawings below. You may color in the parts or use lines. If using lines, be sure you point clearly at the correct part. If you are identifying feathers of the wing (e.g., the lesser coverts) be sure you bracket the entire set of feathers so I can tell you identified them correctly. You may use sources other than your field guide if that helps you. 
 
You only need to label parts listed below. Be sure to learn the parts. You \emph{will} see them again on the lab practical. You may also have to know them for lecture exams. The diagrams below are of a House Sparrow but you must be able to recognize them on other birds, including photos of birds.
 
% Upload photos or scans of your drawings to the Lab~2 drop box by the due date. 

%\bigskip

\subsubsection*{Resting bird}

Label or color the following parts on a standing House Sparrow. Include a key for your colors, if you use them.

\begin{multicols}{3}
auriculars\\
belly\\
bill\\
breast\\
crown\\
flank\\
greater coverts\\
lore\\
malar\\
mantle\\
nape\\
primaries\\
primary coverts\\
rump (anterior to uppertail\newline
\phantom{M}coverts on back)\\
secondaries\\
side\\
supercilium\\
tail\\
tarsus\\
tertials\\
throat\\
undertail coverts\\
uppertail coverts
%vent\\
\end{multicols}

\begin{center}
\includegraphics[width=0.75\linewidth]{topology_sparrow_body_standing}
\end{center}

\newpage


\subsubsection*{Flying bird}

Label or color the following parts on a House Sparrow in flight. Include a key for your colors.


\begin{multicols}{3}
belly\\
bill\\
breast\\
crown\\
flank\\
greater coverts\\
mantle\\
neck\\
primaries\\
rectrices\\
secondaries\\
side\\
tail\\
tertials\\
throat\\
undertail coverts\\
underwing coverts
%vent\\
\end{multicols}

\vspace{\baselineskip}

\begin{center}
\includegraphics[width=0.75\linewidth]{topology_sparrow_body_flying}
\end{center}

\newpage

\subsubsection*{Wing surfaces}

Definitions to help you identify various feather groups. Refer to these definitions and you identify the different groups.

\begin{multicols}{2}
	
	\textsc{Flight feathers} form the main surface of the wing. They generate lift and  thrust.
	
	\textbf{Remiges (singular: remex):} the long, stiff pennaceous flight feathers of the wing.
	
	\textbf{Rectrices (singular: rectrix):} the stiff pennaceous feathers of the tail. Rectrices are paired and numbered from the midline outward.
	
	\textbf{Primaries:} Remiges attached to the manus. They are numbered from closest to the body outward to the wing tip.
	
	\textbf{Secondaries:}  Remiges attached to the antebranchium and wrist. You can usually feel the wrist joint in the wing to determine where the secondaries start. They are numbered from outward in towards body. So, P1 and S1 are adjacent.
	
	\textbf{Tertials or tertiaries:} The three innermost “secondaries.” They are not remiges, however. 
	
	\textbf{Alulars (alular quills):} Three to five feathers attached to the alula. The three feathers together are considered the alula.
	
	\columnbreak
	
	\textsc{Wing coverts} are contour feathers that overlay the remiges on the upper and lower surfaces of the wings. 
	
	\textbf{Greater Primary Coverts:} the feathers that overlay the base of the primaries. There is one primary covert for each primary remex. Greater primary coverts are present on the upper and lower surface of the wing. 
	
	\textbf{Median Primary Coverts:} smaller feathers that overlay the greater primary coverts. They are present on the upper and lower surface of the wing (some species lack median primary coverts on the upper surface). Many species have lesser primary coverts that overlay the median primary coverts. I will not ask you to know them.
	
	\textbf{Greater Seccondary Coverts} and \textbf{Median Secondary Coverts}: same as primary coverts but overlay the secondaries instead of the primaries. Median and lesser secondary coverts can be hard to distinguish from each other in many species so we will treat them only as distinct from the greater secondary coverts.
\end{multicols}

Label or color the following parts on the upper (top drawing) and lower (bottom drawing) wing surfaces of a House Sparrow. Include a key for your colors.


\begin{multicols}{3}
alulars (upper only)\\
greater primary coverts\\
greater secondary coverts\\
lesser secondary coverts\\
median secondary coverts\\
primaries\\
secondaries\\
tertials
%vent\\
\end{multicols}

\vspace{\baselineskip}

\begin{center}
%\includegraphics[width=0.7\linewidth]{topology_sparrow_wing_upper}
\includegraphics[width=0.49\linewidth]{topology_sparrow_wing_upper}
\hfill 
\includegraphics[width=0.49\linewidth]{topology_sparrow_wing_lower}
%\vspace{0.5in}

%\includegraphics[width=0.7\linewidth]{topology_sparrow_wing_lower}

\end{center}

%\newpage
%
%\subsubsection*{Lower wing surface}
%
%Label or color the following parts on the lower wing surface of a House Sparrow. Include a key for your colors.
%
%
%\begin{multicols}{3}
%greater primary coverts\\
%greater secondary coverts\\
%lesser secondary coverts\\
%median secondary coverts\\
%primaries\\
%secondaries\\
%tertials
%%vent\\
%\end{multicols}
%
%\begin{center}
%\end{center}
%

%\newpage




\newpage

Here is a blank skeleton you can use to practice identifying the skeletal structures you are required to know. In addition to individual elements, identify the manus and the antebranchium. Label the parts of the wing where the primaries, the secondaries, and the alula attach.

\vfill

\begin{center}
	\includegraphics[width=0.85\linewidth]{skeleton_pigeon_unlabeled}
	
\end{center}

\vfill

\end{document}  