%!TEX TS-program = lualatex
%!TEX encoding = UTF-8 Unicode

\documentclass[nofonts, letterpaper]{tufte-handout}

%\geometry{showframe} % display margins for debugging page layout

\usepackage{graphicx} % allow embedded images
  \setkeys{Gin}{width=\linewidth,totalheight=\textheight,keepaspectratio}
  \graphicspath{{img/}} % set of paths to search for images
  
\usepackage{fontspec}
  \setmainfont[Ligatures=TeX,Numbers={Proportional}]{Linux Libertine O}
  \setsansfont{Linux Biolinum O}
\usepackage{microtype}
\usepackage{enumitem}
\usepackage{multicol} % multiple column layout facilities
%\usepackage{hyperref}
%\usepackage{fancyvrb} % extended verbatim environments
%  \fvset{fontsize=\normalsize}% default font size for fancy-verbatim environments

% Change the header to shift the title to the left side of the page. 
% Replaced \quad with \hfill.  See \plaintitle in tufte-common.def
{\fancyhead[RE,RO]{\scshape{\newlinetospace{\plaintitle}}\hfill\thepage}}

\makeatletter
% Paragraph indentation and separation for normal text
\renewcommand{\@tufte@reset@par}{%
  \setlength{\RaggedRightParindent}{1.0pc}%
  \setlength{\JustifyingParindent}{1.0pc}%
  \setlength{\parindent}{1pc}%
  \setlength{\parskip}{0pt}%
}
\@tufte@reset@par

% Paragraph indentation and separation for marginal text
\renewcommand{\@tufte@margin@par}{%
  \setlength{\RaggedRightParindent}{0pt}%
  \setlength{\JustifyingParindent}{0.5pc}%
  \setlength{\parindent}{0.5pc}%
  \setlength{\parskip}{0pt}%
}

\makeatother

\title{Study Guide 09}
\author{Evolutionary History and Trends; Jawless Fishes}

\date{} % without \date command, current date is supplied

\begin{document}

\maketitle	% this prints the handout title, author, and date

%\printclassoptions

\section{Vocabulary}\marginnote{\textbf{Study:} Chapters 2, 11.} \marginnote{$^1$ Both Agnatha and Osteichthys are commonly seen and sometimes still used (incorrectly) in the literature.} \marginnote{† indicates extinct taxon.}
\vspace{-1\baselineskip}
\begin{multicols}{2}
monophyletic \\
paraphyletic \\
Cyclostomata \\
Chondrichthyes \\
Actinoptergyii \\
Sarcoptergyii \\
Agnatha$^1$ \\
Osteichthyes$^1$ \\
Conodonta† \\
Ostracoderms† \\
Myxini \\
Petromyzontida \\
paired species / satellite species 
\end{multicols}

\section{Concepts}

\begin{enumerate}
	\item Be able to recognize and distinguish classification groups that are monophyletic or not monophyletic.  Be able to explain why they are monophyletic or not.

	\item Are the fishes, as we currently recognize them, paraphyletic, monophyletic, or polyphyletic?  What about each of the classes? Which are and which, if any, are not?

	\item What characteristic(s) are shared among the jawless fishes?  When did they live? What major living and fossil groups are included within the jawless fishes?

	\item What is the argument against using Agnatha to classify jawless fishes?  Similarly, what is the argument against using Osteichthyes for jawed fishes with bony elements?

	\item Be able to draw a phylogeny with the correct placement of:
	\begin{itemize}
		\item The classes of extant fishes,
		\item Conodonta, and
		\item Ostracodermi.
	\end{itemize}
	We’ll add more taxa as we proceed.  Small steps.
	
	\item What geologic period is commonly known as the Age of Fishes?  Why is this so?

	\item Which fishes are craniates?  Which fishes are chordates?  What are the characteristics that define chordates, and what characteristics define craniates?  Be sure you know which is a phylum and which is a subphylum.
	
		\item What major characteristics define the Myxini?  What are two possible explanations for knotting behavior in the Myxini?

	\item What type of osmoregulation is characteristic of the Myxini?  What characteristic(s) of their habitat makes this type of osmoregulation possible?

	\item What are the major characteristics of Petromyzontida?

	\item Explain the paired species (also known as the satellite species) hypothesis in lampreys.  

\end{enumerate}

\end{document}