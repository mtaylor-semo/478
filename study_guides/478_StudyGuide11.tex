%!TEX TS-program = lualatex
%!TEX encoding = UTF-8 Unicode

\documentclass[letterpaper]{tufte-handout}

%\geometry{showframe} % display margins for debugging page layout

\usepackage{graphicx} % allow embedded images
  \setkeys{Gin}{width=\linewidth,totalheight=\textheight,keepaspectratio}
  \graphicspath{{img/}} % set of paths to search for images
  
\usepackage{fontspec}
  \setmainfont[Ligatures=TeX,Numbers={Proportional}]{Linux Libertine O}
  \setsansfont{Linux Biolinum O}
\usepackage{microtype}
\usepackage{enumitem}
\usepackage{multicol} % multiple column layout facilities
%\usepackage{hyperref}
%\usepackage{fancyvrb} % extended verbatim environments
%  \fvset{fontsize=\normalsize}% default font size for fancy-verbatim environments

% Change the header to shift the title to the left side of the page. 
% Replaced \quad with \hfill.  See \plaintitle in tufte-common.def
{\fancyhead[RE,RO]{\scshape{\newlinetospace{\plaintitle}}\hfill\thepage}}

\makeatletter
% Paragraph indentation and separation for normal text
\renewcommand{\@tufte@reset@par}{%
  \setlength{\RaggedRightParindent}{1.0pc}%
  \setlength{\JustifyingParindent}{1.0pc}%
  \setlength{\parindent}{1pc}%
  \setlength{\parskip}{0pt}%
}
\@tufte@reset@par

% Paragraph indentation and separation for marginal text
\renewcommand{\@tufte@margin@par}{%
  \setlength{\RaggedRightParindent}{0pt}%
  \setlength{\JustifyingParindent}{0.5pc}%
  \setlength{\parindent}{0.5pc}%
  \setlength{\parskip}{0pt}%
}
\makeatother

\title{Study Guide 11}
\author{Sarcoptergyii, Relict Actinopterygians}

\date{} % without \date command, current date is supplie

\begin{document}

\maketitle	% this prints the handout title, author, and date

%\printclassoptions

\section{Vocabulary}\marginnote{\textbf{Study:} pgs. 179--190, 241--255.}
\vspace{-1\baselineskip}
\begin{multicols}{2}
ceratotrichia \\
lepidotrichia \\
Sarcopterygii \\
Coelacanthomorpha \\
diphycercal tail \\
Dipnotetrapodomorpha \\
Actinopterygii \\
branchiostegal rays \\
Cladistia \\
ganoid scales \\
Chondrostei \\
Neoptergyii
\end{multicols}

\section{Concepts}

\begin{enumerate}
	\item What diagnostic features define the Sarcoptergyii?  What are the two subclasses?  Which subclass includes tetrapods?  (Write the subclasses out a gazillion times to get the spelling down!)

	\item What diagnostic features define the Coelacanthomorpha?  What other features are found in this subclass?

	\item What diagnostic features define the Dipnotetrapodomorpha?\sidenote{Your text uses Dipnoi instead of Dipnotetrapodomorpha. However, most authors use Dipnoi (or Dipnomorpha) to represent the lungfishes and Tetrapodomorpha to represent the tetrapods. We will use Dipnotetrapodomorpha but you should be familiar with these other groups.}  What other features are found in this subclass?

	\item What characteristics would you describe to a friend to show that tetrapods belong in the class Sarcopterygii?

	\item What are the diagnostic features of the Actinoptergyii?  What other characters are found in this group?  What are the three subclasses?  

	\item List and explain the evolutionary trends and adaptive advantages within the Actinopterygii for:
	\begin{itemize}
		\item scales
		\item branchiostegal rays
		\item swim bladder
		\item jaws
		\item tail
		\item fins
		\item spines
	\end{itemize}

	\item What are the diagnostic characteristics, if any, of each subclass of the Actinoptergyii?  Is each subclass monophyletic? 

	\item Which subclass contains nearly all of the species we recognize as fishes?

	\item Which subclass or subclasses are represented here in Missouri? Name a fish representative for each subclass that you listed.

\end{enumerate}

\end{document}