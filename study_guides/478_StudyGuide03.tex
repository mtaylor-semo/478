%!TEX TS-program = lualatex
%!TEX encoding = UTF-8 Unicode

\documentclass[nofonts, letterpaper]{tufte-handout}

%\geometry{showframe} % display margins for debugging page layout

\usepackage{graphicx} % allow embedded images
  \setkeys{Gin}{width=\linewidth,totalheight=\textheight,keepaspectratio}
  \graphicspath{{img/}} % set of paths to search for images
  
\usepackage{fontspec}
  \setmainfont[Ligatures=TeX,Numbers={Proportional}]{Linux Libertine O}
  \setsansfont{Linux Biolinum O}
\usepackage{microtype}
\usepackage{enumitem}
\usepackage{multicol} % multiple column layout facilities
%\usepackage{hyperref}
%\usepackage{fancyvrb} % extended verbatim environments
%  \fvset{fontsize=\normalsize}% default font size for fancy-verbatim environments

% Change the header to shift the title to the left side of the page. 
% Replaced \quad with \hfill.  See \plaintitle in tufte-common.def
{\fancyhead[RE,RO]{\scshape{\newlinetospace{\plaintitle}}\hfill\thepage}}

\makeatletter
% Paragraph indentation and separation for normal text
\renewcommand{\@tufte@reset@par}{%
  \setlength{\RaggedRightParindent}{1.0pc}%
  \setlength{\JustifyingParindent}{1.0pc}%
  \setlength{\parindent}{1pc}%
  \setlength{\parskip}{0pt}%
}
\@tufte@reset@par

% Paragraph indentation and separation for marginal text
\renewcommand{\@tufte@margin@par}{%
  \setlength{\RaggedRightParindent}{0pt}%
  \setlength{\JustifyingParindent}{0.5pc}%
  \setlength{\parindent}{0.5pc}%
  \setlength{\parskip}{0pt}%
}

\makeatother

\title{Study Guide 03}
\author{Locomotion and Respiration}

\date{} % without \date command, current date is supplied

\begin{document}

\maketitle	% this prints the handout title, author, and date

%\printclassoptions

\section{Vocabulary}\marginnote{\textbf{Study:} pgs. 113--119, 57--64.} 
\vspace{-1\baselineskip}
\begin{multicols}{2}
myomere\\
hypaxial\\
epaxial\\
undulation \\
oscillation \\
anguilliform swimming \\
subcarangiform swimming \\
carangiform swimming \\
ostraciform swimming \\
rajiform swimming \\
amiiform swimming  \\
labriform swimming \\
balistiform swimming \\
gymnotiform swimming \\
tetraodontform swimming \\
normoxic \\
hypoxic \\
anoxic \\
gill rakers \\
gill arch \\
gill filaments \\
lamellae (singular: lamella) \\
countercurrent exchange \\
buccal cavity \\
opercular cavity \\
ram ventilation \\
suprabranchial organ \\
labyrinth organ
\end{multicols}

\section{Concepts}

\begin{enumerate}
	\item Briefly explain the function of myomeres for swimming in fishes. Draw an illustration of a myomere in relation to the body of the fish. 
	
	\item Explain how myomere contraction and relaxation results in undulation of the body.
	
	\item Be able to differentiate each of the types of swimming listed above.  Which use undulation?  Which use oscillation? Relate this to the assigned reading (Webb, Form and Function) that I handed out in class.
	
	\item List the swimming types that used the median or paired fins. For each, state whether it is based on undulation or oscillation.
	
	\item List the swimming types that uses body or caudal fins. For each, state whether it is based on undulation or oscillation.

	\item How much dissolved oxygen (\%) can pure, cold freshwater hold?  What other factors influence how much oxygen can be dissolved in water.

	\item Describe the specific mechanisms in the gill that fishes use to extract as much oxygen as possible from the water.
	
	\item Describe the relationship between gill surface area and fish form and function. What mechanisms are used to increase the surface area of the gills?

	\item Describe alternative physical and physiological mechanisms that some fishes utilize to obtain oxygen in normal and low oxygen conditions.

	\item Describe and illustrate countercurrent exchange in the gills.  Explain how countercurrent exchange maximizes oxygen uptake.

	\item Describe the mechanics of gill ventilation (pumping water across the gills).

	\item \textbf{Thoughtful:} Despite the fact that O$_2$ has slightly lower solubility in saltwater than in freshwater, most of the air-breathing fishes are freshwater species.  Why do you think this is? 
\end{enumerate}


\end{document}