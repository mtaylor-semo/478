%!TEX TS-program = lualatex
%!TEX encoding = UTF-8 Unicode

\documentclass[letterpaper]{tufte-handout}

%\geometry{showframe} % display margins for debugging page layout

\usepackage{fontspec}
\def\mainfont{Linux Libertine O}
\setmainfont[Ligatures={Common,TeX}, Contextuals={NoAlternate}, BoldFont={* Bold}, ItalicFont={* Italic}, Numbers={OldStyle}]{\mainfont}
\setsansfont[Scale=MatchLowercase, Numbers={OldStyle}]{Linux Biolinum O} 
\setmonofont{Linux Libertine O}
\usepackage{microtype}

\usepackage{graphicx} % allow embedded images
  \setkeys{Gin}{width=\linewidth,totalheight=\textheight,keepaspectratio}
  \graphicspath{{/Users/goby/Documents/teach/434/lectures/}} % set of paths to search for images

\usepackage{amsmath}  % extended mathematics
\usepackage{booktabs} % book-quality tables
\usepackage{units}    % non-stacked fractions and better unit spacing
\usepackage{multicol} % multiple column layout facilities
%\usepackage{fancyvrb} % extended verbatim environments
%  \fvset{fontsize=\normalsize}% default font size for fancy-verbatim environments

\usepackage{enumitem}
\usepackage{mhchem}

\makeatletter
% Paragraph indentation and separation for normal text
\renewcommand{\@tufte@reset@par}{%
  \setlength{\RaggedRightParindent}{1.0pc}%
  \setlength{\JustifyingParindent}{1.0pc}%
  \setlength{\parindent}{1pc}%
  \setlength{\parskip}{0pt}%
}
\@tufte@reset@par

% Paragraph indentation and separation for marginal text
\renewcommand{\@tufte@margin@par}{%
  \setlength{\RaggedRightParindent}{0pt}%
  \setlength{\JustifyingParindent}{0.5pc}%
  \setlength{\parindent}{0.5pc}%
  \setlength{\parskip}{0pt}%
}
\makeatother

% Set up the spacing using fontspec features
   \renewcommand\allcapsspacing[1]{{\addfontfeatures{LetterSpace=15}#1}}
   \renewcommand\smallcapsspacing[1]{{\addfontfeatures{LetterSpace=10}#1}}

\title{{\scshape zo} 478/678 Study Guide 04}

\date{} % without \date command, current date is supplied

\begin{document}

\maketitle	% this prints the handout title, author, and date

%\printclassoptions

\section{Vocabulary}\marginnote{\textbf{Study:} Pages 71, 119--126, 436--437, 156, 456--475.} 
\vspace{-1\baselineskip}
\begin{multicols}{2}
detritivores \\
herbivores \\
carnivores \\
omnivores \\
euryphagous \\
stenophagous \\
monophagous \\
pharyngeal teeth \\
gill rakers \\
promiscuity \\
polygyny \\
polyandry \\
monogamy \\
spawning aggregation \\
semelparity \\
iteroparity \\
water-column (pelagic) spawning \\
substrate (benthic) spawning \\
paternity assurance \\
egg mimicry \\
egg raiding \\
allopaternal care \\
ovoviviparity \\
viviparity \\
egg dumping \\
sneaker male \\
satellite male \\
sequential hermaphroditism \\
protandry \\
protogyny \\
synchronous hermaphroditism 
\end{multicols}

\section{Concepts}

\begin{enumerate}

	\item What is the function of pharyngeal teeth?  Relate general size and shape of pharyngeal teeth to possible diet.

	\item What is the function of gill rakers?  How do they aid in feeding?  Relate general size, shape and numbers of gill rakers to possible diet.

	\item How does gut length relate to diet?

	\item If I give you a particular diet (e.g., monophagous for benthic snails), how much of the overall form and function of the fish could you explain to me?  Think about body shapes, fin shapes and position, type of swimming, mouth position, potential types of jaw teeth, gill rakers and pharyngeal teeth, and length of the gut.  In other words, tie it all together!

	\item Compare and contrast the four mating systems found in fishes (promiscuity, etc). Which types are commonly found among fishes? Which types are rare?
	
	\item What type of mating system is found in fishes that aggregate for spawning? Explain.
	
	\item Describe 3–4 mechanisms fishes may use to find a mate?

	\item Do fishes that spawn in aggregations have to “worry” about finding a mate? Why or why not?
	
%	\item What is the disadvantage of using bright coloration to attract a mate?

	\item What conditions favor semelparity over iteroparity?  Why is iteroparity most common in fishes?

	\item What is the primary difference between ovoviviparity and viviparity in fishes?

	\item Compare and contrast different types of parental care in fishes.
	
	\item Why are males more likely than females to provide parental care?

%	\item Why is male parental care more advantageous than female parental care?

	\item Describe three strategies employed by nest-guarding males to increase chances of mating with a female.
	
	\item Why might egg dumping be a mutualism between the signle nest-building species and the many other species that “dump” their eggs in the nest?

	\item What do most marine fishes have an extended larval stage?  How long can the larval stage last? %Other than dispersal, what other hypotheses explain a lengthy larval period?  What is the disadvantage of a long larval stage?

	\item What conditions favor sneaker and satellite males as an alternative reproductive strategy?  

%	\item Why is pelagic spawning common in marine fishes but uncommon in freshwater fishes? Conversely, why is benthic spawning common in freshwater fishes?

	\item Explain the difference between sequential and simultaneous hermaphroditism? Which is common and which is rare? 
	
	\item Describe the difference between protogyny and protandry. Does this apply to sequential or simultaneous hermaphrodites? Why?

\end{enumerate}


\end{document}