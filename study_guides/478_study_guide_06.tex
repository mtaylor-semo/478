%!TEX TS-program = lualatex
%!TEX encoding = UTF-8 Unicode

\documentclass[letterpaper]{tufte-handout}

%\geometry{showframe} % display margins for debugging page layout

\usepackage{fontspec}
\def\mainfont{Linux Libertine O}
\setmainfont[Ligatures={Common,TeX}, Contextuals={NoAlternate}, BoldFont={* Bold}, ItalicFont={* Italic}, Numbers={OldStyle,Proportional}]{\mainfont}
\setsansfont[Scale=MatchLowercase]{Linux Biolinum O} 
\usepackage{microtype}

\usepackage{graphicx} % allow embedded images
  \setkeys{Gin}{width=\linewidth,totalheight=\textheight,keepaspectratio}
  \graphicspath{	{/Users/goby/teach/163/lectures/}} % set of paths to search for images
\usepackage{amsmath}  % extended mathematics
\usepackage{booktabs} % book-quality tables
\usepackage{units}    % non-stacked fractions and better unit spacing
\usepackage{multicol} % multiple column layout facilities
%\usepackage{fancyvrb} % extended verbatim environments
%  \fvset{fontsize=\normalsize}% default font size for fancy-verbatim environments

\usepackage{enumitem}

\makeatletter
% Paragraph indentation and separation for normal text
\renewcommand{\@tufte@reset@par}{%
  \setlength{\RaggedRightParindent}{1.0pc}%
  \setlength{\JustifyingParindent}{1.0pc}%
  \setlength{\parindent}{1pc}%
  \setlength{\parskip}{0pt}%
}
\@tufte@reset@par

% Paragraph indentation and separation for marginal text
\renewcommand{\@tufte@margin@par}{%
  \setlength{\RaggedRightParindent}{0pt}%
  \setlength{\JustifyingParindent}{0.5pc}%
  \setlength{\parindent}{0.5pc}%
  \setlength{\parskip}{0pt}%
}
\makeatother

% Set up the spacing using fontspec features
   \renewcommand\allcapsspacing[1]{{\addfontfeatures{LetterSpace=15}#1}}
   \renewcommand\smallcapsspacing[1]{{\addfontfeatures{LetterSpace=10}#1}}


\title{{\scshape zo} 478 Study Guide 06}

%\author{}

\date{} % without \date command, current date is supplied

\begin{document}

\maketitle	% this prints the handout title, author, and date

\section*{Classification: Myxini – Chondrichthys.}

%\printclassoptions

\section{Vocabulary}\marginnote{\textbf{Study:} Chapters 2, 11; pages pgs 197--200, 205--229.} \marginnote{$^1$ Both Agnatha and Osteichthys are commonly seen and sometimes still used (incorrectly) in the literature.} \marginnote{† indicates extinct taxon.}
\vspace{-1\baselineskip}
\begin{multicols}{2}
%monophyletic \\
%paraphyletic \\
Cyclostomata \\
Chondrichthyes \\
Actinoptergyii \\
Sarcoptergyii \\
Agnatha$^1$ \\
Osteichthyes$^1$ \\
Conodonta† \\
Ostracoderms† \\
Myxini \\
Petromyzontida \\
paired species / satellite species \\
Placodermi \\
autostylic \\
amphistylic \\
hyostylic \\
ammocoete \\
Chondrichthyes \\
Holocephali \\
Elasmobranchii \\
Selachii \\
Batoidea 
\end{multicols}

\section{Concepts}

\begin{enumerate}
	\item Be able to recognize and distinguish classification groups that are monophyletic or not monophyletic.  Be able to explain why they are monophyletic or not.

	\item Are the fishes, as we currently recognize them, paraphyletic, monophyletic, or polyphyletic?  What about each of the classes? Which are and which, if any, are not?

	\item What characteristic(s) are shared among the jawless fishes?  When did they live? What major living and fossil groups are included within the jawless fishes?

	\item What is the argument against using Agnatha to classify jawless fishes?  Similarly, what is the argument against using Osteichthyes for jawed fishes with bony elements?

	\item Be able to draw a phylogeny with the correct placement of:
	\begin{itemize}
		\item The classes of extant fishes,
		\item Conodonta, and
		\item Ostracodermi.
	\end{itemize}
	We’ll add more taxa as we proceed.  Small steps.
	
	\item What geologic period is commonly known as the Age of Fishes?  Why is this so?

	\item Which fishes are craniates?  Which fishes are chordates?  What are the characteristics that define chordates, and what characteristics define craniates?  Be sure you know which is a phylum and which is a subphylum.
	
		\item What major characteristics define the Myxini?  What are two possible explanations for knotting behavior in the Myxini?

	\item What type of osmoregulation is characteristic of the Myxini?  What characteristic(s) of their habitat makes this type of osmoregulation possible?

	\item What are the major characteristics of Petromyzontida?

	\item Explain the paired species (also known as the satellite species) hypothesis in lampreys.  

	\item What characteristic(s) define the Gnathostomata?  What groups of extant and extinct fishes comprise the Gnathostomata?	

	\item If you formed a classification scheme based on the type of jaw support found in fishes, would the hyostylic classification be monophyletic?  What about for the other two types of suspension?  Why or why not?

	\item What are the possible evolutionary advantages that led to the evolution of jaws?

	\item Compare and contrast the three types of jaw suspension.  For each type of jaw suspension, name 1--2 groups of extant or fossil fishes that have that type of jaw suspension.

	\item What are the characteristics of the Chondrichthyes?

	\item List and briefly describe each of the five basic adaptations that underlie the evolution of the Chondrichthyes.

	\item Draw and label a phylogeny of the Chondrichthyes.  Provide 1--2 key characteristics for each group within the Chondrichthyes.

	\item Our phylogenetic hypothesis for the evolutionary relationships of Chondrichthyes is based almost entirely on current species.  Why do we not rely more heavily on the fossil record to fill in the gaps?

	\item What are the key characteristics of the subclass Holocephali?

	\item Based on their body form, what life history / ecology aspects would you predict for the the Holocephali? Think form and function.

	\item What are the key characteristics of the subclass Elasmobranchii?  What two groups make up the elasmobranchs?

\end{enumerate}

\end{document}