%!TEX TS-program = lualatex
%!TEX encoding = UTF-8 Unicode

\documentclass[letterpaper]{tufte-handout}

%\geometry{showframe} % display margins for debugging page layout

\usepackage{graphicx} % allow embedded images
  \setkeys{Gin}{width=\linewidth,totalheight=\textheight,keepaspectratio}
  \graphicspath{{img/}} % set of paths to search for images
  
\usepackage{fontspec}
  \setmainfont[Ligatures=TeX,Numbers={Proportional}]{Linux Libertine O}
  \setsansfont{Linux Biolinum O}
\usepackage{microtype}
\usepackage{enumitem}
\usepackage{multicol} % multiple column layout facilities
%\usepackage{hyperref}
%\usepackage{fancyvrb} % extended verbatim environments
%  \fvset{fontsize=\normalsize}% default font size for fancy-verbatim environments

% Change the header to shift the title to the left side of the page. 
% Replaced \quad with \hfill.  See \plaintitle in tufte-common.def
{\fancyhead[RE,RO]{\scshape{\newlinetospace{\plaintitle}}\hfill\thepage}}

\makeatletter
% Paragraph indentation and separation for normal text
\renewcommand{\@tufte@reset@par}{%
  \setlength{\RaggedRightParindent}{1.0pc}%
  \setlength{\JustifyingParindent}{1.0pc}%
  \setlength{\parindent}{1pc}%
  \setlength{\parskip}{0pt}%
}
\@tufte@reset@par

% Paragraph indentation and separation for marginal text
\renewcommand{\@tufte@margin@par}{%
  \setlength{\RaggedRightParindent}{0pt}%
  \setlength{\JustifyingParindent}{0.5pc}%
  \setlength{\parindent}{0.5pc}%
  \setlength{\parskip}{0pt}%
}
\makeatother

\title{Study Guide 14}
\author{Invasive Species and Conservation}

\date{} % without \date command, current date is supplied

\begin{document}

\maketitle	% this prints the handout title, author, and date

%\printclassoptions

\section{Vocabulary}\marginnote{\textbf{Study:} \textsc{na}} 
\vspace{-1\baselineskip}
\begin{multicols}{2}
invasive species
\\
exotic species
\\
bait bucket biology
\\
ballast water
\\
reserves 
\end{multicols}

\section{Concepts}

\begin{enumerate}
	\item Why are freshwater fishes the most endangered of all fishes?  What characteristics of the region make them so vulnerable?
	
	\item What features are necessary for a species of fish (or any organism, for that matter) to be a successful invader?
	
	\item List as many ways as you can think of (but at least three) that fishes have invaded outside of their native range.
	
	\item Explain the many ecological impacts of invasive species.
	
	\item Why do Florida and California have so many invasive species?
	
	\item What is an invasive species?  How does this differ from an exotic species?
	
	\item Why can’t deep-sea fishes or cold water fishes typically sustain commercial fisheries?
	
	\item What are the advantages of marine reserves on commercial fisheries?  Are they 100\% sucessful? Why or why not?
	
	\item How can we prevent (or reduce) further species invasions?  
	
	\item Explain the general characteristics of the region or habitat that increase a species vulnerability to extinction or extirpation.  Explain for marine fishes, freshwater fishes, and estuarine fishes.  Which is most vulnerable, and why?
	
	
\end{enumerate}

\end{document}