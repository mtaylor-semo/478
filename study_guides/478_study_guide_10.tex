%!TEX TS-program = lualatex
%!TEX encoding = UTF-8 Unicode

\documentclass[letterpaper]{tufte-handout}

%\geometry{showframe} % display margins for debugging page layout

\usepackage{fontspec}
\def\mainfont{Linux Libertine O}
\setmainfont[Ligatures={Common,TeX}, Contextuals={NoAlternate}, BoldFont={* Bold}, ItalicFont={* Italic}, Numbers={OldStyle,Proportional}]{\mainfont}
\setsansfont[Scale=MatchLowercase]{Linux Biolinum O} 
\usepackage{microtype}

\usepackage{graphicx} % allow embedded images
  \setkeys{Gin}{width=\linewidth,totalheight=\textheight,keepaspectratio}
  \graphicspath{	{/Users/goby/teach/163/lectures/}} % set of paths to search for images
\usepackage{amsmath}  % extended mathematics
\usepackage{booktabs} % book-quality tables
\usepackage{units}    % non-stacked fractions and better unit spacing
\usepackage{multicol} % multiple column layout facilities
%\usepackage{fancyvrb} % extended verbatim environments
%  \fvset{fontsize=\normalsize}% default font size for fancy-verbatim environments

\usepackage{enumitem}

\makeatletter
% Paragraph indentation and separation for normal text
\renewcommand{\@tufte@reset@par}{%
  \setlength{\RaggedRightParindent}{1.0pc}%
  \setlength{\JustifyingParindent}{1.0pc}%
  \setlength{\parindent}{1pc}%
  \setlength{\parskip}{0pt}%
}
\@tufte@reset@par

% Paragraph indentation and separation for marginal text
\renewcommand{\@tufte@margin@par}{%
  \setlength{\RaggedRightParindent}{0pt}%
  \setlength{\JustifyingParindent}{0.5pc}%
  \setlength{\parindent}{0.5pc}%
  \setlength{\parskip}{0pt}%
}
\makeatother

% Set up the spacing using fontspec features
   \renewcommand\allcapsspacing[1]{{\addfontfeatures{LetterSpace=15}#1}}
   \renewcommand\smallcapsspacing[1]{{\addfontfeatures{LetterSpace=10}#1}}


\title{{\scshape zo} 478 Study Guide 10}

%\author{}

\date{} % without \date comma nd, current date is supplied

\begin{document}

\maketitle	% this prints the handout title, author, and date

\section*{Adaptations to “extreme” conditions}

%\printclassoptions

\section{Vocabulary}\marginnote{\textbf{Study:} pgs. 393--401, 405--410, 417--419..} 
\vspace{-1\baselineskip}
\begin{multicols}{2}
mesopelagic \\
bathypelagic \\
benthopelagic \\
evolutionary convergence \\
tropical submergence \\
%hypogean \\
%troglobite \\
Antarctic convergence \\
antifreeze glycoproteins
glomeruli 
\end{multicols}

\section{Concepts}

\begin{enumerate}
	\item What characteristics would you use to distinguish between mesopelagic and bathypelagic fishes?  (e.g., musculature, eye size, etc).  Why do these characteristics differ between mesopelagic and bathypelagic fishes?

	\item List some of the environmental parameters of the deep-sea environment and specific adaptations that have allowed fishes to live in this habitat.

	\item How does the swim bladder and rete mirabile differ between a deep-sea species and a closely related shallow-water species?

	\item Describe specific adaptations that allow fishes to take advantage of limited prey in the deep sea.

	\item What are the uses of bioluminescence by deep-sea fishes?  Provide specific examples.

	\item Why do many mesopelagic fishes have ventral photophores?  Why do many mesopelagic fishes have countershading?

	\item Why do many mesopelagic fishes make nightly migrations to the epipelagic region?  How does this relate to the productivity of the epipelagic region? Your overall answer should address something more than “to feed.”  Relate it to productivity of the region.  

%	\item The epipelagic and benthopelagic (benthic region of the open ocean) contain the highest productivity, compared to the lower productivity of the intervening water column (e.g., mesopelagic and bathypelagic).  What accounts for the increase in productivity at the benthopelagic region?  Think broadly: account for \emph{all} sources of trophic input.

	\item The deep-sea is the largest habitat on earth.  What are the specific difficulties that a fish must overcome to be successful as an inhabitant of this vast volume?  What adaptations do they have to be successful?  Explain.
	
%	\item Explain sexual parasitism in ceratioid anglerfishes.

%	\item Explain how loosejaws use red light to search for prey when they can’t actually see red light.

	\item Explain the evolutionary convergence of troglobitic\sidenote{Adapted to permanant life in caves.} fishes and bathypelagic fishes.  What are the similar environmental factors and how have each group of fishes adapted?

	\item How have troglobitic fishes adapted to the low productivity of the cave environment?

	\item Explain how the grotto sculpin in Perry County appears to represent an adaptive transition from a surface habitat to the cave habitat?  What morphological and physiological changes are occurring?

	\item Polar fishes live at or near freezing temperatures of sea water.  What adaptations prevent fishes from freezing?

	\item Which polar fishes lack glomeruli and why?

	\item How can some polar fishes survive without hemoglobin?  What adaptions are necessary?  Explain how each contributes to oxygen uptake or distribution in fishes without hemoglobin.

	\item All notothenioids lack a swim bladder, but some species have achieved neutral buoyancy.  Explain how they have achieved neutrality.  Why is this important?

\end{enumerate}

\end{document}