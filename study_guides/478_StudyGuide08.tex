%!TEX TS-program = lualatex
%!TEX encoding = UTF-8 Unicode

\documentclass[nofonts, letterpaper]{tufte-handout}

%\geometry{showframe} % display margins for debugging page layout

\usepackage{graphicx} % allow embedded images
  \setkeys{Gin}{width=\linewidth,totalheight=\textheight,keepaspectratio}
  \graphicspath{{img/}} % set of paths to search for images
  
\usepackage{fontspec}
  \setmainfont[Ligatures=TeX,Numbers={Proportional}]{Linux Libertine O}
  \setsansfont{Linux Biolinum O}
\usepackage{microtype}
\usepackage{enumitem}
\usepackage{multicol} % multiple column layout facilities
%\usepackage{hyperref}
%\usepackage{fancyvrb} % extended verbatim environments
%  \fvset{fontsize=\normalsize}% default font size for fancy-verbatim environments

% Change the header to shift the title to the left side of the page. 
% Replaced \quad with \hfill.  See \plaintitle in tufte-common.def
{\fancyhead[RE,RO]{\scshape{\newlinetospace{\plaintitle}}\hfill\thepage}}

\makeatletter
% Paragraph indentation and separation for normal text
\renewcommand{\@tufte@reset@par}{%
  \setlength{\RaggedRightParindent}{1.0pc}%
  \setlength{\JustifyingParindent}{1.0pc}%
  \setlength{\parindent}{1pc}%
  \setlength{\parskip}{0pt}%
}
\@tufte@reset@par

% Paragraph indentation and separation for marginal text
\renewcommand{\@tufte@margin@par}{%
  \setlength{\RaggedRightParindent}{0pt}%
  \setlength{\JustifyingParindent}{0.5pc}%
  \setlength{\parindent}{0.5pc}%
  \setlength{\parskip}{0pt}%
}

\makeatother

\title{Study Guide 08}
\author{Sensory Perception, Communication, Behavior}

\date{} % without \date command, current date is supplied

\begin{document}

\maketitle	% this prints the handout title, author, and date

%\printclassoptions

\section{Vocabulary}\marginnote{\textbf{Study:} Pgs. 75--90, 268--269, 425, 448, 495--496, 515--523.} 
\vspace{-1\baselineskip}
\begin{multicols}{2}
chemoreception \\
olfactory organ \\
mechanoreception \\
sensory hair cell \\
Weberian ossicles \\
otolith \\
lapillus, sagitta, astericus \\
lateral line system \\
neuromast \\
electroreception \\
ampullary receptor \\
tuberous receptor \\
vision \\
choroid gland \\
tapetum lucidum \\
polychromatism \\
Schrekstoff \\
diadromy \\
catadromy \\
anadromy \\
amphidromy \\
shoal \\
school \\
dilution effect \\
confusion effect 
\end{multicols}

\section{Concepts}

\begin{enumerate}
	\item What is the function of pharyngeal teeth?  Relate general size and shape of pharyngeal teeth to possible diet.

	\item What is the function of gill rakers?  How do they aid in feeding?  Relate general size, shape and numbers of gill rakers to possible diet.

	\item How does gut length relate to diet?

	\item If I give you a particular diet (e.g., monophagous for benthic snails), how much of the overall form and function of the fish could you explain to me?  Think about body shapes, fin shapes and position, type of swimming, mouth position, potential types of jaw teeth, gill rakers and pharyngeal teeth, and length of the gut.  In other words, tie it all together!

	\item Describe the different mechanoreception functions for sensory hair cells.  Explain and illustrate how a sensory hair cell functions in hearing or in the lateral line system.

	\item What are the functions of the two types of electroreceptors?

	\item How do fishes meet the high oxygen demain of the unvascularized retina?

	\item How might the number and types of cones in the eye vary with the habitat of fishes? (e.g., why do deep-sea fishes not have cones sensitive to red wave lengths?).

	\item Which otolith functions for orientation?  Which two are used for sound detection?

	\item What are the different forms of diadromy?  What are the adaptive advantages of diadromy?

	\item What reasons have been postulated to explain shoaling in fish?

	\item What is the difference between shoaling and schooling?

	\item What forms of communication do fishes use?  Which one(s) do fishes most commonly use?

	\item Why can bright colors be a dangerous way to attract a mate?

	\item What colors are most commonly found in coral reef fishes?  Why?

	\item What is schreckstoff?  What group of fishes is known for the schreckstoff response?
\end{enumerate}

You would be wise to learn the four orders that make up the Otophysi fishes (Cypriniformes, Characiformes, Siluriformes, Gymnotiformes; see pg. 268) now although they won’t become fair game for tests until the second exam, after we have covered basic groups and systematics of fishes.  You will have to know them eventually, so you might as well start now.

\end{document}