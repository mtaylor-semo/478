%!TEX TS-program = lualatex
%!TEX encoding = UTF-8 Unicode

\documentclass[letterpaper]{tufte-handout}

%\geometry{showframe} % display margins for debugging page layout

\usepackage{graphicx} % allow embedded images
  \setkeys{Gin}{width=\linewidth,totalheight=\textheight,keepaspectratio}
  \graphicspath{{img/}} % set of paths to search for images
  
\usepackage{fontspec}
  \setmainfont[Ligatures=TeX,Numbers={Proportional}]{Linux Libertine O}
  \setsansfont{Linux Biolinum O}
\usepackage{microtype}
\usepackage{enumitem}
\usepackage{multicol} % multiple column layout facilities
%\usepackage{hyperref}
%\usepackage{fancyvrb} % extended verbatim environments
%  \fvset{fontsize=\normalsize}% default font size for fancy-verbatim environments

% Change the header to shift the title to the left side of the page. 
% Replaced \quad with \hfill.  See \plaintitle in tufte-common.def
{\fancyhead[RE,RO]{\scshape{\newlinetospace{\plaintitle}}\hfill\thepage}}

\makeatletter
% Paragraph indentation and separation for normal text
\renewcommand{\@tufte@reset@par}{%
  \setlength{\RaggedRightParindent}{1.0pc}%
  \setlength{\JustifyingParindent}{1.0pc}%
  \setlength{\parindent}{1pc}%
  \setlength{\parskip}{0pt}%
}
\@tufte@reset@par

% Paragraph indentation and separation for marginal text
\renewcommand{\@tufte@margin@par}{%
  \setlength{\RaggedRightParindent}{0pt}%
  \setlength{\JustifyingParindent}{0.5pc}%
  \setlength{\parindent}{0.5pc}%
  \setlength{\parskip}{0pt}%
}
\makeatother

\title{Study Guide 12}
\author{Neoptergyii, Miscellaneous Orders}

\date{} % without \date command, current date is supplied

\begin{document}

\maketitle	% this prints the handout title, author, and date

%\printclassoptions

\section{Vocabulary}\marginnote{\textbf{Study:} pgs. 255--258, 261--263, 268--274, 291-292, 300--303. Skim through chapter 15 to get a feel for the diversity of spiny-rayed fishes. I do not have time in this course to cover the full diversity, unfortunately.} 
\vspace{-1\baselineskip}
\begin{multicols}{2}
Neoptergyii \\
Division Teleostei \\
Otophysans \\
Cypriniformes \\
Siluriformes \\
Perciformes
\end{multicols}

\section{Concepts}

\begin{enumerate}
	\item Name the three groups that compose the Neopterygii.

	\item Know the entire phylogeny of fishes down to the Division Teleostei.  Be able to fill in blanks on a provided phylogeny, and be able to insert a few missing branches into the proper position.  You need to be able to label and position the classes, subclasses, the ostracoderns, and the two orders that are sister to the Division Teleostei.\sidenote{A phylogeny and blank practice tree is available to you online.}

	\item Given a systematic level, you should be able to name the important groups within that systematic level, such as the two subclasses that make up the Class Sarcopterygii, or the three groups that comprise the Subclass Neoptergyii.  Conversely, I can ask you to name the higher systematic level, given a list of groups within that level, such as identifying the subclass that is formed by the Lepisosteiformes, Amiiformes and the Division Teleostei.

	\item What internal anatomical characters related to sound reception unite the Cypriniformes and Siluriformes?  What is the name of this group?  What are the other two orders listed in class that belong this group? Refer back to Study Guide 08.\sidenote{Read also pages 269--274.}

	\item What is the largest order of fishes in the world?  What is the largest family?  What is the largest family of fishes in Missouri?

	\item What is the largest order of teleost fishes?  Is this order monophyletic?  What characters can generally distinguish this order?

	\item What are the scientific names for the eight largest families of fishes within the Perciformes?  In total, approximately how many fishes are represented by these eight families?  

	\item Name the five perciform families that have Missouri representatives.  Which one is the largest in Missouri (in the US for that matter).

\end{enumerate}

\end{document}