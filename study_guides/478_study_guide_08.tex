%!TEX TS-program = lualatex
%!TEX encoding = UTF-8 Unicode

\documentclass[letterpaper]{tufte-handout}

%\geometry{showframe} % display margins for debugging page layout

\usepackage{fontspec}
\def\mainfont{Linux Libertine O}
\setmainfont[Ligatures={Common,TeX}, Contextuals={NoAlternate}, BoldFont={* Bold}, ItalicFont={* Italic}, Numbers={OldStyle,Proportional}]{\mainfont}
\setsansfont[Scale=MatchLowercase]{Linux Biolinum O} 
\usepackage{microtype}

\usepackage{graphicx} % allow embedded images
  \setkeys{Gin}{width=\linewidth,totalheight=\textheight,keepaspectratio}
  \graphicspath{	{/Users/goby/teach/163/lectures/}} % set of paths to search for images
\usepackage{amsmath}  % extended mathematics
\usepackage{booktabs} % book-quality tables
\usepackage{units}    % non-stacked fractions and better unit spacing
\usepackage{multicol} % multiple column layout facilities
%\usepackage{fancyvrb} % extended verbatim environments
%  \fvset{fontsize=\normalsize}% default font size for fancy-verbatim environments

\usepackage{enumitem}

\makeatletter
% Paragraph indentation and separation for normal text
\renewcommand{\@tufte@reset@par}{%
  \setlength{\RaggedRightParindent}{1.0pc}%
  \setlength{\JustifyingParindent}{1.0pc}%
  \setlength{\parindent}{1pc}%
  \setlength{\parskip}{0pt}%
}
\@tufte@reset@par

% Paragraph indentation and separation for marginal text
\renewcommand{\@tufte@margin@par}{%
  \setlength{\RaggedRightParindent}{0pt}%
  \setlength{\JustifyingParindent}{0.5pc}%
  \setlength{\parindent}{0.5pc}%
  \setlength{\parskip}{0pt}%
}
\makeatother

% Set up the spacing using fontspec features
   \renewcommand\allcapsspacing[1]{{\addfontfeatures{LetterSpace=15}#1}}
   \renewcommand\smallcapsspacing[1]{{\addfontfeatures{LetterSpace=10}#1}}


\title{{\scshape zo} 478 Study Guide 08}

%\author{}

\date{} % without \date command, current date is supplied

\begin{document}

\maketitle	% this prints the handout title, author, and date

\section*{General ecology of fishes; ecology of freshwater fishes.}

%\printclassoptions

\section{Vocabulary}\marginnote{\textbf{Study:} pgs. 239 (Freshwater Fishes), 348--354, 492--495, 529 (Populations), 536--544, Schlosser 1987.} 
\vspace{-1\baselineskip}
\begin{multicols}{2}
population \\
assemblage \\
community \\
niche \\
fundamental niche \\
realized niche \\
guild \\
competition \\
resource partitioning \\
character displacement \\
predator \\
prey \\
stream order \\
stream gradient \\
predator-prey interactions
\end{multicols}

\section{Concepts}

\begin{enumerate}
	\item Describe the different mechanisms or “filters” that ultimately determine which species of fishes occur at a given locality.  Provide an example of each, and then explain how each contributes to the observed assemblage or community.

	\item Explain the difference between a population, an assemblage, and a community.

	\item Explain the difference between a fundamental niche, a realized niche, and a guild.   

	\item Explain how interspecific competition can lead to character displacement.

	\item Describe how intraspecific competition can lead to fluctuations in population size.  What effect might population size fluctuations have on the overall assemblage?  Why?
	
	\item What’s the difference between biotic and abiotic factors when discussing any aspect of ecology?  Provide some specific examples.
	
	\item How do stream gradient and stream order influence an assemblage of freshwater fishes?  Think broadly.  Consider all of the topics we discussed in class: temperature, water flow, substrate, community complexity, etc.
	
	\item What is the major determinant of fish assemblages in temperate freshwaters, environmental factors or biological interactions?  Which specific factors or interactions?  Can these change in different zones in a river or a lake?  What about tropical regions?
	
	\item Given a large temperate watershed, how would you expect a first or second order stream to differ from a fifth or sixth order stream?  What would be the dominant groups of fishes?  What are the relative effects of environment and species interactions?
	
	\item What specific processes allow for the coexistence of a large number of minnows or darters in an Ozark stream? Provide a reasonable example.
		
	\item Carefully read the paper handed out in class (also posted with slides for this lecture).\sidenote{Schlosser, I.J. 1987. A conceptual framework for fish communities in small warmwater streams. pp. 17--24. In: Matthews, W.J. and D.C. Heins (eds). \textit{Community and Evolutionary Ecology of North American Stream Fishes}. Univ. Oklahoma Press, Norman, OK.} Study the entire paper but concentrate on the section called “The Conceptual Framework.” Relate this paper to our discussion of stream order and stream gradient, plus the relevant information from the assigned textbook reading. You must read the entire paper because it provide the evidence Schlosser uses to build the conceptual framework.
%
\end{enumerate}

\end{document}