%!TEX TS-program = lualatex
%!TEX encoding = UTF-8 Unicode

\documentclass[letterpaper]{tufte-handout}

%\geometry{showframe} % display margins for debugging page layout

\usepackage{fontspec}
\def\mainfont{Linux Libertine O}
\setmainfont[Ligatures={Common,TeX}, Contextuals={NoAlternate}, BoldFont={* Bold}, ItalicFont={* Italic}, Numbers={OldStyle,Proportional}]{\mainfont}
\setsansfont[Scale=MatchLowercase]{Linux Biolinum O} 
\usepackage{microtype}

\usepackage{graphicx} % allow embedded images
  \setkeys{Gin}{width=\linewidth,totalheight=\textheight,keepaspectratio}
  \graphicspath{	{/Users/goby/teach/163/lectures/}} % set of paths to search for images
\usepackage{amsmath}  % extended mathematics
\usepackage{booktabs} % book-quality tables
\usepackage{units}    % non-stacked fractions and better unit spacing
\usepackage{multicol} % multiple column layout facilities
%\usepackage{fancyvrb} % extended verbatim environments
%  \fvset{fontsize=\normalsize}% default font size for fancy-verbatim environments

\usepackage{enumitem}

\makeatletter
% Paragraph indentation and separation for normal text
\renewcommand{\@tufte@reset@par}{%
  \setlength{\RaggedRightParindent}{1.0pc}%
  \setlength{\JustifyingParindent}{1.0pc}%
  \setlength{\parindent}{1pc}%
  \setlength{\parskip}{0pt}%
}
\@tufte@reset@par

% Paragraph indentation and separation for marginal text
\renewcommand{\@tufte@margin@par}{%
  \setlength{\RaggedRightParindent}{0pt}%
  \setlength{\JustifyingParindent}{0.5pc}%
  \setlength{\parindent}{0.5pc}%
  \setlength{\parskip}{0pt}%
}
\makeatother

% Set up the spacing using fontspec features
   \renewcommand\allcapsspacing[1]{{\addfontfeatures{LetterSpace=15}#1}}
   \renewcommand\smallcapsspacing[1]{{\addfontfeatures{LetterSpace=10}#1}}


\title{{\scshape zo} 478 Study Guide 11}

%\author{}

\date{} % without \date comma nd, current date is supplied

\begin{document}

\maketitle	% this prints the handout title, author, and date

\section*{Invasive species and conservation}

%\printclassoptions

\section{Vocabulary}\marginnote{\textbf{Study:} \textsc{na}} 
\vspace{-1\baselineskip}
\begin{multicols}{2}
invasive species \\
exotic species \\
bait bucket biology \\
ballast water \\
reserves 
\end{multicols}

\section{Concepts}

\begin{enumerate}
	\item What is an invasive species?  How does this differ from an exotic species? Are they mutually exclusive?
	
	\item What features are necessary for a species of fish (or any organism, for that matter) to be a successful invader?
	
	\item List as many ways as you can think of (but at least three) that fishes have invaded outside of their native range.
	
	\item Explain the many ecological impacts of invasive species.
	
	\item Why do Florida and California have so many invasive species?
	
	\item How can we prevent (or reduce) further species invasions?  
	
	\item Why can’t deep-sea fishes or cold water fishes typically sustain commercial fisheries?
	
	\item Why are freshwater fishes the most endangered of all fishes?  What characteristics of the region make them so vulnerable?
	
	\item What are the advantages of marine reserves on commercial fisheries?  Are they 100\% sucessful? Why or why not?
	
	\item Explain the general characteristics of the region or habitat that increase a species vulnerability to extinction or extirpation.  Explain for marine fishes, freshwater fishes, and estuarine fishes.  Which is most vulnerable, and why?
	
	
\end{enumerate}

\end{document}