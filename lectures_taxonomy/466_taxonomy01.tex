%!TEX TS-program = lualatex
%!TEX encoding = UTF-8 Unicode

\documentclass[t]{beamer}

%%%% HANDOUTS For online Uncomment the following four lines for handout
%\documentclass[t,handout]{beamer}  %Use this for handouts.
%\includeonlylecture{student}
%\usepackage{handoutWithNotes}
%\pgfpagesuselayout{3 on 1 with notes}[letterpaper,border shrink=5mm]
%	\setbeamercolor{background canvas}{bg=black!5}


%%% Including only some slides for students.
%%% Uncomment the following line. For the slides,
%%% use the labels shown below the command.
%\includeonlylecture{student}

%% For students, use \lecture{student}{student}
%% For mine, use \lecture{instructor}{instructor}


%\usepackage{pgf,pgfpages}
%\pgfpagesuselayout{4 on 1}[letterpaper,border shrink=5mm]

% FONTS
\usepackage{fontspec}
\def\mainfont{Linux Biolinum O}
\setmainfont[Ligatures=TeX, Contextuals={NoAlternate}, BoldFont={* Bold}, ItalicFont={* Italic}, Numbers={Proportional}]{\mainfont}
\setmonofont[Scale=MatchLowercase]{Linux Libertine Mono O} 
\setsansfont[Scale=MatchLowercase]{Linux Biolinum O} 
\usepackage{microtype}

\usepackage{graphicx}
	\graphicspath{%
	{/Users/goby/Pictures/teach/466/lectures/}%
	{img/}}%
%	{/Users/goby/Pictures/teach/common/}} % set of paths to search for images

%\usepackage{amsmath,amssymb}

%\usepackage{units}

%\usepackage{booktabs}
\usepackage{multicol}
%	\setlength{\columnsep=1em}

%\usepackage{textcomp}
%\usepackage{setspace}
\usepackage{tikz}
%	\tikzstyle{every picture}+=[remember picture,overlay]

\usepackage{forest}
\usetikzlibrary{trees}
\tikzstyle{block} = [rectangle, draw, fill=white, rounded corners, minimum size=2em]
\tikzstyle{branch} = [thick, draw]

%\usetikzlibrary{positioning, backgrounds}


\forestset{
	every leaf node/.style={
		if n children=0{#1}{}
	},
	every tree node/.style={
		if n children=0{}{#1}
	},
	mytree/.style={
		for tree={
			font=\footnotesize\selectfont,
			edge path={
				\noexpand\path [draw, thick, \forestoption{edge}] (!u.parent anchor) |- (.child anchor)\forestoption{edge label};
			},
			every tree node={draw=none,inner sep=0, outer sep=0, minimum size=0},
			%every leaf node/.style={align=left},
			grow'=0,
			parent anchor=east, 
			child anchor=west,
			anchor=west,
			l = 2cm,
			%l sep=1.5cm,
			s sep=0mm,
			draw=none,
			if n children=0{tier=word}{}
		}
	}
}

\mode<presentation>
{
  \usetheme{Lecture}
  \setbeamercovered{invisible}
  \setbeamertemplate{items}[square]
}

%\usepackage{calc}
\usepackage{hyperref}

% shortstack needed to highlight across \\ line break.
\newcommand\sshighlight[1]{%
	\highlight{\shortstack[l]{#1}}%
}



\begin{document}


\lecture{student}{student}
{
\begin{frame}[t,plain]{Our taxonomic goals for this course:}


\hangpara \highlight{Learn higher classification of birds to family level.} Classification will generally follow Sangster~et~al.~2022. Phylogenetic definitions for 25 higher-level clade names of birds. Avian Research 13: 100027.


\hangpara \highlight{Learn identifying characters of bird families.} Families identification will focus on those with Missouri representation.

\end{frame}
}

{
\usebackgroundtemplate{\includegraphics[width=\paperwidth]{theropod_phylogeny}}
\begin{frame}[b,plain]{Recall: the Avialae clade is defined by powered flight, and includes \highlight{Neornithes,} the modern birds.}

\begin{tikzpicture}
	\node at (12.4,0.1) [left] {\tiny Zimmer 2011. \textit{The Tangled Bank}, Roberts and Co.};
	\node at (3.5,1.2)[left] {\normalsize \highlight{Avialae}};
	\node at (2.55,4.25)[left]{\normalsize \textcolor{white}{Maniraptora}};
	\node at (12.01,0.8)[left, fill=white]{\footnotesize \highlight{Neornithes}};
	\node at (11.7,1.2)[left, fill=white]{\phantom{fr}};
%	\node at (12.4,7.6)[gray,left]{See also Fig. 2–10 of your text.};
\end{tikzpicture}
\end{frame}
}


\begin{frame}{You must learn these major taxa of Neornithes.}


\begin{forest} mytree
[[, l sep=+1.7cm, edge label = {node [xshift=-0.8cm, text width = 1.5cm] {\footnotesize Subclass Neornithes}}
	[,name=neognathae, edge label = {node [text width=2cm, midway, xshift=1.1cm] {\footnotesize Infraclass Neognathae}}
		[, name=neoaves, edge label = {node [text width=2cm, midway, xshift=1.1cm] {\footnotesize Superorder Neoaves}}
			[,l-=1cm
				[Clade\\ Passerea, align=left]
				[Clade\\ Columbimorphae, align=left]
			]
			[Clade\\ Mirandornithes, align=left]
		]
		[, name=galloanseres, edge label = {node [text width=2cm, midway, xshift=1.1cm] {\footnotesize Superorder Galloanseres}}
			[Order\\ Galliformes, align=left]
			[Order\\ Anseriformes, align=left]
		]
	]
	[,name=paleognathae, edge label = {node [text width=1.75cm, midway, xshift=1cm] {\footnotesize Infraclass Paleognathae}}
		[Tinamous\\ and “Ratites”, align=left]
		%[Grade\\ “Ratites”, align=left]
	]
]]
\end{forest}

\end{frame}

\lecture{instructor}{instructor}

\begin{frame}{You must learn these major taxa of Neornithes.}

\begin{forest} mytree
[[, l sep=+1.7cm, edge label = {node [xshift=-0.8cm, text width = 1.5cm] {\footnotesize Subclass Neornithes}}
	[,name=neognathae, edge label = {node [text width=2cm, midway, xshift=1.1cm] {\footnotesize \highlight{Infraclass Neognathae}}}
		[, name=neoaves, edge label = {node [text width=2cm, midway, xshift=1.1cm] {\footnotesize Superorder Neoaves}}
			[,l-=1cm
				[Clade\\ Passerea, align=left]
				[Clade\\ Columbimorphae, align=left]
			]
			[Clade\\ Mirandornithes, align=left]
		]
		[, name=galloanseres, edge label = {node [text width=2cm, midway, xshift=1.1cm] {\footnotesize Superorder Galloanseres}}
			[Order\\ Galliformes, align=left]
			[Order\\ Anseriformes, align=left]
		]
	]
	[,name=paleognathae, edge label = {node [text width=1.75cm, midway, xshift=1cm] {\footnotesize \highlight{Infraclass Paleognathae}}}
		[Tinamous\\ and “Ratites”, align=left]
		%[Grade\\ “Ratites”, align=left]
	]
]]
\end{forest}

\end{frame}


\lecture{student}{student}


{
\usebackgroundtemplate{\includegraphics[width=\paperwidth]{skulls_palaeo_neognath}}
\begin{frame}[b,plain]{Palaeo- and Neognaths have different palate structures.}
	\hfill\tiny Modified from Tyne and Berger 1976, \textit{Fundamentals of Ornithology}, Wiley \& Sons.
\end{frame}
}

\lecture{instructor}{instructor}

\begin{frame}{You must learn these major taxa of Neornithes.}

\begin{forest} mytree
[[, l sep=+1.7cm, edge label = {node [xshift=-0.8cm, text width = 1.5cm] {\footnotesize Subclass Neornithes}}
	[,name=neognathae, edge label = {node [text width=2cm, midway, xshift=1.1cm] {\footnotesize Infraclass Neognathae}}
		[, name=neoaves, edge label = {node [text width=2cm, midway, xshift=1.1cm] {\footnotesize Superorder Neoaves}}
			[,l-=1cm
				[Clade\\ Passerea, align=left]
				[Clade\\ Columbimorphae, align=left]
			]
			[Clade\\ Mirandornithes, align=left]
		]
		[, name=galloanseres, edge label = {node [text width=2cm, midway, xshift=1.1cm] {\footnotesize Superorder Galloanseres}}
			[Order\\ Galliformes, align=left]
			[Order\\ Anseriformes, align=left]
		]
	]
	[,name=paleognathae, edge label = {node [text width=1.75cm, midway, xshift=1cm] {\footnotesize Infraclass Paleognathae}}
		[\sshighlight{Tinamous\\ and “Ratites”}, align=left]
		%[Grade\\ “Ratites”, align=left]
	]
]]
\end{forest}

\end{frame}

{
\usebackgroundtemplate{\includegraphics[width=\paperwidth]{crested_tinamou}}
\begin{frame}[b,plain]{\textcolor{white}{Tinamous are capable of weak flight.}}
	\tiny\hfill\textcolor{white}{Crested Tinamou by Evanphoto, Wikimedia Commons.}
\end{frame}
}

{
\usebackgroundtemplate{\includegraphics[width=\paperwidth]{cassowary}}
\begin{frame}[b,plain]{\textcolor{white}{Cassowaries and other “ratites” cannot fly.}}
	\tiny\hfill\textcolor{white}{Double-wattled Cassowary by Brian Gratwicke, Flickr Creative Commons.}
\end{frame}
}

{
\usebackgroundtemplate{\includegraphics[width=\paperwidth]{ratite_phylogeny}}
\begin{frame}[b,plain]
	\tiny\hfill Mitchell et al. 2014, Science 344: 898.
\end{frame}
}

\lecture{instructor}{instructor}

\begin{frame}{You must learn these major taxa of Neornithes.}

\begin{forest} mytree
[[, l sep=+1.7cm, edge label = {node [xshift=-0.8cm, text width = 1.5cm] {\footnotesize Subclass Neornithes}}
	[,name=neognathae, edge label = {node [text width=2cm, midway, xshift=1.1cm] {\footnotesize Infraclass Neognathae}}
		[, name=neoaves, edge label = {node [text width=2cm, midway, xshift=1.1cm] {\footnotesize Superorder Neoaves}}
			[,l-=1cm
				[Clade\\ Passerea, align=left]
				[Clade\\ Columbimorphae, align=left]
			]
			[Clade\\ Mirandornithes, align=left]
		]
		[, name=galloanseres, edge label = {node [text width=2cm, midway, xshift=1.1cm] {\footnotesize \highlight{Superorder Galloanseres}}}
			[\sshighlight{Order\\ Galliformes}, align=left]
			[\sshighlight{Order\\ Anseriformes}, align=left]
		]
	]
	[,name=paleognathae, edge label = {node [text width=1.75cm, midway, xshift=1cm] {\footnotesize Infraclass Paleognathae}}
		[Tinamous\\ and “Ratites”, align=left]
		%[Grade\\ “Ratites”, align=left]
	]
]]
\end{forest}

\end{frame}


\lecture{student}{student}

{
\usebackgroundtemplate{\includegraphics[width=\paperwidth]{galloanseres_phylogeny}}
\begin{frame}[b,plain]
	\tiny Kriegs et al. 2007. BMC Evolutionary Biology 7: 190.
\end{frame}
}

\lecture{instructor}{instructor}

\begin{frame}{You must learn these major taxa of Neornithes.}

\begin{forest} mytree
[[, l sep=+1.7cm, edge label = {node [xshift=-0.8cm, text width = 1.5cm] {\footnotesize Subclass Neornithes}}
	[,name=neognathae, edge label = {node [text width=2cm, midway, xshift=1.1cm] {\footnotesize Infraclass Neognathae}}
		[, name=neoaves, edge label = {node [text width=2cm, midway, xshift=1.1cm] {\footnotesize Superorder Neoaves}}
			[,l-=1cm
				[\sshighlight{Clade\\ Passerea}, align=left]
				[\sshighlight{Clade\\ Columbimorphae}, align=left]
			]
			[\sshighlight{Clade\\ Mirandornithes}, align=left]
		]
		[, name=galloanseres, edge label = {node [text width=2cm, midway, xshift=1.1cm] {\footnotesize Superorder Galloanseres}}
			[Order\\ Galliformes, align=left]
			[Order\\ Anseriformes, align=left]
		]
	]
	[,name=paleognathae, edge label = {node [text width=1.75cm, midway, xshift=1cm] {\footnotesize Infraclass Paleognathae}}
		[Tinamous\\ and “Ratites”, align=left]
		%[Grade\\ “Ratites”, align=left]
	]
]]
\end{forest}

\end{frame}


\lecture{student}{student}

\begin{frame}[t]{Mirandornithes contains the flamingos and grebes.}
\centering
\includegraphics[height=0.82\textheight]{taxonomy_mirandornithes}

\vfilll

\tinyfill \textcopyright\,\href{https://www.deviantart.com/gredinia/art/Bird-cladistic-Mirandornithes-diversity-698780401}{Gredinia, Deviant Art}
\end{frame}

\begin{frame}[t]{Columbimorphae contains the doves.}
\centering
\includegraphics[height=0.82\textheight]{taxonomy_columbimorphae}

\vfilll

\tinyfill \textcopyright\,\href{https://www.deviantart.com/gredinia/art/Bird-cladistic-Columbimorphae-diversity-705097984}{Gredinia, Deviant Art}
\end{frame}

{
\usebackgroundtemplate{\includegraphics[width=\paperwidth]{taxonomy_passerea}}
\begin{frame}[b,plain]
	\tiny\hfill Jarvis et al. 2014. Science 346: 1320.
\end{frame}
}

\end{document}
