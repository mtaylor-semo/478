%!TEX TS-program = lualatex
%!TEX encoding = UTF-8 Unicode

\documentclass[t]{beamer}

%%%% HANDOUTS For online Uncomment the following four lines for handout
%\documentclass[t,handout]{beamer}  %Use this for handouts.
%\includeonlylecture{student}
%\usepackage{handoutWithNotes}
%\pgfpagesuselayout{3 on 1 with notes}[letterpaper,border shrink=5mm]


%%% Including only some slides for students.
%%% Uncomment the following line. For the slides,
%%% use the labels shown below the command.
%\includeonlylecture{student}

%% For students, use \lecture{student}{student}
%% For mine, use \lecture{instructor}{instructor}


%\usepackage{pgf,pgfpages}
%\pgfpagesuselayout{4 on 1}[letterpaper,border shrink=5mm]

% FONTS
\usepackage{fontspec}
\def\mainfont{Linux Biolinum O}
\setmainfont[Ligatures=TeX, Contextuals={NoAlternate}, BoldFont={* Bold}, ItalicFont={* Italic}, Numbers={Proportional}]{\mainfont}
\setmonofont[Scale=MatchLowercase]{Linux Libertine Mono O} 
\setsansfont[Scale=MatchLowercase]{Linux Biolinum O} 
\usepackage{microtype}

\usepackage{graphicx}
	\graphicspath{%
	{/Users/goby/Pictures/teach/466/lectures/}%
	{img/}}%
%	{/Users/goby/Pictures/teach/common/}} % set of paths to search for images

%\usepackage{amsmath,amssymb}

%\usepackage{units}

%\usepackage{booktabs}
\usepackage{multicol}
%	\setlength{\columnsep=1em}

\usepackage{array}
\newcolumntype{L}[1]{>{\raggedright\let\newline\\\arraybackslash\hspace{0pt}}p{#1}}
\newcolumntype{C}[1]{>{\centering\let\newline\\\arraybackslash\hspace{0pt}}p{#1}}
\newcolumntype{R}[1]{>{\raggedleft\let\newline\\\arraybackslash\hspace{0pt}}p{#1}}

%usepackage{tikz}
%	\tikzstyle{every picture}+=[remember picture,overlay]

\usepackage{forest}
\usetikzlibrary{trees}
\tikzstyle{block} = [rectangle, draw, fill=white, rounded corners, minimum size=2em]
\tikzstyle{branch} = [thick, draw]

%\usetikzlibrary{positioning, backgrounds}


\forestset{
	every leaf node/.style={
		if n children=0{#1}{}
	},
	every tree node/.style={
		if n children=0{}{#1}
	},
	mytree/.style={
		for tree={
			font=\footnotesize\selectfont,
			edge path={
				\noexpand\path [draw, thick, \forestoption{edge}] (!u.parent anchor) |- (.child anchor)\forestoption{edge label};
			},
			every tree node={draw=none,inner sep=0, outer sep=0, minimum size=0},
			%every leaf node/.style={align=left},
			grow'=0,
			parent anchor=east, 
			child anchor=west,
			anchor=west,
			l = 2cm,
			%l sep=1.5cm,
			s sep=0mm,
			draw=none,
			if n children=0{tier=word}{}
		}
	}
}

\mode<presentation>
{
  \usetheme{Lecture}
  \setbeamercovered{invisible}
  \setbeamertemplate{items}[square]
}

%\usepackage{calc}
\usepackage{hyperref}

% shortstack needed to highlight across \\ line break.
\newcommand\sshighlight[1]{%
	\highlight{\shortstack[l]{#1}}%
}

\newcommand{\backoneline}{\vspace{-\baselineskip}}

\begin{document}


\lecture{student}{student}


\begin{frame}{You must learn these major taxa of Neornithes.}


\begin{forest} mytree
[[, l sep=+1.7cm, edge label = {node [xshift=-0.8cm, text width = 1.5cm] {\footnotesize Subclass Neornithes}}
	[,name=neognathae, edge label = {node [text width=2cm, midway, xshift=1.1cm] {\footnotesize Infraclass Neognathae}}
		[, name=neoaves, edge label = {node [text width=2cm, midway, xshift=1.1cm] {\footnotesize Superorder Neoaves}}
			[,l-=1cm
				[\sshighlight{Clade\\ Passerea}, align=left]
				[Clade\\ Columbimorphae, align=left]
			]
			[Clade\\ Mirandornithes, align=left]
		]
		[, name=galloanseres, edge label = {node [text width=2cm, midway, xshift=1.1cm] {\footnotesize Superorder Galloanseres}}
			[Order\\ Galliformes, align=left]
			[Order\\ Anseriformes, align=left]
		]
	]
	[,name=paleognathae, edge label = {node [text width=1.75cm, midway, xshift=1cm] {\footnotesize Infraclass Paleognathae}}
		[Tinamous\\ and “Ratites”, align=left]
		%[Grade\\ “Ratites”, align=left]
	]
]]
\end{forest}

\end{frame}


{
\usebackgroundtemplate{\includegraphics[width=\paperwidth]{taxonomy_passerea}}
{	\tikzstyle{every picture}+=[remember picture,overlay]
\definecolor{orange5}{HTML}{F16913}
\begin{frame}[b, plain]

\begin{tikzpicture}

\draw [white, fill=white] (5,0.5) rectangle (12,1.1);

\draw [white, fill=white] (5,6) rectangle (12,7);


\draw[ultra thick, orange5] (0.8,5.71) rectangle (5.0,6.5) node (placeholder){};

\draw [->, orange5, ultra thick] (7,6.05) -- (5.1, 6.05);
\node[minimum width=5.5cm, align=left, color=orange5] at (8.3,6.07) {Strigiformes\\[-0.2em] Acciptriformes};


\draw[ultra thick, orange5] (0.8,5) rectangle (5.0,5.7) node (placeholder){};
\draw [->, orange5, ultra thick] (7,5.35) -- (5.1, 5.35);

\node[minimum width=5.5cm, align=left, color=orange5] at (8.3,5.37) {Pelicaniformes};



\end{tikzpicture}
	\tiny\hfill Jarvis et al. 2014. Science 346: 1320.
\end{frame}
}}


\begin{frame}{Pelicaniformes: \highlight{Ardeidae} —  herons and egrets are wading birds with long bills and legs.}
\vspace{-0.5\baselineskip}
\centering
\includegraphics[height=0.78\textheight]{taxonomy_gbhe_greg}

\vfilll
\tiny Great Blue Heron by \href{https://en.wikipedia.org/wiki/File:Great_Blue_Heron_at_Sunnyvale_California.jpg}{Alpinekid, Wikimedia, \ccbysa{4.0}} \hfill Great Egret by \href{https://en.wikipedia.org/wiki/File:Ardea_modesta.jpg}{JJ Harrison, Wikimedia, \ccbysa{3.0}}
\end{frame}

\begin{frame}{Ardeidae have a pectinate third (middle) toe.}
\includegraphics[width=\linewidth]{taxonomy_pectinate_toe}

The foot is anisodactyl with an incumbent hallux.

\tinyfill \href{https://besgroup.org/2021/09/16/cattle-egret-pectinate-claw/}{\textcopyright\,Amar Singh HSS, Bird Ecology Study Group}

\end{frame}

\begin{frame}{Cattle Egret (left) and Snowy Egret (right)}
\includegraphics[width=\linewidth]{taxonomy_caeg_sneg}
Note difference in bill length and color. Cattle Egret feathers are white when not breeding. Snowy Egret has black legs and yellow feet.

\vfilll

\tiny Cattle Egret by \href{https://flickr.com/photos/peggycadigan/14158508211}{Peggy Cadigan, Flickr, \ccby{2.0}} \hfill Snowy Egret by \href{https://commons.wikimedia.org/wiki/File:Snowy_Egret-1.jpg}{Pat Hansen, Wikimedia, \ccby{4.0}}
\end{frame}

\begin{frame}{Green Heron (left) and Little Blue Heron (middle, right)}
\includegraphics[height=5.2cm]{taxonomy_grhe}\hfill 
\includegraphics[height=5.2cm]{taxonomy_lbhe}

\vfilll

\tiny Green Heron: \href{https://commons.wikimedia.org/wiki/File:Green_heron_in_PP_(14296).jpg}{Rhododendrites, \ccbysa{4.0}} \hfill Little Blue Heron: (middle) \href{https://flickr.com/photos/manjithkaini/4686037391}{Manjith Kainickara, \ccbysa{2.0}}\hfill (right) \href{https://flickr.com/photos/ferjflores/10765140464}{Fernando Flores, \ccbysa{2.0}}
\end{frame}

\begin{frame}{Black-crowned Night Heron and Yellow-crowned Night Heron are most active at night.}
\includegraphics[width=0.49\linewidth]{taxonomy_bcnh} \hfill
\includegraphics[width=0.49\linewidth]{taxonomy_ycnh}

\vfilll

\tiny Black-crowned Night Heron, \href{https://flickr.com/photos/pazzani/6021109879}{Mike's Birds, \ccbysa{2.0}} \hfill Yellow-crowned Night Heron by \href{https://flickr.com/photos/wildreturn/51220695991}{Andy Reago \& Chrissy McClarren, \ccby{2.0}}
\end{frame}




\end{document}
