%!TEX TS-program = lualatex
%!TEX encoding = UTF-8 Unicode

\documentclass[t]{beamer}

%%%% HANDOUTS For online Uncomment the following four lines for handout
%\documentclass[t,handout]{beamer}  %Use this for handouts.
%\includeonlylecture{student}
%\usepackage{handoutWithNotes}
%\pgfpagesuselayout{3 on 1 with notes}[letterpaper,border shrink=5mm]


%%% Including only some slides for students.
%%% Uncomment the following line. For the slides,
%%% use the labels shown below the command.
%\includeonlylecture{student}

%% For students, use \lecture{student}{student}
%% For mine, use \lecture{instructor}{instructor}


%\usepackage{pgf,pgfpages}
%\pgfpagesuselayout{4 on 1}[letterpaper,border shrink=5mm]

% FONTS
\usepackage{fontspec}
\def\mainfont{Linux Biolinum O}
\setmainfont[Ligatures=TeX, Contextuals={NoAlternate}, BoldFont={* Bold}, ItalicFont={* Italic}, Numbers={Proportional}]{\mainfont}
\setmonofont[Scale=MatchLowercase]{Linux Libertine Mono O} 
\setsansfont[Scale=MatchLowercase]{Linux Biolinum O} 
\usepackage{microtype}

\usepackage{graphicx}
	\graphicspath{%
	{/Users/goby/Pictures/teach/466/lectures/}%
	{img/}}%
%	{/Users/goby/Pictures/teach/common/}} % set of paths to search for images

%\usepackage{amsmath,amssymb}

%\usepackage{units}

%\usepackage{booktabs}
\usepackage{multicol}
%	\setlength{\columnsep=1em}

\usepackage{array}
\newcolumntype{L}[1]{>{\raggedright\let\newline\\\arraybackslash\hspace{0pt}}p{#1}}
\newcolumntype{C}[1]{>{\centering\let\newline\\\arraybackslash\hspace{0pt}}p{#1}}
\newcolumntype{R}[1]{>{\raggedleft\let\newline\\\arraybackslash\hspace{0pt}}p{#1}}

%usepackage{tikz}
%	\tikzstyle{every picture}+=[remember picture,overlay]

\usepackage{forest}
\usetikzlibrary{trees}
\tikzstyle{block} = [rectangle, draw, fill=white, rounded corners, minimum size=2em]
\tikzstyle{branch} = [thick, draw]

%\usetikzlibrary{positioning, backgrounds}


\forestset{
	every leaf node/.style={
		if n children=0{#1}{}
	},
	every tree node/.style={
		if n children=0{}{#1}
	},
	mytree/.style={
		for tree={
			font=\footnotesize\selectfont,
			edge path={
				\noexpand\path [draw, thick, \forestoption{edge}] (!u.parent anchor) |- (.child anchor)\forestoption{edge label};
			},
			every tree node={draw=none,inner sep=0, outer sep=0, minimum size=0},
			%every leaf node/.style={align=left},
			grow'=0,
			parent anchor=east, 
			child anchor=west,
			anchor=west,
			l = 1cm,
			%l sep=1.5cm,
			s sep=0mm,
			draw=none,
			if n children=0{tier=word}{}
		}
	}
}

\mode<presentation>
{
  \usetheme{Lecture}
  \setbeamercovered{invisible}
  \setbeamertemplate{items}[square]
}

%\usepackage{calc}
\usepackage{hyperref}
\usepackage{color}

% shortstack needed to highlight across \\ line break.
\newcommand\sshighlight[1]{%
	\highlight{\shortstack[l]{#1}}%
}

\newcommand{\backoneline}{\vspace{-\baselineskip}}

\begin{document}



{
\usebackgroundtemplate{\includegraphics[width=\paperwidth]{taxonomy_passerea}}
{	\tikzstyle{every picture}+=[remember picture,overlay]
\definecolor{orange5}{HTML}{F16913}
\begin{frame}[b, plain]

\begin{tikzpicture}


\draw[ultra thick, orange5] (0.9,7.3) rectangle (4.8,7.7);

\draw [<-, orange5, ultra thick] (4.9,7.5) -- (6.5, 7.5) node[minimum width=2cm, align=left, right] {Piciformes};

\draw[ultra thick, orange5] (0.9,8.7) rectangle (4.8,9.3);

\draw [<-, orange5, ultra thick] (4.9,9.0) -- (6.5, 9.0) node[right] {Passeriformes};

\node[minimum width=2cm, align=left] at (8,5.25) {Passeriformes is divided
into\\ Suboscine and Oscine clades.};

\end{tikzpicture}
	\tiny\hfill Jarvis et al. 2014. Science 346: 1320.
\end{frame}
}}


\begin{frame}{Piciformes: \highlight{Picidae} — woodpeckers.}

More to come\dots
\end{frame}


\begin{frame}{Downy and Hairy Woodpecker identified with practice.}

\vspace{-\baselineskip}

\begin{multicols}{2}
\reflectbox{\includegraphics[width=\linewidth]{taxonomy_dowo_hawo_feeder}}

\columnbreak

Downy Woodpecker is smaller, about the size of a House Sparrow.

\bigskip

Hairy Woodpecker is larger, about the size of an American Robin.

\end{multicols}



\tinyfill

\href{https://www.birdsandblooms.com/birding/bird-species/tell-difference-downy-hairy-woodpeckers/}{Hairy and Downy Woodpecker: Marie Read, \textcopyright\,Birds and Blooms} 
\end{frame}

\begin{frame}

\includegraphics[width=\linewidth]{taxonomy_dowo_hawo_compare}

\vspace{-\baselineskip}

\begin{multicols}{2}

Downy Woodpecker

\medskip

Short bill about 1/3 of head length.

\smallskip

Small spots in outer tail feathers.

\smallskip

Red patch on males usually undivided.


\columnbreak

Hairy Woodpecker

\medskip

Longer bill about length of head.

\smallskip

No spots on outer tail feathers.

\smallskip

Red patch on males usually divided.

\end{multicols}


\vfilll

\tiny \href{https://macaulaylibrary.org/asset/47227441}{Downy Woodpecker: Evan Lipton, ML47227441} \hfill \href{https://macaulaylibrary.org/asset/25034271}{Hairy Woodpecker: Jean-Sébastien Guénette, ML25034271}

\end{frame}




\begin{frame}{Suboscine birds do not learn their song.}

The only suboscines in North America are the Tyrant Flycatchers.

\end{frame}


\begin{frame}{Accipitriformes: \highlight{Cathartidae} — vultures}
\includegraphics[width=0.49\linewidth]{taxonomy_tuvu}\hfill
\includegraphics[width=0.49\linewidth]{taxonomy_blvu}

Turkey Vulture (left) has a red head. Black Vulture (right) has a dark gray head. \emph{Head color is not always visible at a distance or in flight.}

\vspace{0.5\baselineskip}

Head lacks plumage in adults. Feet are weakly raptorial. Bill moderately hooked.

\vfilll

\tiny Turkey Vulture:   \href{https://macaulaylibrary.org/asset/109079321}{Michael S. Taylor, ML109079321}\hfill
Black Vulture: \href{https://commons.wikimedia.org/wiki/File:Vulture,_Black_FG1.jpg}{Dick Daniels, \ccbysa{3.0}}
\end{frame}





\begin{frame}{Black Vulture (top) and Turkey Vulture (bottom) can be identified in flight.}
\includegraphics[width=\linewidth]{taxonomy_tuvu_blvu}

Black Vulture is distinctly smaller with much shorter tails. 

\vspace{0.5\baselineskip}

BLVU flap more frequently. TUVU are “tippy” in flight.
\vspace{0.5\baselineskip}

BLVU has white patches restricted to wing tips. TUVU wings look black or black coverts and silver-gray primaries and secondaries.

\tinyfill \href{https://www.nps.gov/articles/netn-species-spotlight-vultures.htm}{National Park Service, Public Domain.}
\end{frame}


\begin{frame}{Osprey (Pandionidae) is found near open water with fish.}
\includegraphics[width=\linewidth]{taxonomy_ospr}

\vfilll

\tiny \href{https://flickr.com/photos/acrylicartist/6208785891}{Rodney Cambell, \ccby{2.0}} \hfill \href{https://flickr.com/photos/usfwsnortheast/51416053070}{USFWS, Public Domain}
\end{frame}



\begin{frame}{Accipitriformes: \highlight{Accipitridae} — hawks, eagles, and kites.}

\begin{forest} mytree
[,l=1cm
	[
		[
			[
			[
				[
					[“Buteos”, align=left]
					[Mississippi Kite, align=left]
				]
				[Bald Eagle, align=left]
				]
			[, name = accipiter, 
				[Cooper's Hawk]
				[Northern Harrier]
			]
		]
		]
	]
]
\end{forest}

\begin{tikzpicture}[remember picture, overlay]
\draw [thick] (8.5,0.5) -- (8.5,1.5) node[right, midway] {Accipitrinae};;

\draw [thick] (8.5,1.7) -- (8.5,3.4) node[right, midway] {Buteoninae};;

\end{tikzpicture}



\end{frame}

\begin{frame}{Accipitrinae include Northern Harrier and Cooper's Hawk.}
\backoneline

\begin{multicols}{2}
\includegraphics[width=\linewidth]{taxonomy_noha} \hfill
\includegraphics[width=\linewidth]{taxonomy_coha}

\end{multicols}

\vfilll

\tiny Northern Harrier: \href{https://flickr.com/photos/52450054@N04/52848994837}{Judy Gallagher (top)}, \href{https://flickr.com/photos/jwcolo/8714842140}{Jonathan Wisner (bottom)}, \ccby{2.0} \hfill Cooper's Hawk: \href{https://flickr.com/photos/pazzani/36473380835}{Mike's Birds, \ccbysa{2.0}}

\end{frame}

{
\usebackgroundtemplate{\includegraphics[width=\paperwidth]{taxonomy_baea_adult}}
\begin{frame}{\textcolor{white}{Bald Eagle is found near open water with fish.}}


\tinyfill \textcolor{white}{\href{https://www.flickr.com/photos/andymorffew/30742412415}{Andy Morffew, \ccby{2.0}}}
\end{frame}
}

{
\usebackgroundtemplate{\includegraphics[width=\paperwidth]{taxonomy_baea_adult2}}
\begin{frame}{\textcolor{white}{Bald Eagle takes 5 years to attain adult plumage.}}


\tinyfill \textcolor{white}{\href{https://www.flickr.com/photos/andymorffew/30742412415}{Andy Morffew, \ccby{2.0}}}
\end{frame}
}

{
\usebackgroundtemplate{\includegraphics[width=\paperwidth]{taxonomy_baea_imm}}
\begin{frame}{\textcolor{white}{Immature Bald Eagle is mottled brown.}}


\tinyfill \textcolor{white}{\href{https://www.flickr.com/photos/andymorffew/46931504035}{Andy Morffew, \ccby{2.0}}}
\end{frame}
}

{
\usebackgroundtemplate{\includegraphics[width=\paperwidth]{taxonomy_baea_flat}}
\begin{frame}{\textcolor{white}{Bald Eagle soars with wings held flat.}}


\tinyfill \textcolor{white}{\href{https://www.flickr.com/photos/andymorffew/28687274768}{Andy Morffew, \ccby{2.0}}}
\end{frame}
}

{
\usebackgroundtemplate{\includegraphics[width=\paperwidth]{taxonomy_miki_perched}}
\begin{frame}{Mississippi Kite is a small, gray raptor.}


\tinyfill \textcolor{white}{\href{https://flickr.com/photos/wildreturn/28138609132}{Andy Reago \& Chrissy McClarren, \ccby{2.0}}}
\end{frame}
}

{
\usebackgroundtemplate{\includegraphics[width=\paperwidth]{taxonomy_miki_kiting}}
\begin{frame}{\textcolor{white}{Mississippi Kite feeds on insects in the air.}}


\tinyfill \textcolor{white}{\href{https://flickr.com/photos/wildreturn/28138609132}{Andy Reago \& Chrissy McClarren, \ccby{2.0}}}
\end{frame}
}


{
\usebackgroundtemplate{\includegraphics[width=\paperwidth]{taxonomy_miki_chased}}
\begin{frame}{\textcolor{white}{Watch for the white patches on top of the wings.}}


\tinyfill \textcolor{white}{Common Grackle chasing Mississippi Kite, \href{https://flickr.com/photos/wildreturn/28138609132}{Andy Reago \& Chrissy McClarren, \ccby{2.0}}}
\end{frame}
}

{
\usebackgroundtemplate{\includegraphics[width=\paperwidth]{taxonomy_rtha_back}}
\begin{frame}{\textcolor{white}{Red-tailed Hawk is our largest “buteo.”}}


\tinyfill Red-tailed Hawk, \href{https://flickr.com/photos/oregondot/5396663758}{Oregon Dept.~of Transportation, \ccby{2.0}}
\end{frame}
}

{
\usebackgroundtemplate{\includegraphics[width=\paperwidth]{taxonomy_rtha_belly}}
\begin{frame}


\tinyfill Red-tailed Hawk, \href{https://macaulaylibrary.org/asset/24238641}{Alex Lamoreaux, ML24238641}
\end{frame}
}


{
\usebackgroundtemplate{\includegraphics[width=\paperwidth]{taxonomy_rtha_flying}}
\begin{frame}{\textcolor{white}{Note the dark head and leading wing edge, light breast.}}


\tinyfill \textcolor{white}{Red-tailed Hawk, \href{https://macaulaylibrary.org/asset/48674511}{Jonathan Eckerson, ML48674511}}
\end{frame}
}

%{
%\usebackgroundtemplate{\includegraphics[width=\paperwidth]{taxonomy_rsha}}
%\begin{frame}{Red-shouldered Hawk is medium-sized with reddish breast.}
%
%
%\tinyfill Red-shouldered Hawk, \href{https://macaulaylibrary.org/asset/76309941}{Eric Keith, ML76309941}
%\end{frame}
%}

{
\usebackgroundtemplate{\includegraphics[width=\paperwidth]{taxonomy_rsha1}}
\begin{frame}{Red-shouldered Hawk has slender build with long legs.}


\tinyfill Red-shouldered Hawk, \href{https://macaulaylibrary.org/asset/76309941}{Eric Keith, ML76309941}
\end{frame}
}

{
\usebackgroundtemplate{\includegraphics[width=\paperwidth]{taxonomy_rsha_flying1}}
\begin{frame}


\tinyfill \textcolor{white}{Red-shouldered Hawk, \href{https://macaulaylibrary.org/asset/86154661}{Paul Fenwick, ML86154661}}
\end{frame}
}

{
\usebackgroundtemplate{\includegraphics[width=\paperwidth]{taxonomy_rsha_flying2}}
\begin{frame}


\tinyfill Red-shouldered Hawk, \href{https://flickr.com/photos/cuatrok77/8535844534}{cuatrok77, \ccbysa{2.0}}
\end{frame}
}


{
\usebackgroundtemplate{\includegraphics[width=\paperwidth]{taxonomy_strigidae}}
\begin{frame}{Strigiformes: \highlight{Strigidae} — owls are mostly nocturnal.}


\vfilll 



\tiny \colorbox[gray]{0.9}{\parbox{0.3\textwidth}{Great Horned Owl,  \href{https://flickr.com/photos/12463666@N03/32547241817}{Robert Miller, \ccby{2.0}}}}
 \hfill Barred Owl, \href{https://flickr.com/photos/79452129@N02/30833383633}{Fyn Kynd, \ccby{2.0}}
\end{frame}
}


{
\usebackgroundtemplate{\includegraphics[width=\paperwidth]{taxonomy_strigidae_less_common}}

\begin{frame}{Barn Owl is family Tytonidae. All others in Strigidae.}

\begin{tikzpicture}[remember picture, overlay]

\node[minimum width=5.5cm, align=right] at (2.8,-3.5) {\textcolor{white}{\tiny Eastern Screech Owl:   \href{https://flickr.com/photos/wildreturn/30621282464}{Andy Reago \& Chrissy McClarren, \ccby{2.0}}}};

%\node[minimum width=5.5cm, align=right, color=white] at (10.4,-3.5) {\tiny Bald Eagle:   \href{https://flickr.com/photos/jamham/4317710020}{Jim Hamilton, Public Domain}};

\end{tikzpicture}


\vfilll

\tiny \textcolor{white}{Short-eared Owl: \href{https://flickr.com/photos/128941223@N02/49978078296}{Caroline Legg, \ccby{2.0}}} \hfill Barn Owl: \href{https://macaulaylibrary.org/asset/155550701}{Sharif Udden, ML155550701}
\end{frame}
}

{
\usebackgroundtemplate{\includegraphics[width=\paperwidth]{taxonomy_pefa}}
\begin{frame}{Falconiformes: Falconidae — falcons.}


\tinyfill Peregrin Falcon, \href{https://flickr.com/photos/79452129@N02/21571335442}{Fyn Kynd, \ccby{2.0}}
\end{frame}
}

{
\usebackgroundtemplate{\includegraphics[width=\paperwidth]{taxonomy_amke}}
\begin{frame}{American Kestrel is commonly seen near open fields.}

\begin{tikzpicture}[remember picture, overlay]

\node[minimum width=5.5cm, align=right] at (8.8,-3.5) {\textcolor{white}{Males have blue-gray wings.}};

%\node[minimum width=5.5cm, align=right, color=white] at (10.4,-3.5) {\tiny Bald Eagle:   \href{https://flickr.com/photos/jamham/4317710020}{Jim Hamilton, Public Domain}};

\end{tikzpicture}


\tinyfill \textcolor{white}{Male American Kestrel, %
 \href{https://commons.wikimedia.org/wiki/File:American_Kestrel_(Male)_(8238822396).jpg}{Andy Morffew, \ccby{2.0}}}
\end{frame}
}

{
\usebackgroundtemplate{\includegraphics[width=\paperwidth]{taxonomy_amke_female}}
\begin{frame}{American Kestrel often bob its tail when perched.}

\begin{tikzpicture}[remember picture, overlay]

\node[minimum width=5.5cm, align=right] at (8.8,-1.5) {\textcolor{black}{Females have rufous wings.}};

%\node[minimum width=5.5cm, align=right, color=white] at (10.4,-3.5) {\tiny Bald Eagle:   \href{https://flickr.com/photos/jamham/4317710020}{Jim Hamilton, Public Domain}};

\end{tikzpicture}


\tinyfill Female American Kestrel, %
 \href{https://flickr.com/photos/jls195674/16558905212}{John Schmoll, \ccby{2.0}}
\end{frame}
}

{
\usebackgroundtemplate{\includegraphics[width=\paperwidth]{taxonomy_amke_flying}}
\begin{frame}{American Kestrel can hover in flight.}

\begin{tikzpicture}[remember picture, overlay]

\node[minimum width=5.5cm, align=left] at (2.8,-1.5) {Watch for pointed wings\\ and long slender tail.};

%\node[minimum width=5.5cm, align=right, color=white] at (10.4,-3.5) {\tiny Bald Eagle:   \href{https://flickr.com/photos/jamham/4317710020}{Jim Hamilton, Public Domain}};

\end{tikzpicture}


\tinyfill American Kestrel, %
 \href{https://commons.wikimedia.org/wiki/File:American_Kestrel_Diving_(16114435431).jpg}{Andy Morffew, \ccby{2.0}}
\end{frame}
}

\end{document}
