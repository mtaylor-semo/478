%!TEX TS-program = lualatex
%!TEX encoding = UTF-8 Unicode

\documentclass[t]{beamer}

%%%% HANDOUTS For online Uncomment the following four lines for handout
%\documentclass[t,handout]{beamer}  %Use this for handouts.
%\includeonlylecture{student}
%\usepackage{handoutWithNotes}
%\pgfpagesuselayout{3 on 1 with notes}[letterpaper,border shrink=5mm]


%%% Including only some slides for students.
%%% Uncomment the following line. For the slides,
%%% use the labels shown below the command.
%\includeonlylecture{student}

%% For students, use \lecture{student}{student}
%% For mine, use \lecture{instructor}{instructor}


%\usepackage{pgf,pgfpages}
%\pgfpagesuselayout{4 on 1}[letterpaper,border shrink=5mm]

% FONTS
\usepackage{fontspec}
\def\mainfont{Linux Biolinum O}
\setmainfont[Ligatures=TeX, BoldFont={* Bold}, ItalicFont={* Italic}, Numbers={Proportional}]{\mainfont}
\setmonofont[Scale=MatchLowercase]{Linux Libertine Mono O} 
\setsansfont[Scale=MatchLowercase]{Linux Biolinum O} 
\usepackage{microtype}

\usepackage{graphicx}
	\graphicspath{%
	{/Users/goby/Pictures/teach/466/lectures/}%
	{img/}}%
%	{/Users/goby/Pictures/teach/common/}} % set of paths to search for images

%\usepackage{amsmath,amssymb}

%\usepackage{units}

%\usepackage{booktabs}
\usepackage{multicol}
%	\setlength{\columnsep=1em}

\usepackage{array}
\newcolumntype{L}[1]{>{\raggedright\let\newline\\\arraybackslash\hspace{0pt}}p{#1}}
\newcolumntype{C}[1]{>{\centering\let\newline\\\arraybackslash\hspace{0pt}}p{#1}}
\newcolumntype{R}[1]{>{\raggedleft\let\newline\\\arraybackslash\hspace{0pt}}p{#1}}

\usepackage{tikz}
	\tikzstyle{every picture}+=[remember picture,overlay]


\mode<presentation>
{
  \usetheme{Lecture}
  \setbeamercovered{invisible}
  \setbeamertemplate{items}[square]
}

%\usepackage{calc}
\usepackage{hyperref}
\usepackage{color}

% shortstack needed to highlight across \\ line break.
\newcommand\sshighlight[1]{%
	\highlight{\shortstack[l]{#1}}%
}

\newcommand{\backoneline}{\vspace{-\baselineskip}}

\begin{document}



{
	\usebackgroundtemplate{\includegraphics[width=\paperwidth]{taxonomy_passerea}}
	{	\tikzstyle{every picture}+=[remember picture,overlay]
		\definecolor{orange5}{HTML}{F16913}
		\begin{frame}[b, plain]
			
			\begin{tikzpicture}
				
				
				%\draw[ultra thick, orange5] (0.9,7.3) rectangle (4.8,7.7);
				
				%\draw [<-, orange5, ultra thick] (4.9,7.5) -- (6.5, 7.5) node[minimum width=2cm, align=left, right] {Piciformes};
				
				\draw[ultra thick, orange5] (0.9,8.7) rectangle (4.8,9.3);
				
				\draw [<-, orange5, ultra thick] (4.9,9.0) -- (6.5, 9.0) node[right] {Passeriformes};
				
			\end{tikzpicture}
			\tiny\hfill Jarvis et al. 2014. Science 346: 1320.
		\end{frame}
}}



\begin{frame}{Passeriformes: \highlight{Fringillidae} — finches.}

		\includegraphics[width=0.49\linewidth]{taxonomy_amgo_male}\hfill
		\includegraphics[width=0.49\linewidth]{taxonomy_amgo_female}
	
		\backoneline
		
		\begin{multicols}{2}
			American Goldfinch (male, summer)
			
			\columnbreak
			
			American Goldfinch (female)
			
		\end{multicols}	

			In flight, watch for undulating flight (flap-bounding) and “po-ta-to chip” call. 
			
			\medskip
			
			Often in small flocks. Males in winter and immatures look like females. Watch for similar Pine Siskin in winter.
	
	\vfilll
	
	\tiny
	
	\href{https://commons.wikimedia.org/wiki/File:American_goldfinch_in_PP_(72124).jpg}{American Goldfinch male, Rhododendrites, \ccbysa{4.0}} \hfill 
	\href{https://macaulaylibrary.org/asset/410324591}{American Goldfinch female, Matt Mason, \textsc{ml}410324591}

\end{frame}

%

\begin{frame}{Passeriformes: \highlight{Fringillidae} — finches.}
	
	\includegraphics[width=0.49\linewidth]{taxonomy_hofi_male}\hfill
	\includegraphics[width=0.49\linewidth]{taxonomy_hofi_female}
	
	\backoneline
	
	\begin{multicols}{2}
		House Finch (male)
		
		\columnbreak
		
		House Finch (female)
		
	\end{multicols}	
	
	Intensity of reddish head and breast in mature males is a “honest indicator” determined by diet.
	
	\medskip
	
	Native to western U.S. Often in small to large flocks. Immatures look like females. Watch for similar Purple Finch in winter.
	
	\vfilll
	
	\tiny
	
	\href{https://macaulaylibrary.org/asset/283643061}{House Finch male, Martina Nordstrand, \textsc{ml}283643061} \hfill 
	\href{https://macaulaylibrary.org/asset/136083061}{House Finch female, Jonathan Irons, \textsc{ml}136083061}
	
\end{frame}

%

\begin{frame}{Passeriformes: \highlight{Passerellidae} — New World sparrows.}
	
		\includegraphics[width=0.49\linewidth]{taxonomy_eato_female}\hfill
		\includegraphics[width=0.49\linewidth]{taxonomy_eato_male}
	
		\backoneline
		
		\begin{multicols}{2}
				Eastern Towhee (female)
				
				\medskip
				
				Note brown head, breast, and back.
				
				\columnbreak
				
				Eastern Towhee (male)
				
				\medskip
				
				Note black head, breast, and back.
				
			\end{multicols}	
		
		Cardinal-sized sparrow. Forages on the ground like other sparrows.
		
		\medskip
		
		Formerly Emberizidae.
		
		\vfilll
		
		\tiny
		
		\href{https://macaulaylibrary.org/asset/40901041}{Eastern Towhee female, Mark R.~Johnson, \textsc{ml}40901041} \hfill 
		\href{https://macaulaylibrary.org/asset/27668311}{Eastern Towhee male, Davey Walters, \textsc{ml}27668311}
	
\end{frame}

%



\begin{frame}{Most sparrows require practice to identify.}
	
	\includegraphics[width=0.49\linewidth]{taxonomy_chsp}\hfill
	\includegraphics[width=0.49\linewidth]{taxonomy_fisp}
	
	\backoneline
	
	\begin{multicols}{2}
		Chipping Sparrow
		
		\medskip
		
		Note the rusty cap, clean white supercilium, and unstreaked breast.
		
		\columnbreak
		
		Field Sparrow
		
		\medskip
		
		Note the pink bill and legs. Eye ring gives big-eyed appearance.
		
	\end{multicols}	
	
	These are are two common summer sparrows.
	
	\vfilll
	
	\tiny
	
	\href{https://macaulaylibrary.org/asset/27961971}{Chipping Sparrow, Evan Lipton, \textsc{ml}27961971} \hfill 
	\href{https://macaulaylibrary.org/asset/248917631}{Field Sparrow, Ryan Sanderson, \textsc{ml}248917631}
	
\end{frame}

%

\begin{frame}{Two uncommon sparrows in prairie and grassland habitat.}
	
	\includegraphics[width=0.49\linewidth]{taxonomy_lasp}\hfill
	\includegraphics[width=0.49\linewidth]{taxonomy_grsp}
	
	\backoneline
	
	\begin{multicols}{2}
		Lark Sparrow
		
		\medskip
		
		Distinct pattern on head, dark spot on clean breast.
		
		\columnbreak
		
		Grasshopper Sparrow
		
		\medskip
		
		Blocky, flat-topped head with buffy highlights on face and breast.
		
	\end{multicols}	
	
	Two uncommon summer sparrows that nest at Sand Prairie CA.
	
	\vfilll
	
	\tiny
	
	\href{https://macaulaylibrary.org/asset/351375121}{Lark Sparrow, Annie McLeod, \textsc{ml}351375121} \hfill 
	\href{https://macaulaylibrary.org/asset/240593251}{Grasshopper Sparrow, Marky Mutchler, \textsc{ml}240593251}
	
\end{frame}

%

\begin{frame}{Two common winter-resident sparrows.}
	
	\includegraphics[width=0.49\linewidth]{taxonomy_wtsp}\hfill
	\includegraphics[width=0.49\linewidth]{taxonomy_deju}
	
	\backoneline
	
	\begin{multicols}{2}
		
		White-throated Sparrow

		\medskip
		
		Note the white throat and yellow lores.
						
		\columnbreak
		
		Dark-eyed Junco
		
		\medskip
		
		Dark, slaty gray with pink bill.
		
	\end{multicols}	
	
	Other winter sparrows include American Tree, Fox, Savannah, Song, Swamp, Vesper, and White-crowned.
	
	\vfilll
	
	\tiny
	
	\href{https://macaulaylibrary.org/asset/63894671}{White-throated Sparrow, Keenan Yakola, \textsc{ml}63894671} \hfill 
	\href{https://macaulaylibrary.org/asset/47337561}{Dark-eyed Junco, Scott Martin, \textsc{ml}47337561}
	
\end{frame}

%

\begin{frame}{Passeriformes: \highlight{Icteridae} — blackbirds, grackles, and orioles.}
	
		\includegraphics[width=0.49\linewidth]{taxonomy_rwbl_female}\hfill
		\includegraphics[width=0.49\linewidth]{taxonomy_rwbl_male}
	
		\backoneline
		
		\begin{multicols}{2}
				Red-winged Blackbird (female)
								
				\columnbreak
				
				Red-winged Blackbird (male)
				
			\end{multicols}	
		
		Abundant birds that nest around wet areas.
		
		\vfilll
		
		\tiny
		
		\href{https://macaulaylibrary.org/asset/172808941}{Red-winged Blackbird female, Connor Charchuk, \textsc{ml}172808941} \hfill 
		\href{https://macaulaylibrary.org/asset/158001791}{Red-winged Blackbird male, Connor Charchuk, \textsc{ml}158001791}
	
\end{frame}

%

\begin{frame}{Passeriformes: \highlight{Icteridae} — blackbirds, grackles, and orioles.}
	
	\includegraphics[width=0.49\linewidth]{taxonomy_cogr}\hfill
	\includegraphics[width=0.49\linewidth]{taxonomy_bhco}
	
	\backoneline
	
	\begin{multicols}{2}
		Common Grackle is our largest blackbird.
		
		\medskip
		
		Females similar but drab.
		
		\columnbreak
		
		Brown-headed Cowbird is our smallest blackbird.
		
		\medskip
		
		Females uniform light brown.
		
	\end{multicols}	
	
	\vfilll
	
	\tiny
	
	\href{https://macaulaylibrary.org/asset/93138231}{Common Grackle, Jack \& Holly Bartholmai, \textsc{ml}93138231} \hfill 
	\href{https://macaulaylibrary.org/asset/84247171}{Brown-headed Cowbird, Jack \& Holly Bartholmai, \textsc{ml}84247171}
	
\end{frame}

%

\begin{frame}{These blackbirds are declining due to habitat loss.}
	
	\includegraphics[width=0.49\linewidth]{taxonomy_eame}\hfill
	\includegraphics[width=0.49\linewidth]{taxonomy_bobo}
	
	\backoneline
	
	\begin{multicols}{2}
		Eastern Meadowlark.
		
		\medskip
		
		Often visible sitting on fence rows and electrical wires near open fields.
				
		\columnbreak
		
		Bobolink.
				
	\end{multicols}	
	
	\vfilll
	
	\tiny
	
	\href{https://macaulaylibrary.org/asset/52473861}{Eastern Meadowlark, Doris Brookens, \textsc{ml}52473861} \hfill 
	\href{https://macaulaylibrary.org/asset/250241071}{Bobolink, Ryan Sanderson, \textsc{ml}250241071}
	
\end{frame}

%

\begin{frame}{Orioles sing and forage near the tops of tall trees.}
	
	\includegraphics[width=0.49\linewidth]{taxonomy_oror}\hfill
	\includegraphics[width=0.49\linewidth]{taxonomy_baor}
	
	\backoneline
	
	\begin{multicols}{2}
		Orchard Oriole.
		

		\columnbreak
		
		Baltimore Oriole.
		
	\end{multicols}	
	
	Both are found in open woodlands and groves or near wood edges.
	
	\vfilll
	
	\tiny
	
	\href{https://macaulaylibrary.org/asset/238871541}{Orchard Oriole, Bryan Calk, \textsc{ml}238871541} 
	\hfill 
	\href{https://macaulaylibrary.org/asset/131880391}{Baltimore Oriole, Fernando Burgalin Sequeria, \textsc{ml}131880391} 
	
\end{frame}

%
\begin{frame}{Passeriformes: \highlight{Parulidae} — New World warblers.}
	
	\begin{tabular}{ccc}
	\includegraphics[width=0.3\textwidth]{taxonomy_prow} &
	\includegraphics[width=0.3\textwidth]{taxonomy_coye} &
	\includegraphics[width=0.3\textwidth]{taxonomy_kewa} \tabularnewline
	Prothonotary Warbler & Common Yellowthroat & Kentucky Warbler \tabularnewline[1em]
	\includegraphics[width=0.3\textwidth]{taxonomy_nopa} &
	\includegraphics[width=0.3\textwidth]{taxonomy_yewa} &
	\includegraphics[width=0.3\textwidth]{taxonomy_ytwa} \tabularnewline
	Northern Parula &
	%\includegraphics[width=0.3\textwidth]{taxonomy_suta} &
	Yellow Warbler &
	%\includegraphics[width=0.3\textwidth]{taxonomy_suta} \tabularnewline
	Yellow-throated Warbler \tabularnewline
\end{tabular}

	
	\vfilll
	
	\tiny
	
	\href{https://macaulaylibrary.org/asset/226365711}{Ryan Sanderson, \textsc{ml}226365711} \hfill 
	\href{https://macaulaylibrary.org/asset/249203731}{Ryan Sanderson, \textsc{ml}249203731} \hfill
	\href{https://macaulaylibrary.org/asset/229562501}{Brad Imhoff, \textsc{ml}229562501}

	\href{https://macaulaylibrary.org/asset/40144451}{Ryan Schain, \textsc{ml}40144451}
	\hfill
	\href{https://macaulaylibrary.org/asset/234247081}{Brad Imhoff, \textsc{ml}234247081} 
	\hfill
	\href{https://macaulaylibrary.org/asset/225713071}{Ryan Sanderson, \textsc{ml}225713071} 
	
	
\end{frame}

%

\begin{frame}{Passeriformes: \highlight{Cardinalidae} — cardinals, tanagers, and allies.}
	
	\begin{tabular}{ccc}
		\includegraphics[width=0.3\textwidth]{taxonomy_scta} &
		\includegraphics[width=0.3\textwidth]{taxonomy_suta} &
		\includegraphics[width=0.3\textwidth]{taxonomy_noca} \tabularnewline
		Scarlet Tanager & Summer Tanager & Northern Cardinal \tabularnewline[1em]
		\includegraphics[width=0.3\textwidth]{taxonomy_rbgr} &
		\includegraphics[width=0.3\textwidth]{taxonomy_inbu} &
		\includegraphics[width=0.3\textwidth]{taxonomy_dick} \tabularnewline
		Rose-breasted Grosbeak &
		Indigo Bunting &
		Dickcissel \tabularnewline
	\end{tabular}
		
		\vfilll
		
		\tiny
		
		\href{https://flickr.com/photos/usfwsnortheast/50109221467}{USFWS, Public Domain} \hfill 
		\href{https://flickr.com/photos/usfwsnortheast/51177620670}{USFWS, Public Domain} \hfill
		\href{https://macaulaylibrary.org/asset/257386271}{Suzie McCann, \textsc{ml}257386271}
		
		\href{https://macaulaylibrary.org/asset/56385541}{Tom Snow, \textsc{ml}56385541}
		\hfill
		\href{https://www.flickr.com/photos/43322816@N08/51385513517}{USFWS, Public Domain} 
		\hfill
		\href{https://macaulaylibrary.org/asset/232134471}{Martina Nordstrand, \textsc{ml}232134471} 
		
	

	
\end{frame}

%

\begin{frame}{Two species of tanagers are common here.}
	
		\includegraphics[width=0.49\linewidth]{taxonomy_scta}\hfill
		\includegraphics[width=0.49\linewidth]{taxonomy_suta}
	
		\backoneline
		
		\begin{multicols}{2}
				Scarlet Tanager
				
				\medskip
				
				Watch for red body with black wings, stout bill.
				
				\smallskip
				
				Common in deciduous woods.
				
				
				\columnbreak
				
				Summer Tanager
				
				\medskip
				
				Watch for all red body and wings, stout bill.
				
				\smallskip
				
				Common in open woods.
				
			\end{multicols}	
		
		\vfilll
		
		\tiny
		
		\href{https://flickr.com/photos/usfwsnortheast/50109221467}{Scarlet Tanager, USFWS, Public Domain} \hfill 
		\href{https://flickr.com/photos/usfwsnortheast/51177620670}{Summer Tanager, USFWS, Public Domain}
	
\end{frame}

%

\begin{frame}{Our other blue birds\dots}
	
	\includegraphics[width=0.49\linewidth]{taxonomy_inbu}\hfill
	\includegraphics[width=0.49\linewidth]{taxonomy_blgr}
	
	\backoneline
	
	\begin{multicols}{2}
		Indigo Bunting
		
		\medskip
		
		Sparrow-sized, electric blue bird.\newline
		
		
		\columnbreak
		
		Blue Grosbeak
		
		\medskip
		
		Cardinal-sized blue bird with chestnut shoulder patches.
				
	\end{multicols}	
	
	\vfilll
	
	\tiny
	
	\href{https://www.flickr.com/photos/43322816@N08/51385513517}{Indigo Bunting, USFWS, Public Domain}  \hfill 
	\href{https://macaulaylibrary.org/asset/41644021}{Blue Grosbeak, Cliff Peterson, \textsc{ml}41644021}
	
\end{frame}

\begin{frame}{Dickcissel can be abundant around open fields.}
	
	\backoneline
	
	\begin{multicols}{2}
		Watch for yellow breast with black V on throat and upper breast,
		
		\smallskip
		
		Chestnut shoulders.
		
		\smallskip
		
		Large, conical bill.
		
		\smallskip
		
		Resembles sparrow but no sparrow has yellow on breast.
		
		\columnbreak
		
		\includegraphics[width=\linewidth]{taxonomy_dick}
		
	\end{multicols}

	\tinyfill \href{https://macaulaylibrary.org/asset/232134471}{Dickcissel, Martina Nordstrand, \textsc{ml}232134471} 	
\end{frame}


\lecture{instructor}{instructor}

\begin{frame}{Pull out your index card and field guide.}
	
	\hangpara You may use your notes and field guide to answer the identification question on each slide.

	\hangpara Write "Order", "Family", "Species" on the first line of the card, spread across to make three columns.
	
	\hangpara Write all information for one slide on \textit{one} line of your card, one part in each column.
	
	\hangpara One slide, one line.
	
	\hangpara You may not work together or share information.
	

\end{frame}


{
	\usebackgroundtemplate{\includegraphics[width=\paperwidth]{taxonomy7_blbw}}
	\begin{frame}[t]{\textcolor{white}{Pg.\,336: Name the order, family, and identify the species.}}
		
		\tinyfill  \textcolor{white}{\href{https://macaulaylibrary.org/asset/168328191}{Ezra Staengl, ML168328191}}
	\end{frame}
}

{
	\usebackgroundtemplate{\includegraphics[width=\paperwidth]{taxonomy7_wcsp}}
	\begin{frame}[t]{\textcolor{black}{Pg.\,366: Name the order, family, and identify the species.}}
		
		\tinyfill  \textcolor{white}{\href{https://macaulaylibrary.org/asset/45363901}{Michel Bourque, ML45363901}}
	\end{frame}
}

{
	\usebackgroundtemplate{\includegraphics[width=\paperwidth]{taxonomy7_pabu}}
	\begin{frame}[t]{\textcolor{white}{Pg.\,390: Name the order, family, and identify the species.}}
		
		\tinyfill  \textcolor{white}{\href{https://macaulaylibrary.org/asset/26622721}{David Hollie, ML26622721}}
	\end{frame}
}


{
	\usebackgroundtemplate{\includegraphics[width=\paperwidth]{taxonomy7_yhbl}}
	\begin{frame}[t]{\textcolor{white}{Pg.\,399: Name the order, family, and identify the species.}}
		
		\tinyfill  \textcolor{black}{\href{https://macaulaylibrary.org/asset/296937341}{Ian Davies, ML296937341}}
	\end{frame}
}


{
	\usebackgroundtemplate{\includegraphics[width=\paperwidth]{taxonomy7_pisi}}
	\begin{frame}[t]{\textcolor{white}{Pg.\,412: Name the order, family, and identify the species.}}
		
		\tinyfill  \textcolor{white}{\href{https://macaulaylibrary.org/asset/295022341}{David M.\,Bell, ML295022341}}
	\end{frame}
}



\end{document}
